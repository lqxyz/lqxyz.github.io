\documentclass[a4paper, 1pt]{article}

%%%%%% 导入包 %%%%%%
\usepackage{graphicx}
\usepackage{amsmath} 
\usepackage{xeCJK}
\usepackage{booktabs}
\usepackage{multicol}
\setlength{\columnsep}{2em}
\usepackage{setspace} % set row space
\usepackage{geometry} % 页边距
\geometry{left=2.5cm,right=2.5cm,top=2.5cm,bottom=2.5cm}
\usepackage{enumitem}% item等间距过大解决办法
\usepackage[unicode]{hyperref}
\usepackage{xcolor}
\usepackage{cite}
\usepackage{indentfirst}
\setCJKmainfont{Songti SC}
\usepackage{geometry}
\usepackage{listings}
\usepackage[T1]{fontenc}
\usepackage{courier}
\lstset{language=Python,
  backgroundcolor=\color[RGB]{245,245,244},   %代码背景色  
  basicstyle=\ttfamily\footnotesize
  =false, % 应该有bug吧,这样就不在边框上面显示字了
  breaklines=true,
  commentstyle=\color{red!50!green!50!blue!50},%浅灰色的注释
  frame=shadowbox, %把代码用带有阴影的框圈起来 
  extendedchars=false,
  lineskip=-2pt,
  keepspaces=true,
  rulesepcolor=\color{red!20!green!20!blue!20},%代码块边框为淡青色
  %backgroundcolor=\color[RGB]{198, 226, 255},   %代码背景色  
  %numbers=left,%左侧显示行号 往左靠,还可以为right,或none,即不加行号  
  %numberstyle={\color[RGB]{0,192,192}\tiny} ,%设置行号的大小,大小有tiny,scriptsize,footnotesize,small,normalsize,large等  
  keywordstyle=\color{red},
  commentstyle=\color{blue},
  morecomment=[l]{!\ }% Comment only with space after !
  %escapeinside={/*@}{@*/},  
  %escapeinside={\%*}{*)}, 
  showspaces=false,                % show spaces everywhere adding particular underscores; it overrides 'showstringspaces'
  showstringspaces=false,          % underline spaces within strings only
  showtabs=false,                  % show tabs within strings adding particular underscores
  % stepnumber=2,                    % the step between two line-numbers. If it's 1, each line will be numbered
  % stringstyle=\color{mymauve},     % string literal style
  % tabsize=2,                       % sets default tabsize to 2 spaces
  %title=\lstname
  xleftmargin=0em, xrightmargin=0em, aboveskip=1em%设置边距
}

%%%%%% 设置字号 %%%%%%
\newcommand{\chuhao}{\fontsize{42pt}{\baselineskip}\selectfont}
\newcommand{\xiaochuhao}{\fontsize{36pt}{\baselineskip}\selectfont}
\newcommand{\yihao}{\fontsize{28pt}{\baselineskip}\selectfont}
\newcommand{\erhao}{\fontsize{21pt}{\baselineskip}\selectfont}
\newcommand{\xiaoerhao}{\fontsize{18pt}{\baselineskip}\selectfont}
\newcommand{\xiaosihao}{\fontsize{12pt}{\baselineskip}\selectfont}
\newcommand{\wuhao}{\fontsize{10.5pt}{\baselineskip}\selectfont}
\newcommand{\xiaowuhao}{\fontsize{9pt}{\baselineskip}\selectfont}
\newcommand{\liuhao}{\fontsize{7.5pt}{\baselineskip}\selectfont}

%%%% 段落首行缩进两个字 %%%%
\makeatletter
\let\@afterindentfalse\@afterindenttrue
\@afterindenttrue
\makeatother
\setlength{\parindent}{2em}  %中文缩进两个汉字位

\renewcommand{\tablename}{表}

%%%% 正文开始 %%%%
\begin{document}
\begin{spacing}{1.0}

%%%% 定义标题格式,包括title,author,affiliation,email等 %%%%
\title{k-means聚类分析实验报告}
\author{\wuhao 刘群\footnote{电子邮件: liu-q14@mails.tsinghua.edu.cn,学号: 2014211591}\\[2ex]
\wuhao 地球系统科学研究中心\\
}
\date{\wuhao 2015年4月24日}
%%%% 以下部分是正文 %%%%  
\maketitle

%\tableofcontents
%\newpage
\xiaowuhao 
%\begin{multicols}{2}
\section{\xiaosihao k-means算法简介}
k-means是一种聚类分析的算法,属于聚类分析中的划分方法, 主要思想是把数据划分成k个区域,通过基于距离的启发式的迭代,找到最优的聚类方案。其主要过程是:
\begin{enumerate}
\item 从问题空间中随意选取k个点,作为初始簇的中心;
\item  重复以下过程:
\begin{enumerate}
\item 对数据集中的每个数据点,计算它到不同簇中心的距离,根据距离的比较,将该数据点分配到与其距离最近的中心点;
\item 针对每一个簇,计算簇中所有点的均值,并以均值作为新的中心点;
\end{enumerate}
\item 直到各簇的中心点不再发生变化.
\end{enumerate}

但是k-means也有一些问题,即在初始条件下我们并不知道最好要分几类,而且在k值确定的基础上,计算结果倾向于收敛到局部最小值,而不是全局最小值, 并且对于数据中的离群点非常敏感。

%---------------------Module array----------------------%
\section{\xiaosihao k-means.py程序简介}
本次作业的主要任务是利用k-means方法对Iris数据进行分类,由于这组数据里本身就含有三个类别,因此在进行分类时,我也采取了k=3来进行分类。
主要过程如下:\\
\begin{enumerate}
\item 确定k值,由于这里的测试数据本身就是来自3个类别,因此这里取$k=3$.当然,我们应该取不同的k值进行测试,但是对于不同的k值所分出的不同的类别,我们应该如何去判断优劣呢?由于这个标准不能简单的选择到中心点的距离来判断,因此这里我就暂时不对k值进行讨论。
\item 随机选择k个"中心点", 这里我采用的是最简单的选取方法,即采用python中自带的random模块产生k组随机数作为初始迭代的中心点。
\begin{lstlisting}
 center = np.array(random.sample(x_train, k))
\end{lstlisting}
\item 计算所有点到所有中心点的距离,这里我们采用的是欧式距离,
\item  求每个样本聚类的结果,在这里,我们是根据所有样本到中心点的距离来判断的。对于一个样本来说,其归属于距离自己最近的中心点,我们在这里是基于对距离的排序确定的。对于每一个样本来说,都要计算其到各个中心点的值,这样就可以生成$150\times k$的矩阵,在进行计算时,我们可以利用tile函数将数组进行扩充形成一个大矩阵,然后利用矩阵的相关操作进行运算,这样可以提高计算效率。然后我们利用python中的argmin函数,对每一行进行排序即可得到相应的类别分类。

\begin{lstlisting}
for i in range(k):
    C = np.tile(center[i,:], (150,1))
    dist[:,i]=np.sqrt(np.sum((x_train-C)**2,axis=1))

labels = np.argmin(dist, axis=1)
\end{lstlisting}
\item 计算新的中心点, 这里我们采用的方法是将某一类别的数据提取出来,分别计算器几何中心(即每个维度求各自的平均)。
\begin{lstlisting}
newcenter = np.zeros((k,4))
for i in range(k):
    newcenter[i,:] =  np.average(x_train[labels==i,:], axis=0)
\end{lstlisting}
\item 判断中心点是否发生变化, 若没有发生变则训练结束, 若有变化,重复步骤3。在进行判断时,我们引入了一个新的变量changed来判断。由于$newcenter==center$之后,我们得到的是一个bool型矩阵,因此我们又利用了$.all()$函数来判断矩阵中是否全为True,如果全为True,即新旧中心点是相同的,则停止迭代,否则就继续迭代。
\begin{lstlisting}
changed = (newcenter == center).all() == False
\end{lstlisting}
下面是完整kmeans函数。
\begin{lstlisting}
def kmeans(x_train, k):
    changed = True
    center = np.array(random.sample(x_train, k))
    dist = np.zeros((150, k))
    # Loop until the center won't change
    while changed:
        for i in range(k):
            C = np.tile(center[i,:], (150,1))
            dist[:,i]=np.sqrt(np.sum((x_train-C)**2,axis=1))

        labels = np.argmin(dist, axis=1)
        newcenter = np.zeros((k,4))
        for i in range(k):
            newcenter[i,:] =  np.average(x_train[labels==i,:], axis=0)
        changed = (newcenter == center).all() == False
        center = newcenter
    # Calculate the number of each group
    num_of_each_type = np.array([x_train[labels == i,:].shape[0] for i in range(k)])
    # Calcultate the total distance of the classification
    totalDistance = 0
    for i in range(k):
        #print sum(labels==i)
        newCenter = np.tile(center[i,:], (sum(labels==i),1))
        #print "dist"+str(np.sum((x_train[labels==i,:]-newCenter)**2,axis=1))
        totalDistance += np.sum((np.sum((x_train[labels==i,:]-newCenter)**2,axis=1)))
    # return type labels and total distance and number of each type
    return labels, np.sqrt(totalDistance), num_of_each_type
\end{lstlisting}
\item 评价聚类准确性, 在这里调用kmeans函数时,我调用了10次,取出这10次中到中心点距离最小的情形作为最终聚类的结果。
\begin{lstlisting}
# Loading data
iris = datasets.load_iris()
x_train = iris.data
y_train = iris.target

k = 3
n = 10 # num of loops
labels = {}
num = {}
totalDist = np.zeros(n)
minDist = 10**6
minIndex = 0
for j in range(n):
    labels[j], totalDist[j], num[j] = kmeans(x_train, k)
    if totalDist[j] < minDist:
        minIndex = j
        minDist = totalDist[j]
\end{lstlisting}
然后对聚类的结果和原先分好的类的结果进行对比,统计聚类结果中实际上含有的各个类别的比例。然后将计算的结果输出,输出时按照列表的形式打印出来。由于中心点刚开始是随机选择的,也不能确定其到底是属于正确分类中的哪一组,因此真实的类别名用$type0, type1, type2$来表示,而不是用真实的类别名。
\begin{lstlisting}
# Print the classification results
print "Number of samples in each claster is " + str(num[minIndex]) + "."
#print "\t\t %-15s %-15s %-15s" %("setosa", "versicolor", "virginica")
print "\t\t %-15s %-15s %-15s" %("type0", "type1", "type2")
for i in range(k):
    a = sum((y_train[labels[minIndex]==i])==0)*1.0/(x_train[labels[minIndex]==i,:]).shape[0]
    b = sum((y_train[labels[minIndex]==i])==1)*1.0/(x_train[labels[minIndex]==i,:]).shape[0]
    c = sum((y_train[labels[minIndex]==i])==2)*1.0/(x_train[labels[minIndex]==i,:]).shape[0]
    print "%8s\t %-15.4f %-15.4f %-15.4f" %("cluster"+str(i), a, b, c)
\end{lstlisting}
\end{enumerate}
下面是某次程序运行后输出的结果:
\begin{lstlisting}
Number of samples in each claster is [50 38 62].
             type0           type1           type2          
cluster0     1.0000          0.0000          0.0000         
cluster1     0.0000          0.0526          0.9474         
cluster2     0.0000          0.7742          0.2258         
\end{lstlisting}

\section{\xiaosihao 总结}
聚类分析可以将大量数据分成几类,挖掘出数据中隐藏的我们还不确定的规律和特征, 在数据分析中非常有用。但是对于初值的选择十分敏感,因此在选择初值时我们应该格外注意。但是在这里我们主要是采用了随机的方式来选择初值,这样的效果可能不是很好,但是通过10次或者更多次迭代,或许可以得到比较好的结果。
%\end{multicols}
\end{spacing}
\end{document}