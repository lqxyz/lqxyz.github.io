\documentclass[a4paper, 1pt]{article}

%%%%%% 导入包 %%%%%%
\usepackage{graphicx}
\usepackage{amsmath} 
\usepackage{xeCJK}
\usepackage{booktabs}
\usepackage{multicol}
\setlength{\columnsep}{2em}
\usepackage{setspace} % set row space
\usepackage{geometry} % 页边距
\geometry{left=2.5cm,right=2.5cm,top=2.5cm,bottom=2.5cm}
\usepackage{enumitem}% item等间距过大解决办法
%\setenumerate[1]{itemsep=0pt,partopsep=0pt,parsep=\parskip,topsep=5pt}
%\setitemize[1]{itemsep=0pt,partopsep=0pt,parsep=\parskip,topsep=2pt}
%\setdescription{itemsep=0pt,partopsep=0pt,parsep=\parskip,topsep=5pt}
\usepackage[unicode]{hyperref}
\usepackage{xcolor}
\usepackage{cite}
\usepackage{indentfirst}
\setCJKmainfont{Songti SC}
\usepackage{geometry}
\usepackage{listings}
\usepackage[T1]{fontenc}
\usepackage{courier}
\lstset{language=Python,
  backgroundcolor=\color[RGB]{245,245,244},   %代码背景色  
  basicstyle=\ttfamily\footnotesize
  =false, % 应该有bug吧,这样就不在边框上面显示字了
  breaklines=true,
  commentstyle=\color{red!50!green!50!blue!50},%浅灰色的注释
  frame=shadowbox, %把代码用带有阴影的框圈起来 
  extendedchars=false,
  lineskip=-2pt,
  keepspaces=true,
  rulesepcolor=\color{red!20!green!20!blue!20},%代码块边框为淡青色
  %backgroundcolor=\color[RGB]{198, 226, 255},   %代码背景色  
%numbers=left,%左侧显示行号 往左靠,还可以为right,或none,即不加行号  
%numberstyle={\color[RGB]{0,192,192}\tiny} ,%设置行号的大小,大小有tiny,scriptsize,footnotesize,small,normalsize,large等  
  keywordstyle=\color{red},
  commentstyle=\color{blue},
  morecomment=[l]{!\ }% Comment only with space after !
  %escapeinside={/*@}{@*/},  
  %escapeinside={\%*}{*)}, 
  showspaces=false,                % show spaces everywhere adding particular underscores; it overrides 'showstringspaces'
  showstringspaces=false,          % underline spaces within strings only
  showtabs=false,                  % show tabs within strings adding particular underscores
  % stepnumber=2,                    % the step between two line-numbers. If it's 1, each line will be numbered
  % stringstyle=\color{mymauve},     % string literal style
  % tabsize=2,                       % sets default tabsize to 2 spaces
  %title=\lstname
  xleftmargin=0em, xrightmargin=0em, aboveskip=1em%设置边距
}

%%%%%% 设置字号 %%%%%%
\newcommand{\chuhao}{\fontsize{42pt}{\baselineskip}\selectfont}
\newcommand{\xiaochuhao}{\fontsize{36pt}{\baselineskip}\selectfont}
\newcommand{\yihao}{\fontsize{28pt}{\baselineskip}\selectfont}
\newcommand{\erhao}{\fontsize{21pt}{\baselineskip}\selectfont}
\newcommand{\xiaoerhao}{\fontsize{18pt}{\baselineskip}\selectfont}
\newcommand{\xiaosihao}{\fontsize{12pt}{\baselineskip}\selectfont}
\newcommand{\wuhao}{\fontsize{10.5pt}{\baselineskip}\selectfont}
\newcommand{\xiaowuhao}{\fontsize{9pt}{\baselineskip}\selectfont}
\newcommand{\liuhao}{\fontsize{7.5pt}{\baselineskip}\selectfont}

%%%% 段落首行缩进两个字 %%%%
\makeatletter
\let\@afterindentfalse\@afterindenttrue
\@afterindenttrue
\makeatother
\setlength{\parindent}{2em}  %中文缩进两个汉字位

\renewcommand{\tablename}{表}

%%%% 正文开始 %%%%
\begin{document}
\begin{spacing}{1.0}

%%%% 定义标题格式,包括title,author,affiliation,email等 %%%%
\title{KNN算法试验报告}
\author{\wuhao 刘群\footnote{电子邮件: liu-q14@mails.tsinghua.edu.cn,学号: 2014211591}\\[2ex]
\wuhao 地球系统科学研究中心\\
}
\date{\wuhao 2015年4月11日}
%%%% 以下部分是正文 %%%%  
\maketitle

%\tableofcontents
%\newpage
\xiaowuhao 这次作业是让我们编写利用KNN算法实现基于Iris数据集分类的程序, 与课上我们练习的情形类似,但这是一个多维的情形. 这个实验报告的思路是先对KNN算法做一个简单的介绍,然后对程序的整体思路做一个介绍,再对每个函数做详细的介绍,最后进行总结。
%\begin{multicols}{2}
\section{\xiaosihao KNN算法简介}
KNN算法是通过计算某个样本点与已经分好类的数据中所有点的距离,并对这些距离进行排序,找出其中距离最近的k个点,然后根据这k个点中所有类别个数的多少判断该样本点所属的类别。
\section{\xiaosihao 程序的整体思路}
首先,我将已分类的数据读入,由于数据中既有数据也有类别字符串,因此我先将这两者分离。这主要是由读入文件的函数实现的。然后就是编写KNN的分类函数,主要是通过计算新的数据与已有数据中所有样本点的距离来确定。通过计算距离,我们可以确定出所有点中与这个代分类点最近的k个点,然后可以统计这k个点中每个类别出现的次数, 次数最多的类别即为代分类样本点所属的类别。接下来就是对错误率的统计,主要实现思路为将测试用例中的数据依次进行KNN分类,将分类所得的结果与它实际所属的类别进行对比,如果出现错误则进行统计。最后返回错误率.主要函数如下:
\begin{enumerate}\setlength{\itemsep}{-2pt}
\item loadDataSet 主要是读入数据,对数据进行分割形成规则数据集和类别标签  
\item knnClassify 主要进行KNN分类 
\item KNN 这个函数主要是对测试文件中的数据进行分类,统计错误率 
\end{enumerate}


%---------------------Module array----------------------%
\section{\xiaosihao 主要函数的介绍}
\subsection{loadDataSet}
主要是读入数据,对数据进行分割形成规则数据集和类别标签, 现将函数详细介绍如下。open是打开要读取的文件,返回文件的句柄。然后用read可以读入整个文件,但是返回的结果是一个字符串,包含了文件中所有的内容。然后由于每一行后面都有一个$\verb|\|$n,因此可以通过字符串的split命令将整个文件字符串分解为每一行为一个元素的list, 即程序中的wholeData(未防止后面出现空行,所以加上if非空的判断)。注意,其中的每个元素此时仍为字符串。然后对这个列表中的每个元素再做一个split,注意到此时元素中以逗号为分隔符。这样就可以把最后的类别名读入labels列表,把前面的四个数据读入dataSet矩阵。此时要注意的是要想转换成矩阵的形式,需要调用numpy中的array函数,否则的话只是一个单纯的list。  
\begin{lstlisting}
import numpy as np
def loadDataSet(filename):
    wholeData = open(filename).read().split('\n')
    # 将最后的类别名读入labels列表
    labels = [ line.split(',')[4] for line in wholeData if line != '']
    # 将前面四列数据读入dataSet 数组(矩阵形式)
    dataSet = np.array([[float(x) for x in line.split(',')[0:4]] for line in wholeData if line != ''])
    return dataSet, labels
\end{lstlisting}

\subsection{knnClassify}
这个程序主要是进行KNN分类,首先读入已分好类的数据,通过loadDataSet函数得到这些数据及其相应的标签。在进行距离的计算时,可以充分利用python中提供的矩阵运算的特点,利用numpy中的tile函数将新读入的数据扩展成一个与已排好数据相同大小的矩阵,这样可以直接进行计算。最后利用numpy中的argsort函数得到从小到大序列的下标(在原数组中的位置)。然后利用字典,以label为key值,将出现的次数记为其value值,然后统计前k个距离最近的点中不同类别的个数。最后找出个数最大的类别并将其名称返回。
\begin{lstlisting}
def knnClassify(newdata, k):
    # 读入训练数据 read train data
    dataSet, labels = loadDataSet('train_iris.data')
    # 训练数据的行数
    nrows = dataSet.shape[0]
    # 计算新的数据点到所有已分好类的数据点的距离,并对其排序,得到排序序列的index
    # 注意利用矩阵来运算
    sortedIndex= np.argsort(np.sqrt(np.sum((np.tile(newdata, (nrows, 1))-dataSet)**2, axis = 1))) 

    # Error check 错误检测,如果k比行数还大或者k小于1,输出提示信息,并抛出错误 
    if k > nrows or k <= 0:
        print "k should be no larger than " + str(nrows) + "."
    assert k <= nrows and k >= 1
    
    # 利用字典,统计每种类别在前k个中的个数
    classCount = {}
    for i in range(k):
        voteLabel = labels[sortedIndex[i]]
        # 如果voteLabel所对应的value不存在,返回0(get 函数)
        # 相应的label加1
        classCount[voteLabel] = classCount.get(voteLabel, 0) + 1

    # 统计个数最多的Label
    maxCount = 0
    for key, value in classCount.items():
        if value > maxCount:
            maxCount = value
            maxIndex = key

    return maxIndex
\end{lstlisting}


\subsection{KNN}
 这个函数主要是对测试文件进行测试,得到KNN分类的错误率。其主要过程如下,通过loadDataSet函数读入要测试的文件,然后通过对代测试文件中的每一行数据调用knnClassify函数进行分类,将分类结果与其本身的类别做比较,如果不同,则分类错误,利用float将bool型变量转为浮点型的数据。由于将这些0或者1放到了一个list里,然后利用sum函数求和即可得到错误的行数。最后即可算出错误率并返回。
\begin{lstlisting}
def KNN(test_file_name, k):
    # 读入测试文件,得到数据和类别标签 
    testDataSet, testLabels = loadDataSet(test_file_name)
    # 测试文件的行数
    totalRows = testDataSet.shape[0]
    # KNN分类错误的行数
    errorRows = sum([ float( knnClassify(testDataSet[i], k) != testLabels[i] ) for i in range(totalRows) ])
    # 返回错误率
    return errorRows / float(totalRows)
\end{lstlisting}

\section{总结}
KNN算法虽然比较简单,但是对于我来说,由于对python掌握的不是很熟练,尤其是对于numpy等模块里的函数更是不熟悉,因此在编写时感到比较吃力。在编写python程序时,我们可以充分利用comprehension结构,将程序写的简单易读。
%\end{multicols}
\end{spacing}
\end{document}