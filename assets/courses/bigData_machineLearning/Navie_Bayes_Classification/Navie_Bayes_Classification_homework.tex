\documentclass[a4paper, 1pt]{article}

%%%%%% 导入包 %%%%%%
\usepackage{graphicx}
\usepackage{amsmath} 
\usepackage{xeCJK}
\usepackage{booktabs}
\usepackage{multicol}
\setlength{\columnsep}{2em}
\usepackage{setspace} % set row space
\usepackage{geometry} % 页边距
\geometry{left=2.5cm,right=2.5cm,top=2.5cm,bottom=2.5cm}
\usepackage{enumitem}% item等间距过大解决办法
\usepackage[unicode]{hyperref}
\usepackage{xcolor}
\usepackage{cite}
\usepackage{indentfirst}
\setCJKmainfont{Songti SC}
\usepackage{geometry}
\usepackage{listings}
\usepackage[T1]{fontenc}
\usepackage{courier}
\lstset{language=Python,
  backgroundcolor=\color[RGB]{245,245,244},   %代码背景色  
  basicstyle=\ttfamily\footnotesize
  =false, % 应该有bug吧,这样就不在边框上面显示字了
  breaklines=true,
  commentstyle=\color{red!50!green!50!blue!50},%浅灰色的注释
  frame=shadowbox, %把代码用带有阴影的框圈起来 
  extendedchars=false,
  lineskip=-2pt,
  keepspaces=true,
  rulesepcolor=\color{red!20!green!20!blue!20},%代码块边框为淡青色
  %backgroundcolor=\color[RGB]{198, 226, 255},   %代码背景色  
  %numbers=left,%左侧显示行号 往左靠,还可以为right,或none,即不加行号  
  %numberstyle={\color[RGB]{0,192,192}\tiny} ,%设置行号的大小,大小有tiny,scriptsize,footnotesize,small,normalsize,large等  
  keywordstyle=\color{red},
  commentstyle=\color{blue},
  morecomment=[l]{!\ }% Comment only with space after !
  %escapeinside={/*@}{@*/},  
  %escapeinside={\%*}{*)}, 
  showspaces=false,                % show spaces everywhere adding particular underscores; it overrides 'showstringspaces'
  showstringspaces=false,          % underline spaces within strings only
  showtabs=false,                  % show tabs within strings adding particular underscores
  % stepnumber=2,                    % the step between two line-numbers. If it's 1, each line will be numbered
  % stringstyle=\color{mymauve},     % string literal style
  % tabsize=2,                       % sets default tabsize to 2 spaces
  %title=\lstname
  xleftmargin=0em, xrightmargin=0em, aboveskip=1em%设置边距
}

%%%%%% 设置字号 %%%%%%
\newcommand{\chuhao}{\fontsize{42pt}{\baselineskip}\selectfont}
\newcommand{\xiaochuhao}{\fontsize{36pt}{\baselineskip}\selectfont}
\newcommand{\yihao}{\fontsize{28pt}{\baselineskip}\selectfont}
\newcommand{\erhao}{\fontsize{21pt}{\baselineskip}\selectfont}
\newcommand{\xiaoerhao}{\fontsize{18pt}{\baselineskip}\selectfont}
\newcommand{\xiaosihao}{\fontsize{12pt}{\baselineskip}\selectfont}
\newcommand{\wuhao}{\fontsize{10.5pt}{\baselineskip}\selectfont}
\newcommand{\xiaowuhao}{\fontsize{9pt}{\baselineskip}\selectfont}
\newcommand{\liuhao}{\fontsize{7.5pt}{\baselineskip}\selectfont}

%%%% 段落首行缩进两个字 %%%%
\makeatletter
\let\@afterindentfalse\@afterindenttrue
\@afterindenttrue
\makeatother
\setlength{\parindent}{2em}  %中文缩进两个汉字位

\renewcommand{\tablename}{表}

%%%% 正文开始 %%%%
\begin{document}
\begin{spacing}{1.0}

%%%% 定义标题格式,包括title,author,affiliation,email等 %%%%
\title{朴素贝叶斯分类算法实验报告}
\author{\wuhao 刘群\footnote{电子邮件: liu-q14@mails.tsinghua.edu.cn,学号: 2014211591}\\[2ex]
\wuhao 地球系统科学研究中心\\
}
\date{\wuhao 2015年4月14日}
%%%% 以下部分是正文 %%%%  
\maketitle

%\tableofcontents
%\newpage
\xiaowuhao 这次作业是让我们编写利用朴素贝叶斯算法实现基于Iris数据集分类的程序. 这个实验报告的思路是先对朴素贝叶斯算法做一个简单的介绍,然后对函数做详细的介绍,最后进行总结。
%\begin{multicols}{2}
\section{\xiaosihao Naive Bayes算法简介}
Naive Bayes算法是一种基于Bayes概率的一种分类方法,其主要思想是要最大化后验概率。主要假设所有的特征都是相互独立的,所有的特征对结论都是同等重要的。
贝叶斯分类的基本公式如下:
$$p(C|X) = \frac{p(X|C)p(C)}{p(X)} \rightarrow p(C|X) \propto p(X|C)p(C)$$
其中,$X=X(x_1, x_2, \cdots, x_n), C=C(c_1, c_2, \cdots, c_L)$. 如果$x_1, x_2, \cdots, x_n$ 是相互独立的,则有
$$p(X|C)=p(x_1,x_2,\cdots,x_n | C) = p(x_1|C)p(x_2|C)\cdots p(x_n|C)$$
最终,我们要在计算得出的后验概率$p(c_k|X)(k=1,2,\cdots,L)$中选择一个最大值,这样$X$就属于这个概率所对应的类别。

%---------------------Module array----------------------%
\section{\xiaosihao NBIris函数介绍}
这个函数是用来对Iris的数据进行分类的,其主要思路如下:首先通过引入的datasets读入所需的数据。接下来就是各种概率的计算。先计算每种类别的先验概率$p(y)$, 然后需要计算似然概率$p(x|y)$, 由于我们没有任何其他信息,因此这里假设每种类别中所对应的四个参数$x_1, x_2, x_3, x_4$ 均满足正态分布,因此就需要将对应的类别的数据提取出来,计算相应的均值和标准差,因为只需要这两个参数就可以唯一确定正态分布的函数。此时即可计算得出似然概率:
$$p(y|x) \propto p(x_i|y) = \frac{1}{\sigma_y\sqrt{2\pi}}e^{-\frac{(x_i-\mu_y)^2}{2\sigma_y^2}}\quad i=1,2,3,4$$
最后就是要计算
$$\prod_{i=1}^4p(x_i|y)p(y)$$
由于这样计算之后,可能出现的情况是数值特别小,这样会产生很大的误差,因此在这里我们对上式取对数,即计算
$$\text{ln}\left(\prod_{i=1}^4p(x_i|y)p(y)\right)=\sum_{i=1}^n\text{ln }p(x_i|y)+ \text{ln }p(y)$$
即可。最后,我们返回概率中最大的值的索引号即可。
\begin{lstlisting}
import numpy as np
from sklearn import datasets

def NBIris(x_test):
    # load train data to x_train and y_train
    iris = datasets.load_iris()
    x_train = iris.data
    y_train = iris.target

    # calculate the probability of each type
    p_y = np.array([y_train[y_train == i].shape[0] / float(x_train.shape[0]) for i in range(3)])

    # calculate the average and stddev of normal distribution
    ave = np.zeros((3,4))
    sigma = np.zeros((3,4))
    for i in range(3):
        ave[i,:] = np.average(x_train[y_train==i, :], axis=0)
        sigma[i,:] = np.std(x_train[y_train==i, :], axis=0)

    x_test_matrix = np.tile(x_test, (3,1))
    # calcualte the probability of p(xi|y) with the equation of normal distribution
    p_xy = 1.0/sigma/np.sqrt(2*np.pi)*np.exp(-(x_test_matrix-ave)**2/2.0/sigma**2)
    # change the probability to log, because the value may be too small
    log_p_yx = np.sum(np.log(p_xy), axis=1)+np.log(p_y)
    # return the index of max probability, and it's also the type label
    y_test = np.argmax(log_p_yx)
    return y_test
\end{lstlisting}


\section{\xiaosihao 总结}
通过这个程序的编写,我首先熟悉了朴素贝叶斯分类器的主要思想,然后进一步熟悉了Python的相关操作,但是还是感觉对于numpy库中的函数掌握不够熟练,另外,就是在利用这个库的时候,尽量采取矩阵的操作,这样可以提高程序的效率。
%\end{multicols}
\end{spacing}
\end{document}