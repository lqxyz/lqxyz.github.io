\documentclass[a4paper]{article}
\usepackage{geometry}
\geometry{left=2.5cm,right=2.5cm,top=2cm,bottom=2cm}
\usepackage{hyperref}
\usepackage{amsmath}
\usepackage{bm}
\usepackage{amsfonts,amssymb}
\usepackage{graphicx}
\usepackage{enumerate}
\usepackage{enumitem}  % Change the beginning order
\usepackage{booktabs}
\usepackage{float}

%opening
\title{ECMM719, Fluid Dynamics of the Atmosphere and Ocean\\
\textbf{Problem Set 1}}
\author{Qun Liu (Student No: 670016014)\\ \href{ql260@exeter.ac.uk}{ql260@exeter.ac.uk}
\\College of Enginerring, Mathematics and Physical Sciences}
\date{Feburary, 2018}

\begin{document}

\maketitle

\begin{enumerate}[label=\textbf{\arabic*.}]
	\setcounter{enumi}{1}
	%-------------------------- Problem 1 ------------------------------%
	\item 
		\begin{enumerate}[label=(\alph*)]%for small alpha-characters within brackets.
			\setcounter{enumii}{0}
			\item Consider a fluid that obeys the hydrostatic relation
			$$\frac{\partial p}{\partial z}=-\rho g$$
			Suppose also that the fluid is an isothermal ideal gas. Show that the density and pressure both diminish exponentially with height. What is the e-folding height? (This is also called the 'scale height' of the atmosphere.) Write down
			an expression for the height, $z$, as a function of pressure.\\
			
			\textbf{Solution:} According to ideal gas law,
			\begin{equation}\label{eq:gas_law}
				p=\rho RT,
			\end{equation}
			Because the ideal gas is isothermal, so $T=T_0$, where $T_0$ is the surface temperature.
			
			Hence $$\rho = \frac{p}{RT_0}.$$
			Plug it into the hydrostatic relation, 
			$$\frac{d p}{d z}=-\frac{p}{RT_0}g$$
			$$\frac{d p}{p}=-\frac{g}{RT_0}dz ~ \Longrightarrow ~d \operatorname{log} p=-\frac{g}{RT_0}dz $$
			$$$$
			Integrate the above equation from $p_0$ to $p$ ($z$ from $0$ to $z$), hence
			$$\int_{p_0}^pd \operatorname{log} p=-\int_0^z \frac{g}{RT_0}dz$$
			$$\operatorname{log}{(p/p_0)}=-\frac{g}{RT_0}z$$
			Therefore,
			\begin{equation}\label{eq:p}
				p=p_0 \operatorname{exp}\left(-\frac{g}{RT_0}z\right)
			\end{equation}
			Apply the ideal gas law to \eqref{eq:p}, then
			\begin{equation}\label{eq:rho}
				\rho=\rho_0 \operatorname{exp}\left(-\frac{g}{RT_0}z\right),
			\end{equation}
			where $\rho_0 = \frac{p_s}{RT_0}$.
			
			Based on \eqref{eq:p}, the $z$ can be written as
			$$z=-\frac{RT_0}{g}\operatorname{log}\left(\frac{p}{p_0}\right)$$
			
			\item Now suppose that the atmosphere has a uniform lapse rate (i.e., $dT/dz=-\Gamma=$ constant). Show that the height at a pressure $p$ is given by
			$$z=\frac{T_0}{\Gamma}\left[1-\left(\frac{p_0}{p}\right)^{-R\Gamma/g}\right]$$
			where $T_0$is the temperature at $z = 0$.
			
			\textbf{Solution:} We could get from $dT /dz =-\Gamma$ that
			$$T=T_0-\Gamma z$$
			Plug it into the idea gas law \eqref{eq:gas_law}, hence
			$$p=\rho R (T_0-\Gamma z),$$
			so the hydrostatic balance equation becomes 
			$$\frac{d p}{d z}=-\frac{p}{R(T_0-\Gamma z)}g$$
			$$\frac{d p}{p}=-\frac{g}{R(T_0-\Gamma z)}dz~\Longrightarrow~d \operatorname{log}p=\frac{g}{R\Gamma}d\operatorname{log}(T_0-\Gamma z)$$
			
			Integrate the above equation from $p_0$ to $p$ ($z$ from $0$ to $z$), hence
			$$\operatorname{log}\frac{p}{p_0}=\frac{g}{R\Gamma}\operatorname{log}\left(1-\frac{\Gamma z}{T_0}\right)$$
			$$p=p_0\left(1-\frac{\Gamma z}{T_0}\right)^{\frac{g}{R\Gamma}}.$$
			Hence $$z=\frac{T_0}{\Gamma}\left[1-\left(\frac{p_0}{p}\right)^{-R\Gamma/g}\right]$$
			\item Are the answers you obtained in these two parts the same as each other in the isothermal (constant temperature) limit? Explain.
			
			\textbf{Solution:} The two answers are the same in the isothermal limit.
			$$\lim\limits_{\Gamma \longrightarrow 0} T_0\frac{1-\left(\frac{p_0}{p}\right)^{-R\Gamma/g}}{\Gamma} = \lim\limits_{\Gamma \longrightarrow 0}T_0 \frac{-\frac{-R}{g}\operatorname{log}\left(\frac{p_0}{p}\right)\left(\frac{p_0}{p}\right)^{-R\Gamma/g}}{1} = -\frac{RT_0}{g}\operatorname{log}\left(\frac{p}{p_0}\right)$$\\
		\end{enumerate}

	\setcounter{enumi}{3}
	\item \begin{enumerate}[label=(\alph*)] 
		\item In the shallow water equations show that geostrophic flow is associated with a slope of the surface. Suppose we consider the ocean to be a shallow water fluid, and that there is a current 100 km wide flowing North-South with a speed of 1m/s. Estimate the variation in the sea-surface height over the width of the current.
		
		\textbf{Solution:} The momentum equation of shallow water equation is 
		$$\frac{\operatorname{D}\bm{u}}{\operatorname{D}t}+\bm{f}\times \bm{u}=-g\nabla \eta .$$
		The geostrophic balance occurs when the Rossby number $U/fL$ is small, so the momentum equation becomes
		$$\bm{f}\times \bm{u_g}=-g\nabla \eta.$$
		Hence the geostrophic flow is associated with the slope of the surface.
		
		If the flow is in North-South direction, so 
		$$fv_g = -g\frac{\partial \eta}{\partial x},$$
		$$\frac{\partial \eta}{\partial x} = -\frac{fv_g}{g},$$
		
		Substitute $v_g=-1m/s, f\approx 10^{-4}s^{-1}, g=9.8ms^{-2}$ into the above equation, so the sea-surface height gradient is
		$$\frac{\partial \eta}{\partial x} \approx 10^{-5}, $$
		and the height difference at the both sides of flow ($100km$ width) is about $1m$.
		
		\item  In the shallow water equations show that, if the flow is approximately geostrophically balanced, the energy at large scales is predominantly potential energy and that energy at small scales is predominantly kinetic energy. Define precisely what `large scale' and `small scale' mean in this context, and obtain an expression for the transition scale.
		
		\textbf{Solution:}  The kinetic energy of a column is
		$$KE=\frac{1}{2}\rho_0 h\bm u^2,$$
		assuming the density is constant ($\rho=\rho_0$), and the potential energy is
		$$PE=\int_{0}^{h}\rho_0gz\operatorname{d}z=\frac{1}{2}\rho_0gh^2.$$
		The ratio between the $KE$ and $PE$ is
		$$\frac{KE}{PE}=\frac{u^2}{gh}.$$
		
		Noting that the scale for each variable is $u\sim U, h\sim H$, so the ratio has scale
		\begin{equation}\label{eq:ke_pe_ratio}
			\frac{KE}{PE}\sim \frac{U^2}{gH}.
		\end{equation}
		$$$$
		
		The geostrophic balance gives
		$$\bm f\times \bm u_g = -g\nabla \eta, $$
		and the scale of $\eta$ is $\eta \sim H$, so
		\begin{equation*}
			fU\sim \frac{gH}{L},
		\end{equation*}
		and 
		\begin{equation}\label{eq:gh}
		gH\sim fUL.
		\end{equation}
	Substitute \eqref{eq:gh} into \eqref{eq:ke_pe_ratio}, the ratio between kinetic and potential energy becomes
	$$	\frac{KE}{PE}\sim \frac{U}{fL}=Ro,$$
	where Ro is the Rossby number.
	
	The Rossby number $Ro\ll 1$ at large scale, that is $\frac{KE}{PE}\ll 1$, indicating that the energy is predominately by potential energy. 
	
	The Rossby number $Ro\gg 1$ at small scale, that is $\frac{KE}{PE}\gg 1$, showing the energy is predominately by kinetic energy.\\
	
	\end{enumerate}

	\item 	
		\begin{enumerate}[label=(\alph*)] 
		\item In an adiabatic shallow water fluid in a rotating reference frame show that the potential vorticity conservation law is
		$$\frac{\operatorname{D}}{\operatorname{D}t}\left(\frac{\zeta+f}{\eta-h_b}\right)=0,$$
		where $\eta $ is the height of the free surface and $h_b$ is the height of the bottom topography, both referenced to the same flat surface.
		
		\textbf{Solution:} The momentum equation of shallow water system is
		$$\frac{\operatorname{D}\bm{u}}{\operatorname{D}t}+\bm{f}\times \bm{u}=-g\nabla \eta,$$
		$$\frac{\partial \bm{u}}{\partial t}+(\bm u\cdot \nabla ) \bm u +\bm{f}\times \bm{u}=-g\nabla \eta ,$$
		where $\bm f=f\bm k$.
		Using the vector identity 
		$$(\bm u\cdot \nabla ) \bm u = \nabla \frac{\bm u^2}{2}-\bm u\times (\nabla\times \bm u),$$
		then the momentum equation becomes that
		$$\frac{\partial \bm{u}}{\partial t}+(\bm \omega^*+\bm{f})\times \bm{u}=-\nabla \left(g\eta+\frac{\bm u^2}{2}\right) ,$$
		where $\bm \omega^* =\nabla \times \bm u=\zeta\bm k$. Taking the curl of this gives the vorticity equation
		\begin{equation}\label{eq:sw_mom_curl_1}
		\frac{\partial \zeta}{\partial t}+(\bm u\cdot \nabla )(\zeta +f)=-(f+\zeta)\nabla \cdot \bm u.
		\end{equation}
		
		Because $f$ is time independent, so \eqref{eq:sw_mom_curl_1} could be rewritten as 
		\begin{equation}\label{eq:sw_mom_curl}
		\frac{\operatorname{D}( \zeta+f)}{\operatorname{D}t} =\frac{\partial( \zeta+f)}{\partial t}+(\bm u\cdot \nabla )(\zeta +f)=-(f+\zeta)\nabla \cdot \bm u.
		\end{equation}
		
		The mass conservation equation is
		$$\frac{\operatorname{D}h}{\operatorname{D}t}+h\nabla \cdot \bm u=0,$$
		where $h=\eta-h_b$. It could be rewritten as 
		\begin{equation}\label{eq:sw_mass}
		-(\zeta+f)\nabla \cdot \bm u = \frac{(\zeta+f)}{h} \frac{\operatorname{D}h}{\operatorname{D}t}
		\end{equation}
		Combining \eqref{eq:sw_mom_curl} and \eqref{eq:sw_mass} gives
		$$\frac{\operatorname{D}( \zeta+f)}{\operatorname{D}t} = \frac{(\zeta+f)}{h} \frac{\operatorname{D}h}{\operatorname{D}t},$$
		that is
		$$\frac{\operatorname{D}}{\operatorname{D}t}\left(\frac{\zeta+f}{h}\right)=0,$$
		$$\Longrightarrow \frac{\operatorname{D}}{\operatorname{D}t}\left(\frac{\zeta+f}{\eta-h_b}\right)=0.$$
		
		\item An air column at 60$^\circ$N with zero relative vorticity ($\zeta= 0$) stretches from the surface to the tropopause, which we assume is a rigid lid, at 10 km. The air column moves zonally on to a plateau 2.5 km high. What is its relative vorticity? Suppose it then moves southwards to 30$^\circ$N, staying on the plateau. What is its relative vorticity then? (Assume that the density is constant.)	
		
		\textbf{Solution:} According to the potential vorticity conservation,
		we assume that $$\frac{\zeta+f}{\eta-h_b}=\text{Constant}.$$
		At  60$^\circ$N, $f_{ 60^\circ N}=2\times \Omega \times \text{sin}(60^{\circ})\approx 1.26\times 10^{-4}s^{-1}.$ The PV at surface and plateau are equal, that is
		$$\frac{0+f_{ 60^\circ N}}{10km}=\frac{\zeta_{2.5km, 60^\circ N} +f_{ 60^\circ N}}{10km-2.5km},$$
		so the relative vorticity at plateau is
		$$\zeta_{2.5km, 60^\circ N}\approx -3.1\times 10^{-5} s^{-1}.$$
		
		At $30^\circ$N, $f_{ 30^\circ N}=2\times \Omega \times \text{sin}(30^{\circ})\approx 7.27\times 10^ {-5}s^{-1}.$ The PV at $60^\circ$N and $30^\circ$N are equal, that is
		$$\frac{\zeta_{2.5km, 60^\circ N}+f_{60^\circ N}}{10km-2.5km}=\frac{\zeta_{2.5km, 30^\circ N} +f_{30^\circ N}}{10km-2.5km},$$
		so the relative vorticity is
		$$\zeta_{2.5km, 30^\circ N} \approx 2.2\times 10^{-5}s^{-1}.$$
		\end{enumerate}

	\setcounter{enumi}{5}
	\item The shallow water equations, linearized about a state of rest, may be written as
	$$\frac{\partial u'}{\partial t}-f_0v'=-g\frac{\partial \eta'}{\partial x}, \quad \frac{\partial v'}{\partial t}+f_0u'=-g\frac{\partial \eta'}{\partial y},$$
	$$\frac{\partial \eta '}{\partial t} + H \left(\frac{\partial u'}{\partial x}+\frac{\partial v'}{\partial y}\right)=0$$
	Suppose there is a solid boundary (e.g., a coastline) at $x = 0$, with the ocean on one side and land on the other. Look for solutions that have $u'=0$ everywhere, and with $f_0 > 0$. Show that the resulting waves are non-dispersive and travel at a speed $c=\sqrt{gH}$. Is the coastline to the left or to the right of the direction of travel? Suppose that these waves are generated just off the shore of Portugal. Do they move north or south?
	
	\textbf{Solution:} If $u'=0$ everywhere, then the equations becomes 
		$$f_0v'=g\frac{\partial \eta'}{\partial x}, \quad \frac{\partial v'}{\partial t}=-g\frac{\partial \eta'}{\partial y},\quad \frac{\partial \eta '}{\partial t} +H \frac{\partial v'}{\partial y}=0$$
	Using the second and the third equations above, we could get
	$$ \frac{\partial^2 v'}{\partial t^2} = -g\frac{\partial}{\partial y }\left(\frac{\partial \eta '}{\partial t}\right)=c^2\frac{\partial^2 v '}{\partial y^2},$$
	where $c=\sqrt{gH}.$
	Assuming that $$v'=\tilde{v}\operatorname{e}^{i(ly-\omega t)},$$
	and substitute it into the wave equation, and we could get the dispersion relationship
	$$\omega ^2 =c^2l^2\Longrightarrow \omega = cl,$$
	which indicates that the wave is non-dispersive.
	
	The general form  of $v'$ is 
	$$v'=F_1(x,y+ct)+F_2(x,y-ct),$$
	with corresponding surface displacement 
	$$\eta' = \sqrt{H/g}\left(-F_1(x, y+ct)+F_2(x,y-ct)\right)$$
	Substitute $v'$ and $\eta'$ into the first equation of shallow water equations, we could get
	$$f_0(F_1+F_2)=\sqrt{gH}\left(-\frac{\partial F_1}{\partial x}+\frac{\partial F_2}{\partial x}\right),$$
	that is
		$$f_0F_1=-\sqrt{gH}\frac{\partial F_1}{\partial x},$$
		$$f_0F_2=\sqrt{gH}\frac{\partial F_2}{\partial x},$$
	with solutions
	$$F_1 = G_1(y+ct)\operatorname{e}^{-x/L_d},$$
	$$F_2 = G_2(y-ct)\operatorname{e}^{x/L_d},$$
	where $L_d=\sqrt{gH}/f_0.$
	
	If we consider the flow in half-plane $x>0$, then $F_2$ should be neglected for it will grow to infinity when $x$ goes to infinity. The wave will travel in negative $y$ direction, indicating that the coastline is to the right of the direction of travel.
	
	If the waves are generated just off the shore of Portugal, the wave will exist in $x<0$ half plane, the solution form $F_2$ should be remained, so the waves will move north.
	
	\item 
	$$\frac{\partial u}{\partial t}-fv+\frac{\partial \Phi}{\partial x}=0,$$
	$$\frac{\partial v}{\partial t}+fu+\frac{\partial \Phi}{\partial y}=0,$$
	$$\frac{\partial\Phi}{\partial t} + \Phi_0 \left(\frac{\partial u}{\partial x}+\frac{\partial v}{\partial y}\right)=0$$
	To obtain the dispersion relationship, we let
	$$(u,v,\Phi)=(\tilde{u},\tilde{v},\tilde{\Phi})\operatorname{e}^{i(kx+ly-\omega t)},$$
	and substitute them into the governing equations, giving
	$$-i\omega \tilde{u}-f\tilde{v}+ik\tilde{\Phi} = 0,$$
	$$-i\omega \tilde{v}+f\tilde{u}+il\tilde{\Phi} = 0,$$
	$$-i\omega \tilde{\Phi}+i\Phi_0(k\tilde{u}+l\tilde{v})= 0.$$
	Rewrite them into matrix format, that is
	\begin{equation}\label{eq:matrix}
		\left(
		\begin{matrix}
		-i\omega& -f& ik\\
		f & -i\omega & il\\
		i\Phi_0k & i\Phi_0 l & -i\omega \\
		\end{matrix}
		\right)\left(
		\begin{matrix}
		\tilde{u}\\
		\tilde{v}\\
		\tilde{\Phi}
		\end{matrix}
		\right)=0.
	\end{equation}
	To get non-trivial solutions, the determinant of matrix in \eqref{eq:matrix} shoul be 0, that is
	$$	-i\omega(-\omega^2+\Phi_0l^2)+f(-if\omega+\Phi_0lk)+ik(if\Phi_0l-\Phi_0k\omega)=0,$$
	$$\Longrightarrow\omega [\omega^2-f^2-\Phi_0(k^2+l^2)]=0.$$
	For $\omega =0,$ it is the time-independent flow corresponding to the geostrophic balance. For $\omega^2=f^2+\Phi_0(k^2+l^2)$, they are Poincar\'{e} waves.\\
	%\setcounter{enumi}{5}
	\item \begin{enumerate}[label=(\alph*)]%for small alpha-characters within brackets.
		\setcounter{enumii}{0}
		\item Derive an equation for the conservation of energy in the inviscid, adiabatic, Boussinesq equations.
		
		\textbf{Solution:} The Boussinesq equations are
		
		\begin{subequations}
			\begin{equation}\label{eq:mom}
				\frac{\operatorname{D}\bm{v}}{\operatorname{D}t}+\bm{f}\times \bm{v}=-\nabla \phi +b \bm{k},
			\end{equation}
			\begin{equation}\label{eq:mass}
				\nabla \cdot \bm v=0,
			\end{equation}
			\begin{equation}\label{eq:temp}
				\frac{\operatorname{D}b}{\operatorname{D}t} =0.
			\end{equation}
		\end{subequations}
	
		 \eqref{eq:mom} can be rewritten as
		\begin{equation}\label{eq:mom2}
		\frac{\partial \bm{v}}{\partial t}+(\bm v\cdot \nabla)\bm v+\bm{f}\times \bm{v}=-\nabla \phi +b \bm{k},
		\end{equation}
		
		and recall that 
		$$(\bm v\cdot \nabla )\bm v=-\bm v\times \bm w+\nabla \frac{\bm v^2}{2},$$
		hence \eqref{eq:mom2} can be rewritten as
		\begin{equation}\label{eq:mom2}
		\frac{\partial \bm{v}}{\partial t}-\bm v\times \bm w+\nabla \frac{\bm v^2}{2}+\bm{f}\times \bm{v}=-\nabla \phi +b \bm{k},
		\end{equation}
		Taking the dot product of \eqref{eq:mom2} with $\bm v$ yields
		\begin{equation*}
			\frac{1}{2}\frac{\partial \bm{v^2}}{\partial t}-\bm v\cdot (\bm v\times \bm w)+\bm v\cdot\nabla \frac{\bm v^2}{2}+\bm v \cdot (\bm{f}\times \bm{v})=-\bm v\cdot\nabla \phi +b \bm v\cdot\bm{k},
		\end{equation*}
		that is
		\begin{equation}\label{eq:mom3}
			\frac{1}{2}\frac{\partial \bm{v^2}}{\partial t} - \frac{\bm v^2}{2} \nabla \cdot \bm v+\nabla \left( \bm v \frac{\bm v^2}{2}\right) =\phi \nabla \cdot \bm v-\nabla \cdot (\phi \bm v)+bw.
		\end{equation}
		Substitute \eqref{eq:mass} into \eqref{eq:mom3}, we could get
			\begin{equation}\label{eq:energy_1}
					%\frac{1}{2}\frac{\partial \bm{v^2}}{\partial t}+\bm v\cdot  \nabla \frac{\bm v^2}{2} =-\nabla \cdot (\phi \bm v)+bw.
					\frac{1}{2}\frac{\partial \bm{v^2}}{\partial t}=-\nabla \cdot \left[ \bm v \left(\phi +\frac{\bm v^2}{2} \right) \right] +bw.
					%\frac{1}{2}\frac{\operatorname{D}\bm{v}^2}{\operatorname{D}t} = 
			\end{equation}
		Define the potential $\Phi = -z$, so that
$\nabla \Phi = -\bm k$.
		\begin{equation}
			\frac{\operatorname{D}\Phi}{\operatorname{D}t}=-\frac{\partial z}{\partial t}-u\frac{\partial z}{\partial x}-v\frac{\partial z}{\partial y}-w\frac{\partial z}{\partial z}=-w,
		\end{equation}
		and combining this with \eqref{eq:temp} gives
		\begin{equation}\label{eq:potential_deriv}
			\frac{\operatorname{D}b\Phi}{\operatorname{D}t}=b\frac{\operatorname{D}\Phi}{\operatorname{D}t}+\Phi \frac{\operatorname{D}b}{\operatorname{D}t}=\frac{\partial b\Phi }{\partial t}+(\bm v\cdot \nabla)(b\Phi) = -bw.
		\end{equation}
	
	Substitute \eqref{eq:potential_deriv} into \eqref{eq:energy_1}, we could get
	
	\begin{equation}\label{eq:energy_cons}
	\frac{\partial }{\partial t}\left(\frac{\bm{v^2}}{2}+b\Phi\right)+\nabla \cdot \left[ \bm v \left(\phi +\frac{\bm v^2}{2} +b\Phi\right) \right]=0,
	\end{equation}
	that is the form of energy conservation.

	%where 
	
	\item Noting that $E=\frac{\bm{v^2}}{2}+b\Phi$ is the energy density, so \eqref{eq:energy_cons} could be rewritten as 
	$$\frac{\partial E}{\partial t}+\nabla \cdot \left[ \bm v \left(E+\phi \right) \right]=0$$
	The extra term is from the pressure gradient term, because pressure can do work.
	\end{enumerate}

\end{enumerate}

\end{document}
