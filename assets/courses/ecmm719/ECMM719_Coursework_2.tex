\documentclass[a4paper]{article}
\usepackage{geometry}
\geometry{left=2.5cm,right=2.5cm,top=2cm,bottom=2cm}
\usepackage{hyperref}
\usepackage{amsmath}
\usepackage{bm}
\usepackage{amsfonts,amssymb}
\usepackage{graphicx}
\usepackage{enumerate}
\usepackage{enumitem}  % Change the beginning order
\usepackage{booktabs}
\usepackage{float}

\title{ECMM719, Fluid Dynamics of the Atmosphere and Ocean\\
\textbf{Problem Set 2}}
\author{Qun Liu (Student No: 670016014)\\ \href{ql260@exeter.ac.uk}{ql260@exeter.ac.uk}
\\College of Enginerring, Mathematics and Physical Sciences}
\date{March, 2018}

\begin{document}

\maketitle

\begin{enumerate}[label=\textbf{\arabic*.}]
	\setcounter{enumi}{0}
	\item \textbf{Vertically propagating Rossby waves}
		\begin{enumerate}[label=\textbf{(\alph*)}]
			\setcounter{enumii}{0}
			\item 
			\begin{equation}\label{eq:q_ctrl_vert}
				\frac{\partial q}{\partial t}+U\frac{\partial q}{\partial x}+\beta v=0,
			\end{equation}
			where $$q=\frac{\partial ^2 \psi}{\partial x^2}+\frac{\partial ^2\psi}{\partial y^2}+\frac{\partial }{\partial z}\left(\frac{f_0^2}{N^2}\frac{\partial \psi}{\partial z}\right).$$
			
			\textbf{Solution:}
			Assuming that
			$$\psi = \operatorname{Re}\hat{\psi}\operatorname{e}^{i(kx+ly+mz-\omega t)},$$
			and putting it into $q$ gives %the \eqref{eq:q_ctrl_vert}
			\begin{equation}\label{eq:q_psi}
				q= -\left(k^2+l^2+\frac{f_0^2}{N^2}m^2\right)\operatorname{Re}\hat{\psi}\operatorname{e}^{i(kx+ly+mz-\omega t)}.
			\end{equation}
			In addition, 
			\begin{equation}\label{eq:v_psi}
				v=\frac{\partial \psi}{\partial x}=\operatorname{Re}~ik\hat{\psi}\operatorname{e}^{i(kx+ly+mz-\omega t)}.
			\end{equation}
			Plug \eqref{eq:q_psi} and \eqref{eq:v_psi} into \eqref{eq:q_ctrl_vert}, we could get
			$$i\omega\left(k^2+l^2+\frac{f_0^2}{N^2}m^2\right)-ikU\left(k^2+l^2+\frac{f_0^2}{N^2}m^2\right)+ik\beta=0,$$
			hence
			\begin{equation}\label{eq:dispersion}
				\omega = kU-\frac{k\beta}{k^2+l^2+\frac{f_0^2}{N^2}m^2}.
			\end{equation}
			
			%What kind of waves are these?\\
			These waves are three-dimensional Rossby waves.\\
			
			\item Calculate the vertical component of the group velocity for such waves.
		
			\textbf{Solution:} The vertical component of the group velocity is
			$$c_g^z = \frac{\partial \omega}{\partial m}=\frac{2mk\beta\frac{f_0^2}{N^2}}{\left(k^2+l^2+\frac{f_0^2}{N^2}m^2\right)^2}$$
			\item 
			\begin{equation}\label{eq:b_psi}
				b =f_0\frac{\partial \psi}{\partial z}=\operatorname{Re} ~if_0m\hat{\psi}\operatorname{e}^{i(kx+ly+mz-\omega t)}=-f_0m\hat{\psi}\operatorname{sin}(kx+ly+mz-\omega t),
			\end{equation}
			and $v$ could be rewritten as 
			\begin{equation} \label{eq:v_sin}
				v = -k\hat{\psi}\operatorname{sin}(kx+ly+mz-\omega t).
			\end{equation}
			Noting that
			$$L_\lambda = \frac{2\pi}{k},$$
			and combining \eqref{eq:v_sin} and \eqref{eq:b_psi} gives
			\begin{align*} 
			\overline{vb}&=\frac{1}{L_\lambda}\int_{L_\lambda}vb\operatorname{d}x=\frac{1}{L_\lambda}\int_{L_\lambda}kmf_0\hat{\psi}^2\operatorname{sin}^2(kx+ly+mz-\omega t)\operatorname{d}x\\
				&=\frac{kmf_0\hat{\psi}^2}{L_\lambda}\int_{L_\lambda}\frac{1-\operatorname{cos}[2(kx+ly+mz-\omega t)]}{2}\operatorname{d}x\\
				& = \frac{kmf_0\hat{\psi}^2}{L_\lambda}\frac{L_\lambda}{2}= \frac{kmf_0\hat{\psi}^2}{2} = Akm|\hat{\psi}|^2,
			\end{align*}
			where $A=\frac{f_0}{2}.$
			
			%Hence explain why upward wave propagation requires a poleward eddy bouyancy flux...\\
			If upward wave propagation exists, then $m\neq 0$, indicating that the poleward eddy bouyancy flux $\overline{vb}\neq 0$.\\
			
			\item %The phase velocity in $x$ direction is 
			%$$c_p^x = \frac{\omega}{k}=U-\frac{\beta}{k^2+l^2+\frac{f_0^2}{N^2}m^2},$$
			%so if stationary wave exists in $x$ direction, we have $c_p^x=0$, that is
			%$$U=\frac{\beta}{k^2+l^2+\frac{f_0^2}{N^2}m^2}.$$
			
			%The phase velocity in $z$ direction is
			%$$c_p^z = \frac{\omega}{m}=\frac{kU}{m}-\frac{k\beta}{m\left(k^2+l^2+\frac{f_0^2}{N^2}m^2\right)}.$$
			%If $c_p^z>0$, then 
			%$$\frac{kU}{m}>\frac{k\beta}{m\left(k^2+l^2+\frac{f_0^2}{N^2}m^2\right)}$$
			Noting that $\omega = ck$, and putting it into \ref{eq:dispersion} we obtain
			%$$ck = kU-{k^2+l^2}$$
			\begin{equation}\label{eq:vert_disper}
				m^2=\frac{N^2}{f_0^2}\left(\frac{\beta}{U-c}-k^2-l^2\right).
			\end{equation}
			For stationary waves, $c=0$, then \eqref{eq:vert_disper} becomes
			\begin{equation}\label{eq:vert_disper2}
				m^2=\frac{N^2}{f_0^2}\left(\frac{\beta}{U}-k^2-l^2\right).
			\end{equation}
			For waves to propagate upwards we require that $m^2>0$, and from \eqref{eq:vert_disper2} we obtain
			$$\frac{\beta}{k^2+l^2}>U>0.$$
	\end{enumerate}

\item \textbf{Ekman layers}

 The Ekman-layer equations are
	\begin{equation}\label{eq:eckman_v}
		- f v = - \frac { \partial \phi } { \partial x } + \frac { \partial \tau _ { x } } { \partial z } ,
	\end{equation}
	\begin{equation}\label{eq:eckman_u}
		\quad f u = - \frac { \partial \phi } { \partial y } + \frac { \partial \tau _ { y } } { \partial z }
	\end{equation}

		\begin{enumerate}[label=\textbf{(\alph*)}]
		\setcounter{enumii}{0}
		\item 
		We could rewrite \eqref{eq:eckman_v} and \eqref{eq:eckman_u} in the vector form, that is
		\begin{equation}
			\bm f\times \bm u = -\nabla \phi +  \frac{\partial \bm\tau } { \partial z },
		\end{equation}
		where $\bm u=(u,v), \bm \tau =(\tau_x, \tau_y)$. In the  Eckman layer itself we have
		\begin{equation}\label{eq:eckman_layer_vec}
			\bm f\times \bm u_{_E} =  \frac{\partial \bm\tau } { \partial z }.
		\end{equation}
		Integrate from the bottom to top of the Eckman layer, we could obtain
		\begin{equation}\label{eq:eckman_layer_transp}
			\bm f\times \bm M_{_E}=  \int_{-H_E}^{0} \frac{\partial \bm\tau } { \partial z }dz=\bm \tau_{_T}-\bm \tau_{_B},
		\end{equation} 
		where $\bm M_{_E}=\int_{-H_E}^{0} \bm u_{_E}dz$ is the agostrophic transport, and $\bm \tau_{_T}$ and $\bm \tau_{_B}=0$ are the wind stress at the top and bottom of Eckman layer. From \eqref{eq:eckman_layer_transp} we could get
		$$\bm M_{_E}= \bm \tau_{_T}-\bm \tau_{_B},$$
		$$\bm M_{_E}= \frac{1}{f}\bm {k}\times \bm \tau_{_T}, $$ which is at the right angle of surface stress.
		
		%We suppose the flow in the interior area far away from the surface is in geostrophic balance, and rewrite the velocity and pressure field into sum of  interior geostrophic part and a boundary layer part:
		%\begin{equation}\label{eq:uv_g_E}
		%	u=u_g+u_{_E}, \quad v=v_g+v_{_E};
		%\end{equation}
		%\begin{equation}\label{eq:phi_g_E}
		%	\phi=\phi_g+\phi_{_E}.
		%\end{equation}
		%Because $\phi$ is not a function of $z$, that is
		%$$\frac { \partial \phi } { \partial z } = 0.$$
		%$$\Longrightarrow \frac { \partial \phi_g } { \partial z } = 0, \quad \frac { \partial \phi_{_E} } { \partial z } = 0$$
		
		%Put \ref{eq:uv_g_E} and \ref{eq:phi_g_E} into \eqref{eq:eckman_v} and \eqref{eq:eckman_u}, and recall that $	- f v_{g} = - \frac { \partial \phi_{g} } { \partial x }, f u_{g} = -\frac { \partial \phi_{g} } { \partial y }$, we obtains that 
		%\begin{equation}\label{eq:eckman_v_E}
		%	- f v_{_E} = - \frac { \partial \phi_{_E} } { \partial x } + \frac { \partial \tau _ { x } } { \partial z },
		%\end{equation}
		%\begin{equation}\label{eq:eckman_u_E}
		%	f u_{_E} = - \frac { \partial \phi_{_E} } { \partial y } + \frac { \partial \tau _ { y } } { \partial z },
		%\end{equation}
		
		\item The mass continuity is 
		\begin{equation}\label{eq:mass}
			\frac { \partial u } { \partial x } + \frac { \partial v } { \partial y } + \frac { \partial w } { \partial z } = 0.
		\end{equation}
		%Integrate \eqref{eq:mass} from $z=-H_o$ to $z=0$ gives us
		%$$\int_{-H_o}^{0}\left(	\frac { \partial u } { \partial x } + \frac { \partial v } { \partial y }\right)dz = -\int_{-H_o}^{0}\frac { \partial w } { \partial z }dz = -(w_{_T}-w_{_B})=0.$$
		%$$\Longrightarrow $$
		%$$\int_{-H_o}^{0}  \frac { \partial v } { \partial y } dz =  -\int_{-H_o}^{0}\frac { \partial u } { \partial x } dz$$
		
		If we rewrite the velocity into sum of interior geostrophic part and a boundary layer part:
		\begin{equation}\label{eq:uv_g_E}
			u=u_g+u_{a}, \quad v=v_g+v_{a},
		\end{equation}
		and they satisfy that
		\begin{equation}\label{eq:fvg}
			-f v _ { g }  = -\frac { \partial{ \phi } } { \partial x },
		\end{equation}
		\begin{equation}\label{eq:fug}
			f u _ { g }  =- \frac { \partial{ \phi } } { \partial y },
		\end{equation}
		where $f=f_0+\beta y.$ We could obtain that
		\begin{equation}\label{eq:u_v_a}
		f \left( v _ { g } - v \right) = \frac { \partial{ \tau } _ { x } } { \partial z } ,\quad f \left( u - u _ { g } \right) = \frac { \partial { \tau } _ { y } } { \partial z }.
		\end{equation}
		
		Also, $-\frac{\partial \eqref{eq:fvg}}{\partial y}+\frac{\partial \eqref{eq:fug}}{\partial x}$ gives that
		\begin{equation}\label{eq:ug_vg_geo}
			f \left( \frac { \partial u _ { g } } { \partial x } + \frac { \partial v _ { g } } { \partial y } \right) = - \beta v _ { g }.
		\end{equation}
		
		With \eqref{eq:uv_g_E}, the mass continuity equation \eqref{eq:mass} becomes 
		$$	\frac { \partial u_g } { \partial x } + \frac { \partial u_a } { \partial y } +\frac { \partial v_g } { \partial x } + \frac { \partial v_a } { \partial y }+ \frac { \partial w } { \partial z } = 0.$$
		Integrating it from the ocean bottom $z=-H_o$ to the surface $z=0$ gives
		\begin{equation}\label{eq:int_u_v_He}
			\int _ { - H _ { o} } ^ { 0} \left( \frac { \partial u _ { a } } { \partial x } + \frac { \partial v _ { a } } { \partial y } + \frac { \partial u _ { g } } { \partial x } + \frac { \partial v _ { g } } { \partial y } \right)=-(w_{_T}-w_{_B})=0.
		\end{equation}
		From \eqref{eq:u_v_a} we get the divergence of agostrophic velocity satisfying 
		\begin{equation}\label{eq:ua_va}
			v_a = -\frac{1}{f}\frac{\partial  \tau_{x}}{\partial z},\quad u_a = \frac{1}{f}\frac{\partial  \tau_{y}}{\partial z},
		\end{equation}
		$$\Longrightarrow \int _ { - H _ { o} } ^ { 0} \left( \frac { \partial u _ { a } } { \partial x } + \frac { \partial v _ { a } } { \partial y }\right)dz=\int_{ -H_o }^{0}\frac { \partial } {\partial z} \left[\frac{\partial }{\partial x} \left(\frac{\tau_y}{f}\right)- \frac{\partial }{\partial y}\left(\frac{\tau_x}{f}\right) \right] dz =  \frac { \partial } { \partial x } \left( \frac { \tau  _ { y 0} } { f } \right) - \frac { \partial } { \partial y } \left( \frac { \tau _ { x0} } { f } \right),$$
		where $\tau_{x0}$ and $\tau_{y0}$ are the components of stress at the surface.
		
		Plugging $u_a, v_a$ and \eqref{eq:ug_vg_geo} into \eqref{eq:int_u_v_He}, we obtains
		\begin{equation}\label{eq:beta_vg}
			\int _ { - H _ { o } } ^ { 0} \beta  v _ { g } d z =  f\left[ \frac { \partial } { \partial x } \left( \frac { \tau  _ { y 0} } { f } \right) - \frac { \partial } { \partial y } \left( \frac { \tau _ { x0} } { f } \right) \right] .
		\end{equation}
		
		\item $-\frac{\partial \eqref{eq:eckman_v}}{\partial y}+\frac{\partial \eqref{eq:eckman_u}}{\partial x}\Longrightarrow$
		\begin{equation}\label{eq:u_v_eckman}
		f \left( \frac { \partial u  } { \partial x } + \frac { \partial v } { \partial y } \right) = - \beta v+\frac { \partial } {\partial z} \left( \frac{\partial \tau_y}{\partial x} - \frac{\partial \tau_x}{\partial y} \right) .
		\end{equation}
		
		Combing \eqref{eq:u_v_eckman} and \eqref{eq:mass} and integrating from bottom to the surface of the ocean, we obtains
		\begin{equation}\label{eq:beta_v}
			\int _ { - H _ { 0} } ^ { 0} \beta v d z =\int_{ -H_o }^{0}\frac { \partial } {\partial z} \left( \frac{\partial \tau_y}{\partial x} - \frac{\partial \tau_x}{\partial y} \right) dz =\frac{\partial \tau_{y0}}{\partial x} - \frac{\partial \tau_{x0}}{\partial y}
		\end{equation}
	
		\item 
		Because $f=f_0+\beta y$ is not a function of $x$, so \eqref{eq:beta_vg} could be rewritten as
			$$\int _ { - H _ { o } } ^ { 0} \beta  v _ { g } d z =   \frac { \partial \tau_{ y0} } { \partial x } - f\frac { \partial } { \partial y } \left( \frac { \tau _ { x0} } { f } \right) = \frac { \partial \tau_{ y0} } { \partial x }-\frac{\partial \tau_{x0}}{\partial y}-\frac{\beta}{f}\tau_{x0}.$$
			
		From \eqref{eq:ua_va} we could get
		$$\int _ { - H_{ o } } ^ { 0} \beta  v_{a} d z =  - \int _ { - H_{ o } } ^ { 0}  \frac{\beta}{f}\frac{\partial  \tau_{x}}{\partial z} dz = \frac{\beta}{f}\tau_{x0}.$$
		
		%$$\int _ { - H_{ 0} } ^ { 0} \beta v d z  = \int _ { - H _ { 0} } ^ { 0} \beta (v_g+v_a) d z $$
		
		Hence,
		$$\int _ { - H _ { o } } ^ { 0} \beta  v _ { g } d z =  \int _ { - H _ { o } } ^ { 0} \beta  v d z -\int _ { - H _ { o } } ^ { 0} \beta  v _ { a } d z,$$
		which means that the two equations \eqref{eq:beta_vg} and \eqref{eq:beta_v} are nevertheless consistent.\\
		\end{enumerate}
	
	\setcounter{enumi}{2}
	\item \textbf{Rossby waves and jets}
		\begin{enumerate}[label=\textbf{(\alph*)}]
			\setcounter{enumii}{0}
			\item Show that if there is a source of Rossby waves at any given latitude in the Northern Hemisphere, we expect that eastward flow will be generated there. Your answer will involve relating flux of momentum in Rossby waves and relating it to group velocity. How does your answer differ in the Southern Hemisphere?
			
			\textbf{Solution:} The stream function for quasi-linear Rossby waves is
				
			$$\psi = \operatorname{Re} C e ^ { i ( k x + l y - \omega t ) } = \operatorname{Re} C e ^ { i ( k x + l y - k c t ) },$$
			where $C=a+ib$ is a complex constant, $a, b \in R$. The dispersion relation is
			$$\omega = c k = Uk - \frac { \beta k } { k ^ { 2} + l ^ { 2} } ,$$ %\equiv \omega _ { R }
			supposing that there is no medional shear in zonal flow. The meridional group velocity is
			$$c_{ g } ^ { y } = \frac { \partial \omega } { \partial l } = \frac { 2\beta k l } { \left( k ^ { 2} + l ^ { 2} \right) ^ { 2} }.$$
			
			%$$l = \pm \left( \frac { \beta } { U - c } - k ^ { 2} \right) ^ { 1/ 2}$$
			The velocity variations associated with the Rossby waves are
			$$u  = -\frac{\partial \psi}{\partial y} = - \operatorname{Re} C \text{i} l e ^ { i ( k x + l y - \omega t ) }= al\sin (kx+ly-\omega t)+bl\cos (kx+ly-\omega t),$$
			$$ v  = \frac{\partial \psi}{\partial x}  =\operatorname{Re} C i k e ^ { i ( k x + l y - \omega t ) }=-ak\sin (kx+ly-\omega t)-bk\cos (kx+ly-\omega t).$$
			
			$$\begin{aligned} \overline{uv}&= \frac{1}{L}\int_{0}^{L}uvdx\\
			&=\frac{1}{L}\int_{0}^{L}-kl\left[a\sin (kx+ly-\omega t)+b\cos (kx+ly-\omega t)\right] \left[a\sin (kx+ly-\omega t)+b\cos (kx+ly-\omega t)\right]dx\\
			& =\frac{1}{L}\int_{0}^{L}-kl \left( a^2\sin ^2 (kx+ly-\omega t)+b^2\cos ^2 (kx+ly-\omega t)\right) dx\\
			&=-\frac{1}{2}(a^2+b^2)kl=-\frac{1}{2}|C|^2kl
			\end{aligned}$$
			
			Because the energy travels at the group velocity, so the $c_g^y$ will be away from the source region. In the northern hemisphere, $c_g^y >0~ (kl>0 )$ in the north of the latitude where the disturbance occurred, and $c_g^y <0~ (kl<0 )$ in the south of the latitude. Therefore, $\overline{uv}<0$ in the north of the latitude, and  $\overline{uv}>0$ in the south of the stirring source, indicating 
			\begin{equation}\label{eq:uv_zonal}
				\frac{\partial \overline{uv}}{\partial y}<0.
			\end{equation}
			
			The momentum equation is
			\begin{equation*}\label{eq:u_mom}
				\frac{\partial u}{\partial t}+u\frac{\partial u}{\partial x}+v\frac{\partial u}{\partial y}-fv=-\frac{\partial \phi}{\partial t}
			\end{equation*}
			
			\begin{equation*}\label{eq:u_mom}
				\frac{\partial u}{\partial t}+\frac{\partial u^2}{\partial x}+\frac{\partial uv}{\partial y}-fv=-\frac{\partial \phi}{\partial t}
			\end{equation*}
			
			Taking zonally average of the above equation,
			 $$ \frac{1}{L}\int_{0}^{L}\left(  \frac{\partial u}{\partial t}+\frac{\partial u^2}{\partial x}+\frac{\partial uv}{\partial y}-fv\right) dx  = \frac{1}{L}\int_{0}^{L} -\frac{\partial \phi}{\partial t}dx$$
			 
			 \begin{equation}\label{eq:mom_zonal}
			 	\frac{\partial \overline{u}}{\partial t}+\frac{\partial \overline{uv}}{\partial y}=0.
			 \end{equation}
			 Put \eqref{eq:uv_zonal} into \eqref{eq:mom_zonal}, we could get
			 $$\frac{\partial \overline{u}}{\partial t}=-\frac{\partial \overline{uv}}{\partial y}>0,$$
			 hence there will be eastward flow at the given latitude.\\
			 
			 In the southern hemisphere, the eastward flow will also be generated.\\
			 
			 \item Consider two interacting Rossby waves in a single-layer barotropic fluid. Each Rossby wave generates a wave with a velocity amplitude of $10 ms^{-1}$. Being explicit about the assumptions you make, what is the acceleration of the mean flow? How long will it take to generate a mean flow of $20 m s^{-1}$?
			 
		\end{enumerate}
	
	
	\setcounter{enumi}{3}
	\item \textbf{ Geostrophic adjustment with a velocity jump.}
	%\begin{enumerate}[label=(\alph*)]
	\begin{enumerate}[label=\textbf{(\alph*)}]
		\setcounter{enumii}{0}
		\item Linearized potential vorticity in shallow water system.
		
		\textbf{Solution:} 
		The potential vorticity conservation is
		\begin{equation}\label{eq:PV}
			\frac{\partial Q'}{\partial t}+\bm u\cdot \nabla Q'=0,
		\end{equation}
		where $Q'=\frac{\zeta'+f_0}{h}$, $h$ is the depth of the fluid.
		In the linearised case with constant Coriolis parameter, the $Q'$ could be rewritten as
		$$Q'=\frac{\zeta'+f_0}{H+h'},$$
		where $H$ is the mean thickness, and $h'$ is the deviation free surface height. We can write
		$$Q'=\frac{\zeta'+f_0}{H(1+\frac{h'}{H})}\approx \frac{1}{H}(\zeta'+f_0)\left(1-\frac{h'}{H}\right),$$
		because $f_0\gg |\zeta'|$ and $H\gg |h'|$, so $h'/H$ close to $0$, and $\frac{1}{1+\frac{h'}{H}}\approx 1-\frac{h'}{H}$. In addition, if we neglect $\zeta'h'/H$, the $Q'$ could be rewritten as
		\begin{equation}\label{eq:PV_linear}
			Q'\approx \frac{1}{H}(\zeta'+f_0)\left(1-\frac{h'}{H}\right)=\frac{1}{H}\left(\zeta'+f_0-f_0\frac{h'}{H}\right)=\frac{f_0}{H}+\frac{q'}{H},
		\end{equation}
		where 
		\begin{equation}\label{eq:q'}
			q'=\zeta'-f_0\frac{h'}{H}.
		\end{equation}
		
		Put the \eqref{eq:PV_linear} and $q'$ into the \eqref{eq:PV}, we could get
		\begin{equation}\label{eq:q'_1}
		 \frac{\partial q'}{\partial t}+\bm u\cdot \nabla q'=0, 
		\end{equation}
		noting that $f_0$ is a constant, so the $\partial f_0/\partial t=0$ and $\nabla f_0=0.$ In addition, the advective  term $\bm u\cdot \nabla q'$ us second order term in perturbed quantities and so is neglected. Hence \eqref{eq:q'_1} becomes
		\begin{equation}\label{eq:q'_1}
		\frac{\partial q'}{\partial t}=0, 
		\end{equation}
		which is the linearized potential vorticity conservation with $q'=\zeta'-f_0\frac{h'}{H}$.
		
		\item \textbf{Solution:}  In the geostrophic balance, we have
		$$f_0 u = -g\frac{\partial h'}{\partial y},\qquad f_0 v = g\frac{\partial h'}{\partial x},$$
		
		and if we define 
		\begin{equation}\label{eq:psi_h'}
			\psi = g\frac{ h'}{f_0},
		\end{equation}
		
		so the $u$ and $v$ become 
		$$u = -\frac{\partial \psi}{\partial y}, \qquad v = \frac{\partial \psi}{\partial x}.$$
		Therefore, $\zeta'$ could be written as
		\begin{equation}\label{eq:zeta'}
			\zeta'=\frac{\partial v}{\partial x}-\frac{\partial u}{\partial y}=\frac{\partial^2 \psi}{\partial x^2}+\frac{\partial^2 \psi}{\partial y^2}=\nabla ^2\psi.
		\end{equation}		
		
		%According to \eqref{eq:psi_h'}, $$h'=\frac{f_0}{g}\psi,$$
		Put \eqref{eq:psi_h'} and \eqref{eq:zeta'} into \eqref{eq:q'}, we could get
		\begin{equation}\label{eq:q'_psi}
			q'=\zeta'-f_0\frac{h'}{H}= \nabla ^2\psi-\frac{f_0^2}{gH}\psi=\nabla ^2\psi-\frac{1}{L_d^2}\psi,
		\end{equation}
		where $L_d=\sqrt{gH}/f_0$.	\\
		
		\item \textbf{Solution:} In the initial state, $\eta =0, u=0,$
		and 
		\begin{equation}
			v(x)=v_0 \operatorname{sgn}(x)=\left\{
			\begin{aligned}
				& ~~~v_0, \qquad x>0,\\
				& -v_0, \qquad x<0,\\
			\end{aligned}\right.
		\end{equation}
		so the potential vorticity is
		\begin{equation}\label{eq:q_delta}
			q=\zeta -f_0\frac{\eta}{H}=\zeta =\frac{\partial v}{\partial x}-\frac{\partial u}{\partial y}=v_0\frac{\partial \operatorname{sgn}(x)}{\partial x}=2v_0\delta(x),
		\end{equation}
		where $\delta(x)$ is Dirac delta function.
		
		Put \eqref{eq:q_delta} into \eqref{eq:q'_psi}(where $q', \zeta', h'$ are $q, \zeta ,\eta $ respectively.), hence \eqref{eq:q'_psi} becomes 
		\begin{equation}
			\nabla ^2\psi-\frac{1}{L_d^2}\psi=2v_0\delta(x).
		\end{equation}
		If we only consider $x$ direction, we have
		\begin{equation}\label{eq:psi_x}
		\frac{\partial^2 \psi}{\partial x^2}-\frac{1}{L_d^2}\psi=2v_0\delta(x).
		\end{equation}
		
		Suppose that
		$$\psi(x) = \psi_0\operatorname{e}^{kx}+C,$$
		where $\psi_0$ and $C$ are constants, and put it into \eqref{eq:psi_x}, we could get
		$$\left(k^2-\frac{1}{L_d^2}\right)\psi_0\operatorname{e}^{kx} -\frac{C}{L_d^2}= 2v_0\delta(x)$$  
		
		If $x\neq 0$, $\delta(x)=0$, hence
		$$k^2=\frac{1}{L_d^2}, \text{ and } C=0.$$
		In order to get a stable solution, $k=-\frac{1}{L_d}$ when $x>0$, and $k=\frac{1}{L_d}$ when $x<0$, that is
		$$\psi(x) = \psi_0\operatorname{e}^{-\frac{|x|}{L_d}}.$$
		
		%Because the initial state of $v$ is $v_0\operatorname{sgn}(x)$, and we have
		We can calculate the derivation of $\psi$, that is
		$$\frac{\partial \psi}{\partial x}=-\frac{|x|}{x}\frac{\psi_0}{L_d}\operatorname{e}^{-\frac{|x|}{L_d}},$$
		$$\frac{\partial^2 \psi}{\partial x^2}=2\frac{\psi_0}{L_d}\delta(x) +\frac{\psi_0}{L_d^2}\operatorname{e}^{-\frac{|x|}{L_d}},$$
		and put them into \eqref{eq:psi_x}, we could get
		$$2\frac{\psi_0}{L_d}=2v_0,$$
		$$\psi_0=v_0L_d,$$
		hence 
		\begin{equation}
			\psi(x) = v_0L_d\operatorname{e}^{-\frac{|x|}{L_d}}.
		\end{equation}
		
		\begin{enumerate}[label=(\roman*)]
			\setcounter{enumii}{0}
			\item Find the equilibrium height and velocity fields at $t=\infty$ in the linear approximation.
			
			\textbf{Solution:} According to the definition of $\psi$, i.e. \eqref{eq:psi_h'}, the height $\eta$ here or ($h'$ in \eqref{eq:psi_h'}) is 
			$$\eta = \frac{f_0}{g}\psi= \frac{f_0v_0L_d}{g}\operatorname{e}^{-\frac{|x|}{L_d}}=\left\{\begin{aligned}
			 \frac{f_0v_0L_d}{g}\operatorname{e}^{-\frac{x}{L_d}}, \quad x>0, \\
			 \frac{f_0v_0L_d}{g}\operatorname{e}^{\frac{x}{L_d}}, \quad x<0.
			\end{aligned}\right..$$
			
			For the velocity,
			$$u=-\frac{\partial \psi}{\partial y}=0,$$
			$$v=\frac{\partial \psi}{\partial x}=-v_0\frac{|x|}{x}\operatorname{e}^{-\frac{|x|}{L_d}}=\left\{\begin{aligned}
			-v_0\operatorname{e}^{-\frac{x}{L_d}}, \quad x>0, \\
			v_0\operatorname{e}^{\frac{x}{L_d}}, \quad x<0.
			\end{aligned}\right.$$\\
			\item  What are the initial and final kinetic and potential energies?
			
			\textbf{Solution:} The height is $0$ in the initial state, so the initial potential energy is $0$, that is $$PE_I = 0.$$In the final state, the potential energy is
			$$PE_F = \frac{1}{2}g\int_{-\infty}^{\infty}\eta^2dx=\frac{1}{2}\left(\frac{f_0v_0L_d}{g}\right)^2\left(\int_{-\infty}^{0}\operatorname{e}^{\frac{2x}{L_d}} dx +\int_{0}^{\infty}\operatorname{e}^{-\frac{2x}{L_d}}dx \right) =\frac{L_d}{2}\left(\frac{f_0v_0L_d}{g}\right)^2.$$
			
			The kinetic energy in the initial state is 
			$$KE_I = \frac{1}{2}H \int_{-\infty}^{\infty}v_0^2dx \longrightarrow \infty,$$
			but for the final state, the kinetic energy is
			$$PE_F = \frac{1}{2}Hv_0^2 \left(\int_{-\infty}^{0}\operatorname{e}^{\frac{2x}{L_d}} dx +\int_{0}^{\infty}\operatorname{e}^{-\frac{2x}{L_d}}dx \right)=\frac{1}{2}Hv_0^2L_d.$$
			\end{enumerate}	
	\end{enumerate}	

\item \textbf{Geostrophic Theory}
	%\begin{enumerate}[label=(\alph*)]
	\begin{enumerate}[label=\textbf{(\alph*)}]
	\setcounter{enumii}{0}
	\item The potential vorticity equation for shallow water is
	\begin{equation}\label{eq:pv_Q}
	\frac{\operatorname{D}Q }{\operatorname{D}t} = \frac{\operatorname{D} }{\operatorname{D}t}\frac{\zeta+f}{h}=0
	\end{equation}
	where $f=f_0+\beta y$. $Q$ could be rewritten as
	$$Q=\frac{\zeta+f}{h}=\frac{\zeta+f}{H+\eta} = \frac{1}{H}~\frac{\zeta+f}{1+\eta/H}\approx\frac{1}{H}(\zeta+f)\left(1-\frac{\eta}{H}\right)\approx\frac{1}{H}\left(f_0+\beta y+\zeta -f_0\frac{\eta}{H}\right), $$
	where $|\eta| \ll H$ and $|\beta y|\ll f_0$, and terms $-\beta y\frac{\eta}{H} $ and $-\zeta \frac{\eta}{H} $ are neglected. Put it back into the \eqref{eq:pv_Q}, we obtain
	$$\frac{\operatorname{D} }{\operatorname{D}t} \frac{1}{H}\left(f_0+\beta y+\zeta -f_0\frac{\eta}{H}\right) =\frac{\operatorname{D} }{\operatorname{D}t} \left(\beta y+\zeta -f_0\frac{\eta}{H}\right) = 0,$$
	and define 
	\begin{equation}\label{eq:q_pv}
			q=\beta y+\zeta -f_0\frac{\eta}{H},
	\end{equation}

	hence 
	\begin{equation}\label{eq:q_conserv}
		\frac{\operatorname{D}q}{\operatorname{D}t} =0.
	\end{equation}
	
	In the geostrophic balance, we have
	$$f_0 u = -g\frac{\partial \eta}{\partial y},\qquad f_0 v = g\frac{\partial \eta}{\partial x},$$
	
	and if we define 
	\begin{equation}\label{eq:psi_eta}
	\psi = g\frac{\eta}{f_0},
	\end{equation}
	
	so the $u$ and $v$ become 
	$$u = -\frac{\partial \psi}{\partial y}, \qquad v = \frac{\partial \psi}{\partial x}.$$
	Therefore, $\zeta$ could be written as
	\begin{equation}\label{eq:zeta}
	\zeta=\frac{\partial v}{\partial x}-\frac{\partial u}{\partial y}=\frac{\partial^2 \psi}{\partial x^2}+\frac{\partial^2 \psi}{\partial y^2}=\nabla ^2\psi.
	\end{equation}
	
	Put \eqref{eq:zeta} and \eqref{eq:psi_eta} into \eqref{eq:q_pv}, we could get
	\begin{equation}\label{eq:q_pv_psi}
		q= \nabla^2\psi-\frac{f_0^2}{gH}\psi+\beta y= \nabla^2\psi-k_d^2\psi+\beta y,
	\end{equation}
	where $k_d = \frac{f_0}{\sqrt{gH}}$.
	
	Plug \eqref{eq:q_pv_psi} into \eqref{eq:q_conserv}, we can obtain
	$$\frac{\operatorname{D} }{\operatorname{D}t}(\nabla^2\psi-k_d^2\psi+\beta y)=\frac{\operatorname{D} }{\operatorname{D}t}(\nabla^2\psi-k_d^2\psi)+\frac{\operatorname{D} \beta y}{\operatorname{D}t}=\frac{\operatorname{D} }{\operatorname{D}t}(\nabla^2\psi-k_d^2\psi)+\beta v = 0.$$
	
	\item \textbf{Solution:} 
	
	The momentum equation in shallow water system is 
	\begin{equation}\label{eq:sw_mom}
		\frac { \partial \bm u } { \partial t } + \bm u \cdot \nabla \bm u + \bm f \times \bm u = - g \nabla \eta,
	\end{equation}
	where $\eta$ is the surface height of shallow water. The mass continuity equation is
	$$\frac { D h } { D t } + h \nabla \cdot \bm u = 0,$$
	here we suppose the bottom is flat (that is $\eta_b = 0$), then the total fluid thickness is $\eta =h$, and we have
	$$\frac { D \eta } { D t } + \eta \nabla \cdot \bm u = 0.$$
	If we rewrite $\eta =H+\Delta \eta$ ($H$ is the mean thickness, ), and multiply $1/H$ at both sides of the mass continuity equation, it will become
	\begin{equation}\label{eq:sw_mass}
		\frac { 1} { H } \frac { D \eta } { D t } + \left( 1+ \frac { \Delta \eta } { H } \right) \nabla \cdot \bm u = 0.
	\end{equation}
	We assume the scales of velocity, length, and time are
	$$ ( x ,y ) \sim L ,\quad   ( u ,v ) \sim U, \quad t\sim \frac{L}{U}.$$
	
	In geostrophic balance,
	$$-fv= -g\frac{\partial \eta}{\partial x},$$
	so $\Delta \eta $ has the scale 
		$$\Delta \eta\sim \frac{fUL}{g}=Ro ~H \frac{L^2}{L_d^2},$$
	where $Ro=\frac{U}{fL}, L_d=\frac{\sqrt{gH}}{f}$. Hence

		\begin{subequations}\label{eq:scales}
				\begin{equation}
					( x ,y ) = L ( \hat { x } ,\hat { y } ) ,\quad ( u ,v ) = U ( \hat { u } ,\hat { v } ), \quad t=\frac{L}{U}\hat{t},
				\end{equation}
				\begin{equation}
					\Delta \eta = Ro ~H \frac{L^2}{L_d^2} \hat{\eta}, \quad \eta = \left(1+Ro\frac{L^2}{L_d^2} \hat{\eta}\right)H.
				\end{equation}
		\end{subequations}

	%$$( x ,y ) = L ( \hat { x } ,\hat { y } ) ,\quad ( u ,v ) = U ( \hat { u } ,\hat { v } ), \quad t=\frac{L}{U}\hat{t},$$
	%$$\Delta \eta = Ro ~H \frac{L^2}{L_d^2} \hat{\eta}, \quad \eta = \left(1+Ro\frac{L^2}{L_d^2} \hat{\eta}\right)H$$
	Putting the scales \eqref{eq:scales} into \eqref{eq:sw_mom} and \eqref{eq:sw_mass}, we obtains
	\begin{equation}\label{eq:sw_mom_scale}
		Ro \left[ \frac { \partial \hat { \bm u } } { \partial \hat { t } } + ( \hat {  \bm u } \cdot \nabla ) \hat {  \bm u } \right] + \hat {  \bm f } \times \hat { \bm u } = - \nabla \hat { \eta },
	\end{equation}
	\begin{equation}\label{eq:sw_mass_scale}
		Ro \left( \frac { L } { L _ { d } } \right) ^ { 2} \frac { D \hat{ \eta } } { D \hat { t } } + \left[ 1+ R o \left( \frac { L } { L _ { d } } \right) ^ { 2} \hat { \eta } \right] \nabla \cdot\hat { \bm u } = 0.
	\end{equation}
	
	In large scale movement, we assume that	$Ro\ll 1$ and the scale of the motion is larger than the deformation scale, that is $L^2/L_d^2\gg 1$. We also assume that $Ro ~L^2/L_d^2 \sim O(1)$. Hence if we write in non-scaling form,  \eqref{eq:sw_mass_scale} and \eqref{eq:sw_mom_scale} will become 
	
	$$\bm f \times \bm u = - g \nabla \eta,$$
	$$\frac { D h } { D t } + h \nabla \cdot \bm u = 0.$$
	After some manipulations, the above equations could be rewritten as
	$$\frac{D }{Dt}\left(\frac{f}{h}\right)=0, \quad \bm f \times \bm u = - g \nabla \eta.$$
	\par
	The differences between the derivation of planetary geostrophic equation and quasi-geostrophic equations are that we assume the motion is significantly large than the deformation scale, that is $L^2\gg L^2_d$, in planetary geostrophic equations, but which doesn't hold in quasi-geostrophic equations.
	
	%The shallow water vorticity equation is
	%\begin{equation}\label{eq:swpv}
	%	\frac { \partial \zeta } { \partial t } + ( \bm u \cdot \nabla ) ( \zeta + f ) = - ( \zeta + f ) \nabla \cdot \bm u
	%\end{equation}
	
	\end{enumerate}


\item \textbf{Western boundary layers}

	Consider the barotropic vorticity equation in the form
	\begin{equation}\label{eq:wb_ctrl_original}
		\frac { \partial \zeta } { \partial t } + J ( \psi ,\zeta ) + \beta v = - r \zeta + \nu \nabla ^ { 2} \zeta + F ( x ,y )
	\end{equation}
	where $\zeta = \nabla ^ { 2} \psi ,v = \partial \psi / \partial x$, $r$ and $\nu$ are constants, and the flow is two-dimensional. We suppose the fluid is contained in a square container of side a, with $0\leq x \leq a$ and $0\leq y \leq a$ and $F = - A \sin \pi y / a$ where $A$ is a constant. We expect that the nonlinear term and both frictional terms are 'small', and we are interested in steady states for which $\partial \zeta / \partial t = 0$.
	
	\begin{enumerate}[label=\textbf{(\alph*)}]
		\setcounter{enumii}{0}
		\item Nondimensionalize these equations, and obtain estimates of the sizes of each term. State explicitly the conditions under which each of the frictional terms, and the nonlinear term, are indeed small.\\
		
		\textbf{Solution:}
		\eqref{eq:wb_ctrl_original} could be rewritten as
		\begin{equation}\label{eq:wb_ctrl_org}
			\frac { \partial \zeta } { \partial t } + u\frac{\partial \zeta}{\partial x} +  v\frac{\partial \zeta}{\partial y} + \beta v = -r\zeta + \nu \nabla ^ { 2} \zeta + F ( x ,y )
		\end{equation}
		
		\begin{equation}\label{eq:wb_ctrl_org1}
		\frac { \partial \nabla^{2}\psi} { \partial t } -\frac{\partial \psi}{\partial y}\frac{\partial  \nabla^{2}\psi}{\partial x} +  \frac{\partial \psi}{\partial x}\frac{\partial  \nabla^{2}\psi}{\partial y} + \beta \frac{\partial \psi}{\partial x} = -r\nabla^{2}\psi+ \nu \nabla^2\left(\nabla^{2}\psi\right) + F ( x ,y )
		\end{equation}
		
		Rescale the variables by setting
		$$(u,v)=U(\hat{u},\hat{v}), \quad x=a\hat{x},\quad y=a\hat{y},\quad t=\frac{a}{U}\hat{t}, \quad \zeta = \frac{U}{a}\hat{\zeta}, \quad \psi=aU \hat{\psi}, $$
		where the hatted variables are nondimensional and has the $O(1)$ order. Equation \eqref{eq:wb_ctrl_org} becomes
		\begin{equation}\label{eq:wb_ctrl_rescale}
			\frac{U^2}{a^2}\left(\frac { \partial \hat{\zeta} } { \partial \hat{t} } + \hat{u}\frac{\partial \hat{\zeta}}{\partial \hat{x}} +  \hat{v}\frac{\partial \hat{\zeta}}{\partial \hat{y}}\right) + U \beta \hat{v} = -r\frac{U}{a}\hat{\zeta} + \nu \frac{U}{a^3} \nabla^{2}\hat{\zeta} +F(a\hat{x},a\hat{y})
		\end{equation}
		
		The advective term of vorticity could be neglected if the ratio
		$$\frac{\frac{U^2}{a^2}}{U\beta}=\frac{U}{\beta a^2}\ll 1.$$
	
		The nonlinear term $\nu \nabla ^ { 2} \zeta$ is small comparing to the $\beta$-effect if
		$$\dfrac{\frac{\nu U}{a^3}}{U\beta} =\frac{\nu }{\beta a^3}\ll 1.$$
		
		The frictional term $-r\zeta$ could be neglected if
		$$\frac{r\frac{U}{a}}{U\beta}=\frac{r}{\beta a}\ll 1.$$ 
		
		\item Neglecting the nonlinear term and both frictional terms, obtain the solution to $\beta \frac{\partial \psi}{\partial x} = -A\sin (\pi y/a).$
Can this solution satisfy the boundary conditions needed
if the frictional terms are present. Explain briefly.\\
		\textbf{Solution:}
		Integrate at the both sides of $\beta \frac{\partial \psi}{\partial x} = -A\sin (\pi y/a)$, and we obtains
		$$\beta \int\frac{\partial \psi}{\partial x}dx = \int -A\sin (\pi y/a)dx,$$
		%$$\Longrightarrow \psi(x,y) -\psi(0,y) =  -\frac{A}{\beta}x~\sin (\pi y/a)+g(y),$$
		$$\Longrightarrow \psi(x,y) =  -\frac{A}{\beta}x~\sin (\pi y/a)+g(y),$$
		where $g(y)$ is an arbitrary function of integration that represents an arbitrary zonal flow.
		
		If we assume $\psi=0$ at $x=0$, then $$g(y)=0;$$
		If we assume $\psi=0$ at $x=a$, then $g(y)$ should be $$g(y)=\frac{Aa}{\beta}~\sin (\pi y/a).$$
		Hence
		$$\psi(x,y) =  -\frac{A}{\beta}x~\sin (\pi y/a) \quad or \quad \psi(x,y) =  \frac{A}{\beta}(a-x)~\sin (\pi y/a).$$
		
		If frictional term exists, then it is 
		$$-r\nabla^2\psi = -r\left(\frac{\partial^2 \psi}{\partial x^2} + \frac{\partial^2 \psi}{\partial y^2}\right)=-r\left( \frac{A\pi^2x}{\beta a^2 }\sin (\pi y/a) + g''(y)\right)$$
		
		I think it doesn't satisfy the boundary conditions needed by frictional terms, because frictional terms should be largest at boundaries $x=0$ or $x=a$, but small in the interior of the ocean.\\
		
		
		\item Suppose that $\nu = 0$, and neglect the nonlinear term, and assume that the term $r\zeta$ is indeed small but nonzero. Show that we can expect a boundary current on one
side of the ocean (which?) and estimate its thickness.\\
		
		\textbf{Solution:} If $\nu$ is small, and the non-linear terms are neglected, then equation \eqref{eq:wb_ctrl_org1} will becomes
		$$\beta \frac{\partial \psi}{\partial x} = -r\nabla^{2}\psi+ F(x ,y ),$$
		$$\Longrightarrow \beta \frac{\partial \psi}{\partial x} = -r\nabla^{2}\psi-A\sin (\pi y/a).$$
		
		Assume $\psi=\psi_I+\phi$, where $\phi$ is a boundary layer correction, and $\psi_I$ will satisfy the Sverdrup balance, we will have
		$$\beta \frac{\partial (\psi_I+\phi)}{\partial x} = -r\nabla^{2}(\psi_I+\phi)-A\sin (\pi y/a).$$
		$$\Longrightarrow \beta \frac{\partial \phi}{\partial x} +r\nabla^{2}(\psi_I+\phi) = 0.$$
		With the scale analysis, we have
		\begin{equation}\label{eq:phi}
			\frac{\partial \hat{\phi}}{\partial \hat{x}} +\frac{r}{\beta a} \left(\nabla^2 \hat{\psi}_I+ \frac{\partial^2 \hat{\phi}}{\partial\hat{x}^2} + \frac{\partial^2 \hat{\phi}}{ \partial \hat{y}^2} \right) = 0.
		\end{equation}		
		Because the boundary layer correction $\phi(\hat{x},\hat{y})$ will vary rapidly with $\hat{x}$, so we stretch the $\hat{x}$-coordinate and let
		\begin{subequations}\label{eq:coordinate_stretch}
			\begin{equation}
				\hat{x}=\epsilon \alpha \quad or \quad \hat{x}-1=\epsilon\alpha,
			\end{equation}
			\begin{equation}
				0<\alpha<\frac{1}{\epsilon}, \quad or \quad -\frac{1}{\epsilon}<\alpha<0,
			\end{equation}
		\end{subequations}
		
		where $\epsilon$ is a small parameter and $\alpha$ is stretched coordinate. We only suppose the $\phi$ will be $\phi(\alpha, \hat{y})$, hence \eqref{eq:phi} becomes
		\begin{equation}\label{eq:phi_alpha}
			\frac{1}{\epsilon}\frac{\partial \hat{\phi}}{\partial \alpha } +\frac{r}{\beta a} \left(\nabla^2 \hat{\psi}_I+ \frac{1}{\epsilon^2} \frac{\partial^2 \hat{\phi}}{ \partial\alpha^2} + \frac{\partial^2 \alpha}{ \partial \hat{y}^2} \right) = 0.
		\end{equation}
		We choose $\epsilon=\frac{r}{\beta a}$, hence the leading-order terms in \eqref{eq:phi_alpha} will be
		$$\frac{\partial^2 \hat{\phi}}{ \partial\alpha^2} + \frac{\partial \hat{\phi}}{ \partial\alpha} = 0,$$
		the solution of which is
		$$\hat{\phi} = A(\hat{y})+B(\hat{y})e^{-\alpha}.$$
		
		The solution indicates that $\phi$ will decay in positive $\alpha$ direction when $\alpha>0$, and we don't choose the solution when $\alpha<0$, for it will grow exponentially. Therefore, $x=\epsilon \alpha$ will be chosen, which means that the boundary current will appear in the western side of the ocean.
		
		If we set $A(\hat{y})=0$, and with dimensional variables the $\phi$ is
		$$\phi = B(y/a)e^{-x\beta /r}.$$
		
		If we choose $\psi_I(x,y) =  \frac{A}{\beta}(a-x)~\sin (\pi y/a)$, the $\psi$ satisfy
		$$\psi=\psi_I+\phi =0 \quad \text{at}\quad x=0,$$
		then 
		$$\phi = -\frac{Aa}{\beta}\sin (\pi y/a) e^{-x\beta /r},$$
		and we obtain
		$$\psi = \frac{A}{\beta}(a-x-ae^{-x\beta /r})~\sin (\pi y/a).$$
		
		
		\item Now suppose that $r = 0$, and neglect the nonlinear term. Estimate the size of the
boundary current that now arises.\\
		
		\textbf{Solution:} If $r=0$ and the non-linear terms are neglected ($\nu \neq 0$), then 
		
		then equation \eqref{eq:wb_ctrl_org1} will becomes
		$$\beta \frac{\partial \psi}{\partial x} - \nu \nabla^2\left(\nabla^{2}\psi\right)= F(x ,y ),$$
		$$\Longrightarrow \beta \frac{\partial \psi}{\partial x} - \nu \nabla^2\left(\nabla^{2}\psi\right) = A\sin (\pi y/a),$$
		
		$$\Longrightarrow \beta \frac{\partial \psi}{\partial x} - \nu \left( \frac{\partial^4 \psi}{\partial x^4} + \frac{\partial^4 \psi}{\partial y^4} + 2\frac{\partial^4 \psi}{\partial x^2\partial y^2} \right) = A\sin (\pi y/a).$$
		Assume $\psi=\psi_I+\phi$, where $\phi$ is a boundary layer correction, and $\psi_I$ will satisfy the Sverdrup balance, we will have
		$$\beta \frac{\partial \phi}{\partial x} - \nu \left(\nabla^2\left(\nabla^{2}\psi_I\right)+ \frac{\partial^4 \phi}{\partial x^4} + \frac{\partial^4 \phi}{\partial y^4} + 2\frac{\partial^4 \phi}{\partial x^2\partial y^2} \right) = 0.$$
		
		With the scale analysis, we obtains
		\begin{equation}\label{eq:phi_nu}
			\frac{\partial \hat{\phi}}{\partial \hat{x}} - \frac{\nu}{\beta a^3} \left(\nabla^2\left(\nabla^{2}\hat{\psi}_I\right)+ \frac{\partial^4 \hat{\phi}}{\partial \hat{x}^4} + \frac{\partial^4 \hat{\phi}}{\partial \hat{y}^4} + 2\frac{\partial^4 \hat{\phi}}{\partial \hat{x}^2\partial \hat{y}^2} \right) = 0.
		\end{equation}
		
		Similar as what we have done in part (c), here we also stretch the $\hat{x}$-coordinate as shown in equation \eqref{eq:coordinate_stretch}. Put them into \eqref{eq:phi_nu}, we obtain
		\begin{equation}\label{eq:phi_nu_alpha1}
		\frac{1}{\epsilon} \frac{\partial \hat{\phi}}{\partial \hat{\alpha}} - \frac{\nu}{\beta a^3} \left(\nabla^2\left(\nabla^{2}\hat{\psi}_I\right)+ \frac{1}{\epsilon ^4}\frac{\partial^4 \hat{\phi}}{\partial \hat{\alpha}^4} + \frac{\partial^4 \hat{\phi}}{\partial \hat{y}^4} + \frac{2}{\epsilon^2}\frac{\partial^4 \hat{\phi}}{\partial \hat{\alpha}^2\partial \hat{y}^2} \right) = 0.
		\end{equation}
		We choose $$\epsilon=\sqrt[3]{\frac{\nu}{\beta}}\frac{1}{a}, (\text{ or } \frac{\nu}{\beta a^3}= \epsilon^3)$$ hence the leading-order terms in \eqref{eq:phi_nu_alpha1} will be
		$$\frac{\partial^4 \hat{\phi}}{ \partial\alpha^4} - \frac{\partial \hat{\phi}}{ \partial\alpha} = 0,$$
		the solution of which is
		$$\hat{\phi} = A(y)+B(y)e^{\alpha}.$$
		
		Here we will choose $x-1=\epsilon \alpha$, that is to say in the eastern boundary. 
		
		If we set $A(\hat{y})=0$, and with dimensional variables the $\phi$ is
		$$\phi = B(y/a)e^{(x-a)/\sqrt[3]{\nu/\beta}}.$$
		
		If we choose $\psi_I(x,y) =  -\frac{A}{\beta}x~\sin (\pi y/a)$, the $\psi$ satisfy
		$$\psi=\psi_I+\phi =0 \quad \text{at}\quad x=a,$$
		then 
		$$\phi = \frac{Aa}{\beta}\sin (\pi y/a) e^{(x-a)/\sqrt[3]{\nu/\beta}},$$
		and we obtain
		$$\psi = \frac{A}{\beta}\left(ae^{(x-a)/\sqrt[3]{\nu/\beta}}-x\right)~\sin (\pi y/a).$$
		
	\end{enumerate}

\end{enumerate}

\end{document}
