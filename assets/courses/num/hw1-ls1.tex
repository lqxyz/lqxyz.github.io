\documentclass[a4paper]{article}
\usepackage{geometry}
\geometry{left=2.5cm,right=2.5cm,top=2.5cm,bottom=2.5cm}
\usepackage{listings}
\lstset{language=Matlab}%代码语言使用的是matlab
\lstset{breaklines}%自动将长的代码行换行排版
\lstset{extendedchars=false}%解决代码跨页时,章节标题,页眉等汉字不显示的问题
%\documentclass{article}
\usepackage{CJK}
\usepackage{amsmath}
\usepackage{amsfonts,amssymb}
\usepackage{graphicx}
\usepackage{enumerate}
\usepackage{booktabs}
\usepackage{threeparttable}
\usepackage{rotating}

\begin{document}

\begin{CJK*}{GBK}{song}
\title{\bf{线性方程组部分实验题1实习报告}}
\author{\bf{姓名:刘群\ 学号:2014211591 \ 院系: 地学中心}\\ \bf{Email}:\ liu-q14@mails.tsinghua.edu.cn \\ (本文档由\LaTeX{}编写)}
\maketitle

%\textbf{注: 实验中所得数据的表格在文档最后!}

\section{问题的描述}
\ \ \ \ 设$H_n=[h_{ij}] \in \mathbb{R}^{n \times n}$是Hilbert矩阵,即 $$h_{ij}=\frac{1}{i+j-1}\cdot$$
对$n=2,3,4,\cdot \cdot \cdot$\ (根据你的计算机性能选取合适的\ $n$, 建议算到\ $n = 15$左右)\\

\begin{enumerate}[(a)]
\item 取$x=\left(
\begin{array}{ccc}
1\\
\rotatebox{90}{$\cdots$} \\
1
\end{array}
\right) \in \mathbb{R}^n$, 及$b_n=H_n x.$\ 再用Gauss消去法和Cholesky分解法来求解$H_n y=b_n$, 看看误差有多大.\\ \label{a}
\item 计算条件数:\ cond$(H_n)_2.$ \\
\item 使用某种正则化方法改善(\ref{a})中的结果.
\end{enumerate}

\section{Gauss消去法和Cholesky分解(没有规则化)}
\subsection{原理介绍}

~~\ \ Gauss消去法是一种较为简单的求解线性代数方程组的方法, 但是它有一个要求, 就是要求在消元的过程中, 对角线元素不能出现0或者很小的数, 否则会造成结果的不稳定. 采用列主元消去法可以很大程度上减少这个问题, 列主元消去法会在每一列中选择一个绝对值最大的数来作为主元, 其实质就是在方程的两边同时乘以初等排列矩阵. 现将该方法介绍如下:


对于矩阵$H\in \mathbb{R}^{n\times n}, \text{如果}det(H)\neq 0$, 则存在一个排列阵$P(PP^T=PP=I)$,使得$PA=LU$,其中$L$是单位下三角矩阵,$U$ 是上三角矩阵.如果记
$$I_{ij}= \left(                 %左括号
            \begin{array}{cccccccccc}   %该矩阵一共4列,每一列都居中放置
            1& & & & & & & & &  \\  %第一行元素
             & 1& & & & & & & & \\  %第二行元素
             & &\rotatebox{-45}{$\cdots$}& & & &  & & &\\
             & & & 0 & & & & 1 & &   \\
             & & & & \rotatebox{-45}{$\cdots$} & & &  & & \\
             & & & & & 1 & & &  & \\  %第一行元素
             & & & & & &\rotatebox{-45}{$\cdots$} & &  &\\
             & & & 1 & & & & 0 & &   \\
             & & & & & & & & \rotatebox{-45}{$\cdots$}&   \\
             & & & & & & & & & 1\\
            \end{array}
            \right)                 %右括号
$$
表示第i行和第j行进行交换的初等排列阵, 则有在第$k$步进行消去变换计算$A^{(k+1)}$时, 我们先在$A^{(k)}$的第$k$列中,从第$k$个到第$n$个元素中选取$a_{i_kk}^{(k)}$, 使
$$ a_{i_kk}^{(k)}= \mathop{\text{max}}_{k\leq i \leq n}|a_{ik}^{(k)}| $$

由于$A$非奇异, 因此有$a_{i_kk}^{(k)}\neq 0$. 我们将$A^{(k)}$的第$i_k$行与第$k$行对调, 换行的过程相当于左乘$I_{i_kk}$,然后按照正常的Gauss消元法进行计算. 从而有经过$n-1$步的换行和消去后, $A$可以化成上三角矩阵$U$, 有
$$L_{n-1}^{-1}I_{i_{n-1}n-1}\cdots L_{2}^{-1}I_{i_{2}2}L_{1}^{-1}I_{i_{1}1}A=U$$
$$A=I_{i_{1}1}L_{1} I_{i_{2}2}L_{2} \cdots I_{i_{n-1}n-1}L_{n-1}U$$

可以证明,
$$P= I_{i_{n-1}n-1}\cdots I_{i_{2}2} I_{i_{1}1}$$

从而我们可以通过解两个线性方程组$Ly=Pb=b'$和$Ux=y$即可求出$x$, 其公式如下:
\begin{equation}\label{lu}
    \left\{
    \begin{array}{ll}
    y_1 = b'_1 & \\
    y_i=b'_i-\displaystyle{\sum_{k=1}^{i-1}}l_{ik}y_k, & i=2,\cdots,n,\\
    x_n = y_n / u_{nn}, & \\
    x_i=\left. \left(y_i-\displaystyle{\sum_{k=i+1}^{n}}u_{ki}x_k\right)\middle/ u_{ii}\right., & i=n-1,\cdots ,1. \\
    \end{array} \right.
\end{equation}\par
Cholesky分解的定义如下:如果矩阵$H \in \mathbb{R}^{n \times n}$是对称正定矩阵, 则存在唯一的对角元素为正的下三角矩阵$L$,使 $$H=LL^T.$$
这称为矩阵的Cholesky分解.并且, 根据矩阵相乘的规则, 即 \\
$$ H=LL^T= \left(                 %左括号
        \begin{array}{cccc}   %该矩阵一共4列,每一列都居中放置
            l_{11} & \  & \  & \   \\  %第一行元素
            l_{21} & l_{22} & \ & \ \\  %第二行元素
            \rotatebox{90}{$\cdots$} & \rotatebox{90}{$\cdots$}  & \rotatebox{-45}{$\cdots$} & \ \\
            l_{n1} & l_{n2} & \cdots & l_{nn} \\
        \end{array}
        \right)                 %右括号
        \
        \left(                 %左括号
        \begin{array}{cccc}   %该矩阵一共4列,每一列都居中放置
            l_{11} & l_{21}  & \cdots  & l_{n1}   \\  %第一行元素
            \  & l_{22} & \cdots & l_{n2} \\  %第二行元素
            \ & \  & \rotatebox{135}{$\cdots$} & \rotatebox{90}{$\cdots$}  \\
            \ & \ & \ & l_{nn} \\
        \end{array}
        \right) $$
我们可以得出$L$的计算公式为:\par

对于$j=1,2,\cdots,n$,有 %本段文字如果和下段之间有空行的话,第二段开头就会空两格
\begin{equation}\label{chol_L}
    \left\{
    \begin{array}{ll}
    l_{11}=a_{11}^{\frac{1}{2}}, & j=1 \\
    l_{i1}=a_{1i}/l_{11}, & i=2,3,\cdots,n \\
    l_{jj}=\left(a_{jj}-\displaystyle{\sum_{k=1}^{j-1}}l_{jk}^2\right)^{\frac{1}{2}}, & j=2,3,\cdots,n \ \\
    l_{ij}=\left. \left(a_{ij}-\displaystyle{\sum_{k=1}^{j-1}}l_{ik}l_{jk}\right)\middle/ l_{jj}\right., & i=j+1,\cdots ,n \\
    \end{array} \right.
\end{equation}
从而我们可以通过解两个线性方程组$Ly=b$和$L^Tx=y$即可求出$x$, 其公式如下:
\begin{equation}\label{chol}
    \left\{
    \begin{array}{ll}
    y_1=b_1/l_{11} & \\
    y_i=\left. \left(b_i-\displaystyle{\sum_{k=1}^{i-1}}l_{ik}y_k\right)\middle/ l_{ii} \right., & i=2,3,\cdots,n,\\
    x_n = y_n / l_{nn},& \\
    x_i=\left. \left(y_i-\displaystyle{\sum_{k=i+1}^{n}}l_{ki}x_k\right)\middle/ l_{ii}\right., & i=n-1,n-2,\cdots,1. \\
    \end{array} \right.
\end{equation}


\subsection{方案设计}

~~~~这里我们首先采用自己编写的函数lu\_pivot\_in\_col.m对矩阵进行列主元的三角分解, 该函数需要一个方阵作为输入参数, 会返回两个参数, 一个是LU矩阵, 一个是$p$ 向量. 其中LU矩阵存放的是LU分解的L矩阵和U矩阵, 这里我们采用了紧凑存储的方法, 原因是L的对角线元素全是1, 因此我们可以不必存储, 只存储其对角线下方的部分, 而在计算时考虑到L的对角线为1这一性质即可. U由于是上三角矩阵, 对角线元素实际上是U的对角线的结果. 返回值$p$是一个$n$维向量, 它告诉我们A的哪些行进行了交换, 利用$b(p)$可以很容易得到交换后的$b$向量.\par
之后我们编写了另一个函数Gauss.m来进行求解线性代数方程组$Ax=b$, 该函数有两个参数, 分别是$A$和$b$, 返回值是$x$. 我们利用了公式\eqref{lu}进行求解.\par
对于Cholesky分解, 我编写了函数cholesky\_llt.m进行求解, 该函数需要一个对称正定的矩阵A作为输入参数, 返回值是下三角矩阵L, 满足$LL^T=A$,在程序中,我们主要采用了公式\eqref{chol_L}求解L. 然后我们用另一个函数Cholesky.m来求解$Ax=b$, 该程序主要是用了公式\eqref{chol}来求解$x$.\par
最后我们编写了一个函数Gauss\_Cholesky\_hw\_1.m来调用上面的函数, 并将结果写入了一个Excel表格里.
\subsection{计算结果及分析}
~~~~从Table \ref{tab:Gauss} 和 Table \ref{tab:Chol}中我们以看出, 当n比较小时,无论是用列主元的Gauss消去法还是Cholesky分解的方法, 计算的结果都比较准确,基本都在1左右, 但是当n比较大时, 可以发现,二者的误差都逐渐增大. 特别是当$n=14,15$时,采用Cholesky分解的方法计算出的$x$与真值的差距是巨大的, 此时计算结果的某些分量已经达到了-1386.8, 是真值的1000多倍, 这个误差是相当大的. 同时, 我们可以看到, 即使采用列主元的Gauss消去法, 当$n=15$时, 计算结果的误差也已经达到了6.8, 是真值的近7倍, 误差也是相当大的.

\begin{center}
\begin{table*}[!htbp]
\resizebox{\textwidth}{!}{ %
\begin{threeparttable}[!htpb]
 \caption{\label{tab:Gauss}n=2-15时Gauss消去法计算所得的结果}
 \begin{tabular}{ccccc ccccc ccccc}
 \toprule
 &n=2&n=3 &n=4 &n=5 &n=6 &n=7 &n=8 &n=9 &n=10 &n=11 &n=12 &n=13 &n=14 &n=15 \\ \hline
$x_{1}$	&	1	&	1	&	1	&	1	&	1	&	1	&	1	&	1	&	0.999999999	&	0.999999984	&	0.999999921	&	0.999999975	&	0.999999841	&	0.999999944	\\
$x_{2}$	&	1	&	1	&	1	&	1	&	1	&	1.000000001	&	0.999999999	&	1.000000001	&	1.000000055	&	1.000001675	&	1.000010096	&	1.000003249	&	1.000019848	&	1.000009542	\\
$x_{3}$	&		&	1	&	1	&	1	&	1	&	0.999999995	&	1.000000012	&	0.999999976	&	0.999998824	&	0.999956045	&	0.999681244	&	0.999895713	&	0.999424804	&	0.999600436	\\
$x_{4}$	&		&		&	1	&	1	&	1	&	1.00000002	&	0.999999936	&	1.000000177	&	1.000010779	&	1.000496008	&	1.004358381	&	1.001470786	&	1.00630048	&	1.007183282	\\
$x_{5}$	&		&		&		&	1	&	1	&	0.999999964	&	1.000000172	&	0.99999935	&	0.999948201	&	0.99702249	&	0.967949318	&	0.988670157	&	0.975819634	&	0.930952619	\\
$x_{6}$	&		&		&		&		&	1	&	1.000000031	&	0.999999757	&	1.000001324	&	1.000143329	&	1.010535419	&	1.141212086	&	1.05322407	&	0.927171589	&	1.395123538	\\
$x_{7}$	&		&		&		&		&		&	0.99999999	&	1.000000173	&	0.999998487	&	0.999763483	&	0.976936364	&	0.60559056	&	0.838019841	&	2.122456722	&	-0.415367935	\\
$x_{8}$	&		&		&		&		&		&		&	0.999999951	&	1.000000908	&	1.000229733	&	1.031589063	&	1.715462185	&	1.329438053	&	-4.190702077	&	4.182171234	\\
$x_{9}$	&		&		&		&		&		&		&		&	0.999999777	&	0.999878845	&	0.973654119	&	0.159620838	&	0.548198707	&	14.58369777	&	-3.139483841	\\
$x_{10}$	&		&		&		&		&		&		&		&		&	1.000026752	&	1.012232893	&	1.616512671	&	1.412437251	&	-21.42182258	&	2.776223143	\\
$x_{11}$	&		&		&		&		&		&		&		&		&		&	0.997575928	&	0.743287496	&	0.759926419	&	24.83668839	&	4.474986798	\\
$x_{12}$	&		&		&		&		&		&		&		&		&		&		&	1.046315261	&	1.080571363	&	-14.87182213	&	-5.997088687	\\
$x_{13}$	&		&		&		&		&		&		&		&		&		&		&		&	0.988144421	&	7.032767487	&	6.789625421	\\
$x_{14}$	&		&		&		&		&		&		&		&		&		&		&		&		&	0	&	-1.424390455	\\
$x_{15}$	&		&		&		&		&		&		&		&		&		&		&		&		&		&	1.420455005	\\


 \bottomrule
\end{tabular}
%\small Note: Robust standard errors in parentheses. Intercept included but not reported. \begin{tablenotes}\item[*] significant at 5\% level \item[**] significant at 10\% level\end{tablenotes}
\end{threeparttable}}%
\end{table*}
\end{center}

%%%%%%%%%%%%%%%%%%%%%%%%%%%%%%%%%%%%%%%%%%%%%%% Cholesky results %%%%%%%%%%%%%%%%%%%%%%%%%%%%%%%%%%%%%%%
\begin{center}
\begin{table*}[!htpb]
\resizebox{\textwidth}{!}{ %
\begin{threeparttable}[!htpb]
 \caption{\label{tab:Chol}n=2-15时Cholesky分解法计算所得的结果}
 \begin{tabular}{ccccc ccccc ccccc}
 \toprule
 &n=2&n=3 &n=4 &n=5 &n=6 &n=7 &n=8 &n=9 &n=10 &n=11 &n=12 &n=13 &n=14 &n=15 \\ \hline
$x_{1}$	&	1	&	1	&	1	&	1	&	1	&	1	&	1	&	1	&	1	&	1.000000003	&	0.999999956	&	1.000000024	&	1.000002451	&	1.000017654	\\
$x_{2}$	&	1	&	1	&	1	&	1	&	1	&	1	&	1.000000004	&	1.000000032	&	0.999999994	&	0.999999743	&	1.000005452	&	0.999995851	&	0.99961627	&	0.997243176	\\
$x_{3}$	&		&	1	&	1	&	1	&	1	&	0.999999998	&	0.99999995	&	0.999999462	&	1.00000014	&	1.00000646	&	0.999831576	&	1.00017182	&	1.014866935	&	1.106468084	\\
$x_{4}$	&		&		&	1	&	1	&	1.000000001	&	1.000000009	&	1.00000027	&	1.000003838	&	0.999998708	&	0.999929629	&	1.002260478	&	0.996990027	&	0.749868125	&	-0.784817299	\\
$x_{5}$	&		&		&		&	1	&	0.999999999	&	0.999999983	&	0.999999277	&	0.999985939	&	1.000006214	&	1.000409988	&	0.983640437	&	1.028072151	&	3.281402001	&	17.2130681	\\
$x_{6}$	&		&		&		&		&	1	&	1.000000015	&	1.000001019	&	1.000028682	&	0.999982858	&	0.998586045	&	1.071085628	&	0.84320811	&	-11.63990446	&	-88.4284312	\\
$x_{7}$	&		&		&		&		&		&	0.999999995	&	0.999999277	&	0.999967089	&	1.000028139	&	1.003027305	&	0.803852299	&	1.559881072	&	46.3531319	&	320.3718586	\\
$x_{8}$	&		&		&		&		&		&		&	1.000000204	&	1.000019863	&	0.999972846	&	0.995933591	&	1.352021842	&	-0.323020933	&	-108.1650276	&	-764.200155	\\
$x_{9}$	&		&		&		&		&		&		&		&	0.999995096	&	1.000014218	&	1.003333667	&	0.590433724	&	3.092833447	&	179.8737222	&	1250.307986	\\
$x_{10}$	&		&		&		&		&		&		&		&		&	0.999996884	&	0.998475602	&	1.297916548	&	-1.191906961	&	-197.7776513	&	-1386.759858	\\
$x_{11}$	&		&		&		&		&		&		&		&		&		&	1.000297968	&	0.876895003	&	2.458332867	&	147.0800464	&	1029.755894	\\
$x_{12}$	&		&		&		&		&		&		&		&		&		&		&	1.02205708	&	0.441917381	&	-65.99397825	&	-487.6930875	\\
$x_{13}$	&		&		&		&		&		&		&		&		&		&		&		&	1.09352518	&	17.93676655	&	140.5586004	\\
$x_{14}$	&		&		&		&		&		&		&		&		&		&		&		&		&	-0.712859484	&	-21.56977351	\\
$x_{15}$	&		&		&		&		&		&		&		&		&		&		&		&		&		&	3.125	\\
 \bottomrule
\end{tabular}
%\small Note: Robust standard errors in parentheses. Intercept included but not reported. \begin{tablenotes}\item[*] significant at 5\% level \item[**] significant at 10\% level\end{tablenotes}
\end{threeparttable}}%
\end{table*}
\end{center}



下面我们看一下剩余向量和误差向量的情况, $y$为我们求得的结果, $x$为方程的真值, 记剩余向量为$r_n=b-Hy$, 误差向量为$\Delta x= y-x$.从Table \ref{tab:Gauss_r}, Table \ref{tab:Chol_r}, Table \ref{tab:Gauss_dx}和Table \ref{tab:Chol_dx}中我们可以看出, 当不采用正则化方法, 单纯采用Gauss消元法和Cholesky分解法时虽然残差向量比较小,但是由于矩阵的条件数很大, 因此误差向量仍然比较大, 特别是Cholesky分解法, 当$n=14,15$时, 其误差向量变得非常大, 偏差是真值的1000多倍.
%%%%%%%%%%%%%%%%%%%%%%%%%%%%%%%%%%%%%%%%%%%%%%% Gauss r  %%%%%%%%%%%%%%%%%%%%%%%%%%%%%%%%%%%%%%%
\begin{center}
\begin{table*}[!htpb]
\resizebox{\textwidth}{!}{ %
\begin{threeparttable}[!htpb]
 \caption{\label{tab:Gauss_r}n=2-15时Gauss消去法计算结果的残差向量$r=b-Hy$}
 \begin{tabular}{ccccc ccccc ccccc}
 \toprule

 &n=2&n=3 &n=4 &n=5 &n=6 &n=7 &n=8 &n=9 &n=10 &n=11 &n=12 &n=13 &n=14 &n=15 \\ \hline
$r_{1}$	&	0	&	0	&	0	&	0	&	0	&	4.44089E-16	&	4.44089E-16	&	0	&	4.44089E-16	&	-4.44089E-16	&	4.44089E-16	&	4.44089E-16	&	4.44089E-16	&	0	\\
$r_{2}$	&	0	&	0	&	2.22045E-16	&	0	&	0	&	0	&	2.22045E-16	&	0	&	0	&	0	&	0	&	0	&	0	&	0	\\
$r_{3}$	&		&	0	&	0	&	0	&	-2.22045E-16	&	0	&	2.22045E-16	&	2.22045E-16	&	0	&	2.22045E-16	&	2.22045E-16	&	2.22045E-16	&	8.88178E-16	&	-2.22045E-16	\\
$r_{4}$	&		&		&	1.11022E-16	&	0	&	0	&	2.22045E-16	&	0	&	2.22045E-16	&	-2.22045E-16	&	2.22045E-16	&	4.44089E-16	&	-2.22045E-16	&	4.44089E-16	&	0	\\
$r_{5}$	&		&		&		&	-2.22045E-16	&	0	&	2.22045E-16	&	-2.22045E-16	&	0	&	0	&	-2.22045E-16	&	6.66134E-16	&	0	&	0	&	-2.22045E-16	\\
$r_{6}$	&		&		&		&		&	0	&	0	&	2.22045E-16	&	0	&	0	&	-2.22045E-16	&	-2.22045E-16	&	0	&	2.22045E-16	&	0	\\
$r_{7}$	&		&		&		&		&		&	0	&	0	&	-2.22045E-16	&	-1.11022E-16	&	0	&	2.22045E-16	&	0	&	-2.22045E-16	&	-2.22045E-16	\\
$r_{8}$	&		&		&		&		&		&		&	0	&	1.11022E-16	&	0	&	-1.11022E-16	&	0	&	0	&	-2.22045E-16	&	0	\\
$r_{9}$	&		&		&		&		&		&		&		&	0	&	1.11022E-16	&	1.11022E-16	&	1.11022E-16	&	1.11022E-16	&	1.11022E-16	&	0	\\
$r_{10}$	&		&		&		&		&		&		&		&		&	1.11022E-16	&	-1.11022E-16	&	0	&	1.11022E-16	&	-3.33067E-16	&	-1.11022E-16	\\
$r_{11}$	&		&		&		&		&		&		&		&		&		&	0	&	1.11022E-16	&	0	&	1.11022E-16	&	0	\\
$r_{12}$	&		&		&		&		&		&		&		&		&		&		&	1.11022E-16	&	0	&	0	&	0	\\
$r_{13}$	&		&		&		&		&		&		&		&		&		&		&		&	1.11022E-16	&	-1.11022E-16	&	-2.22045E-16	\\
$r_{14}$	&		&		&		&		&		&		&		&		&		&		&		&		&	0	&	-1.11022E-16	\\
$r_{15}$	&		&		&		&		&		&		&		&		&		&		&		&		&		&	-1.11022E-16	\\

 \bottomrule
\end{tabular}
%\small Note: Robust standard errors in parentheses. Intercept included but not reported. \begin{tablenotes}\item[*] significant at 5\% level \item[**] significant at 10\% level\end{tablenotes}
\end{threeparttable}}%
\end{table*}
\end{center}

%%%%%%%%%%%%%%%%%%%%%%%%%%%%%%%%%%%%%%%%%%%%%%% Chol r %%%%%%%%%%%%%%%%%%%%%%%%%%%%%%%%%%%%%%%
\begin{center}
\begin{table*}[!htbp]
\resizebox{\textwidth}{!}{ %
\begin{threeparttable}[!htbp]
 \caption{\label{tab:Chol_r}n=2-15时Cholesky分解法计算结果的残差向量$r=b-Hy$}
 \begin{tabular}{ccccc ccccc ccccc}
 \toprule

 &n=2&n=3 &n=4 &n=5 &n=6 &n=7 &n=8 &n=9 &n=10 &n=11 &n=12 &n=13 &n=14 &n=15 \\ \hline
$r_{1}$	&	0	&	2.22045E-16	&	0	&	0	&	0	&	4.44089E-16	&	0	&	0	&	-4.44089E-16	&	4.44089E-16	&	-4.44089E-16	&	0	&	-4.44089E-16	&	2.93099E-14	\\
$r_{2}$	&	0	&	0	&	0	&	0	&	2.22045E-16	&	-2.22045E-16	&	-2.22045E-16	&	2.22045E-16	&	0	&	0	&	4.44089E-16	&	-4.44089E-16	&	-2.22045E-15	&	4.08562E-14	\\
$r_{3}$	&		&	0	&	0	&	0	&	0	&	-2.22045E-16	&	-2.22045E-16	&	2.22045E-16	&	-2.22045E-16	&	-2.22045E-16	&	2.22045E-16	&	0	&	-5.77316E-15	&	2.75335E-14	\\
$r_{4}$	&		&		&	0	&	0	&	-1.11022E-16	&	0	&	0	&	2.22045E-16	&	0	&	2.22045E-16	&	4.44089E-16	&	2.22045E-16	&	6.66134E-16	&	3.01981E-14	\\
$r_{5}$	&		&		&		&	0	&	0	&	-1.11022E-16	&	2.22045E-16	&	0	&	0	&	0	&	0	&	2.22045E-16	&	-4.21885E-15	&	1.28786E-14	\\
$r_{6}$	&		&		&		&		&	1.11022E-16	&	-1.11022E-16	&	-1.11022E-16	&	0	&	-2.22045E-16	&	2.22045E-16	&	0	&	0	&	6.66134E-16	&	2.64233E-14	\\
$r_{7}$	&		&		&		&		&		&	0	&	-1.11022E-16	&	1.11022E-16	&	-1.11022E-16	&	-2.22045E-16	&	2.22045E-16	&	0	&	-2.22045E-15	&	1.11022E-15	\\
$r_{8}$	&		&		&		&		&		&		&	-2.22045E-16	&	1.11022E-16	&	0	&	-2.22045E-16	&	-2.22045E-16	&	-2.22045E-16	&	-2.44249E-15	&	4.88498E-15	\\
$r_{9}$	&		&		&		&		&		&		&		&	-1.11022E-16	&	-1.11022E-16	&	0	&	0	&	-1.11022E-16	&	-2.55351E-15	&	2.88658E-15	\\
$r_{10}$	&		&		&		&		&		&		&		&		&	-1.11022E-16	&	0	&	0	&	0	&	-4.44089E-16	&	9.10383E-15	\\
$r_{11}$	&		&		&		&		&		&		&		&		&		&	-1.11022E-16	&	0	&	2.22045E-16	&	1.11022E-15	&	1.40998E-14	\\
$r_{12}$	&		&		&		&		&		&		&		&		&		&		&	1.11022E-16	&	0	&	0	&	6.43929E-15	\\
$r_{13}$	&		&		&		&		&		&		&		&		&		&		&		&	2.22045E-16	&	5.55112E-16	&	9.76996E-15	\\
$r_{14}$	&		&		&		&		&		&		&		&		&		&		&		&		&	-6.73905E-14	&	-4.91607E-13	\\
$r_{15}$	&		&		&		&		&		&		&		&		&		&		&		&		&		&	-2.61735E-12	\\

 \bottomrule
\end{tabular}
%\small Note: Robust standard errors in parentheses. Intercept included but not reported. \begin{tablenotes}\item[*] significant at 5\% level \item[**] significant at 10\% level\end{tablenotes}
\end{threeparttable}}%
\end{table*}
\end{center}

%%%%%%%%%%%%%%%%%%%%%%%%%%%%%%%%%%%%%%%%%%%%%%% Gauss delta x  %%%%%%%%%%%%%%%%%%%%%%%%%%%%%%%%%%%%%%%
\begin{center}
\begin{table*}[!htpb]
\resizebox{\textwidth}{!}{ %
\begin{threeparttable}[!htpb]
 \caption{\label{tab:Gauss_dx}n=2-15时Gauss消去法计算结果的误差$\Delta x=y-x$}
 \begin{tabular}{ccccc ccccc ccccc}
 \toprule

 &n=2&n=3 &n=4 &n=5 &n=6 &n=7 &n=8 &n=9 &n=10 &n=11 &n=12 &n=13 &n=14 &n=15 \\ \hline
$\Delta x_{1}$	&	4.44089E-16	&	4.44089E-16	&	-7.10543E-15	&	-2.73115E-14	&	-7.32525E-13	&	-1.3022E-11	&	1.71303E-11	&	-1.87108E-11	&	-6.26144E-10	&	-1.57617E-08	&	-7.93263E-08	&	-2.54816E-08	&	-1.59191E-07	&	-5.56988E-08	\\
$\Delta x_{2}$	&	-6.66134E-16	&	-9.99201E-16	&	8.85958E-14	&	2.16049E-13	&	2.11939E-11	&	5.22731E-10	&	-9.09995E-10	&	1.39681E-09	&	5.46692E-08	&	1.67488E-06	&	1.00961E-05	&	3.24905E-06	&	1.98477E-05	&	9.54183E-06	\\
$\Delta x_{3}$	&		&	0	&	-2.27041E-13	&	-1.59872E-13	&	-1.44818E-10	&	-5.05746E-09	&	1.18254E-08	&	-2.44254E-08	&	-1.17566E-06	&	-4.3955E-05	&	-0.000318756	&	-0.000104287	&	-0.000575196	&	-0.000399564	\\
$\Delta x_{4}$	&		&		&	1.52989E-13	&	-5.53668E-13	&	3.79525E-10	&	1.97236E-08	&	-6.38286E-08	&	1.7658E-07	&	1.07791E-05	&	0.000496008	&	0.004358381	&	0.001470786	&	0.00630048	&	0.007183282	\\
$\Delta x_{5}$	&		&		&		&	5.54667E-13	&	-4.21393E-10	&	-3.62458E-08	&	1.71606E-07	&	-6.49577E-07	&	-5.17989E-05	&	-0.00297751	&	-0.032050682	&	-0.011329843	&	-0.024180366	&	-0.069047381	\\
$\Delta x_{6}$	&		&		&		&		&	1.66833E-10	&	3.1379E-08	&	-2.42657E-07	&	1.3236E-06	&	0.000143329	&	0.010535419	&	0.141212086	&	0.05322407	&	-0.072828411	&	0.395123538	\\
$\Delta x_{7}$	&		&		&		&		&		&	-1.03186E-08	&	1.72648E-07	&	-1.51301E-06	&	-0.000236517	&	-0.023063636	&	-0.39440944	&	-0.161980159	&	1.122456722	&	-1.415367935	\\
$\Delta x_{8}$	&		&		&		&		&		&		&	-4.87136E-08	&	9.08402E-07	&	0.000229733	&	0.031589063	&	0.715462185	&	0.329438053	&	-5.190702077	&	3.182171234	\\
$\Delta x_{9}$	&		&		&		&		&		&		&		&	-2.22957E-07	&	-0.000121155	&	-0.026345881	&	-0.840379162	&	-0.451801293	&	13.58369777	&	-4.139483841	\\
$\Delta x_{10}$	&		&		&		&		&		&		&		&		&	2.6752E-05	&	0.012232893	&	0.616512671	&	0.412437251	&	-22.42182258	&	1.776223143	\\
$\Delta x_{11}$	&		&		&		&		&		&		&		&		&		&	-0.002424072	&	-0.256712504	&	-0.240073581	&	23.83668839	&	3.474986798	\\
$\Delta x_{12}$	&		&		&		&		&		&		&		&		&		&		&	0.046315261	&	0.080571363	&	-15.87182213	&	-6.997088687	\\
$\Delta x_{13}$	&		&		&		&		&		&		&		&		&		&		&		&	-0.011855579	&	6.032767487	&	5.789625421	\\
$\Delta x_{14}$	&		&		&		&		&		&		&		&		&		&		&		&		&	-1	&	-2.424390455	\\
$\Delta x_{15}$	&		&		&		&		&		&		&		&		&		&		&		&		&		&	0.420455005	\\

 \bottomrule
\end{tabular}
%\small Note: Robust standard errors in parentheses. Intercept included but not reported. \begin{tablenotes}\item[*] significant at 5\% level \item[**] significant at 10\% level\end{tablenotes}
\end{threeparttable}}%
\end{table*}
\end{center}

%%%%%%%%%%%%%%%%%%%%%%%%%%%%%%%%%%%%%%%%%%%%%%% Chol delta x %%%%%%%%%%%%%%%%%%%%%%%%%%%%%%%%%%%%%%%
\begin{center}
\begin{table*}[!htpb]
\resizebox{\textwidth}{!}{ %
\begin{threeparttable}[!htpb]
 \caption{\label{tab:Chol_dx}n=2-15时Cholesky分解法计算结果的误差向量$\Delta x=y-x$}
 \begin{tabular}{ccccc ccccc ccccc}
 \toprule

 &n=2&n=3 &n=4 &n=5 &n=6 &n=7 &n=8 &n=9 &n=10 &n=11 &n=12 &n=13 &n=14 &n=15 \\ \hline
$\Delta x_{1}$	&	4.44089E-16	&	2.22045E-16	&	-7.99361E-15	&	3.37508E-14	&	-1.57718E-12	&	-5.39546E-12	&	-7.30218E-11	&	-4.61957E-10	&	7.032E-11	&	2.54623E-09	&	-4.39606E-08	&	2.37187E-08	&	2.45052E-06	&	1.76542E-05	\\
$\Delta x_{2}$	&	-7.77156E-16	&	-7.77156E-16	&	9.30367E-14	&	-6.94222E-13	&	4.43434E-11	&	2.279E-10	&	3.87268E-09	&	3.18366E-08	&	-6.38709E-09	&	-2.56847E-07	&	5.45222E-06	&	-4.14859E-06	&	-0.00038373	&	-0.002756824	\\
$\Delta x_{3}$	&		&	2.22045E-16	&	-2.32703E-13	&	3.14349E-12	&	-2.96859E-10	&	-2.27604E-09	&	-5.01765E-08	&	-5.38188E-07	&	1.39941E-07	&	6.46042E-06	&	-0.000168424	&	0.00017182	&	0.014866935	&	0.106468084	\\
$\Delta x_{4}$	&		&		&	1.55209E-13	&	-4.89664E-12	&	7.66126E-10	&	9.07199E-09	&	2.69923E-07	&	3.83755E-06	&	-1.29232E-06	&	-7.03708E-05	&	0.002260478	&	-0.003009973	&	-0.250131875	&	-1.784817299	\\
$\Delta x_{5}$	&		&		&		&	2.44804E-12	&	-8.40536E-10	&	-1.69396E-08	&	-7.2327E-07	&	-1.4061E-05	&	6.21446E-06	&	0.000409988	&	-0.016359563	&	0.028072151	&	2.281402001	&	16.2130681	\\
$\Delta x_{6}$	&		&		&		&		&	3.29605E-10	&	1.48451E-08	&	1.01948E-06	&	2.86823E-05	&	-1.71424E-05	&	-0.001413955	&	0.071085628	&	-0.15679189	&	-12.63990446	&	-89.4284312	\\
$\Delta x_{7}$	&		&		&		&		&		&	-4.92899E-09	&	-7.23219E-07	&	-3.29114E-05	&	2.81389E-05	&	0.003027305	&	-0.196147701	&	0.559881072	&	45.3531319	&	319.3718586	\\
$\Delta x_{8}$	&		&		&		&		&		&		&	2.03509E-07	&	1.98632E-05	&	-2.71544E-05	&	-0.004066409	&	0.352021842	&	-1.323020933	&	-109.1650276	&	-765.200155	\\
$\Delta x_{9}$	&		&		&		&		&		&		&		&	-4.9042E-06	&	1.42184E-05	&	0.003333667	&	-0.409566276	&	2.092833447	&	178.8737222	&	1249.307986	\\
$\Delta x_{10}$	&		&		&		&		&		&		&		&		&	-3.11622E-06	&	-0.001524398	&	0.297916548	&	-2.191906961	&	-198.7776513	&	-1387.759858	\\
$\Delta x_{11}$	&		&		&		&		&		&		&		&		&		&	0.000297968	&	-0.123104997	&	1.458332867	&	146.0800464	&	1028.755894	\\
$\Delta x_{12}$	&		&		&		&		&		&		&		&		&		&		&	0.02205708	&	-0.558082619	&	-66.99397825	&	-488.6930875	\\
$\Delta x_{13}$	&		&		&		&		&		&		&		&		&		&		&		&	0.09352518	&	16.93676655	&	139.5586004	\\
$\Delta x_{14}$	&		&		&		&		&		&		&		&		&		&		&		&		&	-1.712859484	&	-22.56977351	\\
$\Delta x_{15}$	&		&		&		&		&		&		&		&		&		&		&		&		&		&	2.125	\\

 \bottomrule
\end{tabular}
%\small Note: Robust standard errors in parentheses. Intercept included but not reported. \begin{tablenotes}\item[*] significant at 5\% level \item[**] significant at 10\% level\end{tablenotes}
\end{threeparttable}}%
\end{table*}
\end{center}



\section{矩阵的条件数}
\subsection{原理简介}

~~~~~矩阵的条件数是一种矩阵病态程度的度量, 条件数越大, 矩阵的病态程度越严重.其定义如下:\par
设矩阵$H \in \mathbb{R}^{n \times n}$,det$H\neq 0$, 对于矩阵的任意一种从属范数$\|\cdot\|$, 称$$cond(H)=\|H\| \|H^{-1}\| $$为$H$的条件数, 其中常见的有$cond(H)_p=\|H\|_p \|H^{-1}\|_p,p=1,2,\infty.$

对于$p=2$时,其定义如下:
$$\|H\|_2=\sqrt{\lambda_1}, \ \text{其中}\lambda_1\text{为}H^*H\text{的最大的特征值}, H^*\text{为}H\text{的共轭转置}.$$
\subsection{实验方法及结果分析}
~~~~我们这里采用了Matlab内置的函数cond来进行计算,其调用格式如下$cond(A,p)$, $p$表示的是范数的类型, 这里求的是2范数, 因此有$p=2$. 计算结果如Table \ref{tab:condNum}所示.
\par
从Table \ref{tab:condNum}中我们可以看出,当n很大时,$H_n$的条件数已经变得很大, 从而说明当n很大时,矩阵$H_n$已经变得相当病态.因此对此时$H_nx=b$进行求解时, $x$ 的误差会非常大, 为此, 我们需要引入正则化方法来解决病态矩阵的求解问题.
%%%%%%%%%%%%%%%%%%%%%%%%%%%% COND NUM%%%%%%%%%%%%%%%%%%%%%%%%%%%%%%%
\begin{center}
\begin{table*}[!htpb]
\resizebox{\textwidth}{!}{ %
\begin{threeparttable}[!htpb]
 \caption{\label{tab:condNum}n=2-15时的条件数}
 \begin{tabular}{ccccc ccccc ccccc}
 \toprule
   n   &2&3 &4 &5 &6 &7 &8 &9 &10 &11 &12 &13 &14 &15 \\ \hline
 条件数&19.28 &524.06&1.55e4&4.77e5&1.50e7&4.75e8&1.52e10&4.93e11&1.60e13 &5.23e14&1.68e16&1.76e18&3.08e17&4.43e17\\

 \bottomrule
\end{tabular}
%\small Note: Robust standard errors in parentheses. Intercept included but not reported. \begin{tablenotes}\item[*] significant at 5\% level \item[**] significant at 10\% level\end{tablenotes}
\end{threeparttable}}%
\end{table*}
\end{center}


\section{正则化方法的引入}
\subsection{Tikhonov正则化方法介绍}
~~~~从上面的分析我们可以看出, 当n变大时,Hilbert矩阵的条件数是非常大的, 因此那些线性方程组是病态方程组. 为了解决Hilbert矩阵条件数过大的问题, 我们就要引入一些正则化方法加以改善. 其中Tikhonov正则化方法是一种非常常用的正则化方法,现将该方法简单介绍如下:\par
我们要解$Hx=b$,可以转化为求方程
\begin{equation}\label{TikhonovEqu}
(H^*H+\alpha I)x=H^*b
\end{equation}

其中$H^*$是$H$的共轭转置, 即我们只需要在$H^*H$的对角线上添加$\alpha$即可.

\subsection{实验方法及结果分析}
~~~~~~在编写函数时,取$\alpha=10^{-4}$, 我们将$Hx=b$改写为$(H^*H+\alpha I)x=H^*b$, 然后调用Gauss.m和Cholesky.m两个函数进行求解,具体过程可以参看函数Gauss\_Cholesky\_hw\_1.m.\par

Table \ref{tab:Gauss_regu} 和Table \ref{tab:Chol_regu}是采用正则化方法之后的计算结果,从Table \ref{tab:Gauss_regu} 和Table \ref{tab:Chol_regu}可以看出, 引入正则化方法之后, 极大地改善了含有病态矩阵的线性代数方程组的稳定性, 计算结果与真值的差距已经非常小了, 基本都在1附近.\par
%%%%%%%%%%%%%%%%%%%%%%%%%%%%%%%%%%
%%%%%%%%%%%%%%%%%%%%%%% REGU RESULTS %%%%%%%%%%%%%%%%%%%%%%%%%%5555
\begin{center}
\begin{table*}[!htbp]
\resizebox{\textwidth}{!}{ %
\begin{threeparttable}[!htbp]
 \caption{\label{tab:Gauss_regu}n=2-15时Gauss消去法采用正则化方法之后的计算结果}
 \begin{tabular}{ccccc ccccc ccccc}
 \toprule
 &n=2&n=3 &n=4 &n=5 &n=6 &n=7 &n=8 &n=9 &n=10 &n=11 &n=12 &n=13 &n=14 &n=15 \\ \hline
$x_{1}$	&	1.000438967	&	0.992619358	&	0.995134696	&	0.998359323	&	1.001138087	&	1.003598269	&	1.005585626	&	1.00691644	&	1.007546531	&	1.007575603	&	1.007166775	&	1.006475747	&	1.005620933	&	1.004682094	\\
$x_{2}$	&	0.99916193	&	1.042289352	&	1.015987988	&	0.994796648	&	0.979290524	&	0.967852149	&	0.960817279	&	0.958382388	&	0.959923433	&	0.964275489	&	0.970255341	&	0.97696141	&	0.983817512	&	0.990497701	\\
$x_{3}$	&		&	0.95870273	&	1.014589791	&	1.027876821	&	1.029851394	&	1.026392831	&	1.019618896	&	1.011230634	&	1.002699462	&	0.995006491	&	0.988594933	&	0.983530217	&	0.979690176	&	0.97689327	\\
$x_{4}$	&		&		&	0.967952515	&	1.007969035	&	1.027758701	&	1.037385065	&	1.039734941	&	1.0367165	&	1.030291224	&	1.022230845	&	1.013806481	&	1.005755418	&	0.998422593	&	0.991922188	\\
$x_{5}$	&		&		&		&	0.964670288	&	0.997867221	&	1.019208575	&	1.03185877	&	1.037449063	&	1.037708282	&	1.034448448	&	1.029200914	&	1.023041803	&	1.016632629	&	1.010342294	\\
$x_{6}$	&		&		&		&		&	0.956378774	&	0.986882212	&	1.008586463	&	1.022730057	&	1.030572461	&	1.033650406	&	1.033463281	&	1.031212646	&	1.027744038	&	1.023608233	\\
$x_{7}$	&		&		&		&		&		&	0.948592762	&	0.977613818	&	0.99920894	&	1.014133406	&	1.023542683	&	1.028769494	&	1.031018489	&	1.031226391	&	1.030062172	\\
$x_{8}$	&		&		&		&		&		&		&	0.943298706	&	0.971039677	&	0.992101907	&	1.007247589	&	1.017590121	&	1.024271344	&	1.02825881	&	1.030291863	\\
$x_{9}$	&		&		&		&		&		&		&		&	0.94074089	&	0.966933662	&	0.987046046	&	1.001960815	&	1.012724318	&	1.020297258	&	1.025455445	\\
$x_{10}$	&		&		&		&		&		&		&		&		&	0.94021977	&	0.964528187	&	0.983411179	&	0.997806068	&	1.008644042	&	1.016714064	\\
$x_{11}$	&		&		&		&		&		&		&		&		&		&	0.94078177	&	0.963049662	&	0.980610451	&	0.994352896	&	1.005063413	\\
$x_{12}$	&		&		&		&		&		&		&		&		&		&		&	0.941666107	&	0.961952085	&	0.978240955	&	0.991306252	\\
$x_{13}$	&		&		&		&		&		&		&		&		&		&		&		&	0.942429097	&	0.960928865	&	0.976071193	\\
$x_{14}$	&		&		&		&		&		&		&		&		&		&		&		&		&	0.942882682	&	0.959842976	\\
$x_{15}$	&		&		&		&		&		&		&		&		&		&		&		&		&		&	0.942992049	\\

 \bottomrule
\end{tabular}
%\small Note: Robust standard errors in parentheses. Intercept included but not reported. \begin{tablenotes}\item[*] significant at 5\% level \item[**] significant at 10\% level\end{tablenotes}
\end{threeparttable}}%
\end{table*}
\end{center}

\begin{center}
\begin{table*}[!htbp]
\resizebox{\textwidth}{!}{ %
\begin{threeparttable}[!htbp]
 \caption{\label{tab:Chol_regu}n=2-15时Cholesky分解法采用正则化方法之后的计算结果}
 \begin{tabular}{ccccc ccccc ccccc}
 \toprule
 &n=2&n=3 &n=4 &n=5 &n=6 &n=7 &n=8 &n=9 &n=10 &n=11 &n=12 &n=13 &n=14 &n=15 \\ \hline
$x_{1}$	&	1.000438967	&	0.992619358	&	0.995134696	&	0.998359323	&	1.001138087	&	1.003598269	&	1.005585626	&	1.00691644	&	1.007546531	&	1.007575603	&	1.007166775	&	1.006475747	&	1.005620933	&	1.004682094	\\
$x_{2}$	&	0.99916193	&	1.042289352	&	1.015987988	&	0.994796648	&	0.979290524	&	0.967852149	&	0.960817279	&	0.958382388	&	0.959923433	&	0.964275489	&	0.970255341	&	0.97696141	&	0.983817512	&	0.990497701	\\
$x_{3}$	&		&	0.95870273	&	1.014589791	&	1.027876821	&	1.029851394	&	1.026392831	&	1.019618896	&	1.011230634	&	1.002699462	&	0.995006491	&	0.988594933	&	0.983530217	&	0.979690176	&	0.97689327	\\
$x_{4}$	&		&		&	0.967952515	&	1.007969035	&	1.027758701	&	1.037385065	&	1.039734941	&	1.0367165	&	1.030291224	&	1.022230845	&	1.013806481	&	1.005755418	&	0.998422593	&	0.991922188	\\
$x_{5}$	&		&		&		&	0.964670288	&	0.997867221	&	1.019208575	&	1.03185877	&	1.037449063	&	1.037708282	&	1.034448448	&	1.029200914	&	1.023041803	&	1.016632629	&	1.010342294	\\
$x_{6}$	&		&		&		&		&	0.956378774	&	0.986882212	&	1.008586463	&	1.022730057	&	1.030572461	&	1.033650406	&	1.033463281	&	1.031212646	&	1.027744038	&	1.023608233	\\
$x_{7}$	&		&		&		&		&		&	0.948592762	&	0.977613818	&	0.99920894	&	1.014133406	&	1.023542683	&	1.028769494	&	1.031018489	&	1.031226391	&	1.030062172	\\
$x_{8}$	&		&		&		&		&		&		&	0.943298706	&	0.971039677	&	0.992101907	&	1.007247589	&	1.017590121	&	1.024271344	&	1.02825881	&	1.030291863	\\
$x_{9}$	&		&		&		&		&		&		&		&	0.94074089	&	0.966933662	&	0.987046046	&	1.001960815	&	1.012724318	&	1.020297258	&	1.025455445	\\
$x_{10}$	&		&		&		&		&		&		&		&		&	0.94021977	&	0.964528187	&	0.983411179	&	0.997806068	&	1.008644042	&	1.016714064	\\
$x_{11}$	&		&		&		&		&		&		&		&		&		&	0.94078177	&	0.963049662	&	0.980610451	&	0.994352896	&	1.005063413	\\
$x_{12}$	&		&		&		&		&		&		&		&		&		&		&	0.941666107	&	0.961952085	&	0.978240955	&	0.991306252	\\
$x_{13}$	&		&		&		&		&		&		&		&		&		&		&		&	0.942429097	&	0.960928865	&	0.976071193	\\
$x_{14}$	&		&		&		&		&		&		&		&		&		&		&		&		&	0.942882682	&	0.959842976	\\
$x_{15}$	&		&		&		&		&		&		&		&		&		&		&		&		&		&	0.942992048	\\

 \bottomrule
\end{tabular}
%\small Note: Robust standard errors in parentheses. Intercept included but not reported. \begin{tablenotes}\item[*] significant at 5\% level \item[**] significant at 10\% level\end{tablenotes}
\end{threeparttable}}%
\end{table*}
\end{center}
%%%%%%%%%%%%%%%r dx%%%%%%%%%%%%%%%%%%%%%%%%%%%%%%%%%%%%%%%%%%%%%%%%%%%%%%%r dx regu%%%%%%%%%%%%%%%%%%%%%%%%%%%%%%%%%%%%%%%
%%%%%%%%%%%%%%%r dx%%%%%%%%%%%%%%%%%%%%%%%%%%%%%%%%%%%%%%%%%%%%%%%%%%%%%%%r dx regu %%%%%%%%%%%%%%%%%%%%%%%%%%%%%%%%%%%%%%%
\begin{center}
\begin{table*}[!htbp]
\resizebox{\textwidth}{!}{ %
\begin{threeparttable}[!htbp]
 \caption{\label{tab:Gauss_r_regu}n=2-15时Gauss消去法正则化后计算结果的残差向量$r=b-Hy$}
 \begin{tabular}{ccccc ccccc ccccc}
 \toprule

 &n=2&n=3 &n=4 &n=5 &n=6 &n=7 &n=8 &n=9 &n=10 &n=11 &n=12 &n=13 &n=14 &n=15 \\ \hline
$r_{1}$	&	-0.000193354	&	-0.000152626	&	-3.64793E-05	&	7.70014E-05	&	0.00014631	&	0.000182718	&	0.000199694	&	0.000204546	&	0.000200949	&	0.000190971	&	0.000176033	&	0.000157302	&	0.000135818	&	0.000112523	\\
$r_{2}$	&	0.000587568	&	0.00013927	&	-0.000371582	&	-0.00068654	&	-0.000787298	&	-0.000779593	&	-0.00072272	&	-0.000640672	&	-0.000543539	&	-0.000436759	&	-0.000324321	&	-0.000209697	&	-9.59872E-05	&	1.41608E-05	\\
$r_{3}$	&		&	0.000545975	&	0.000362803	&	8.39121E-05	&	-0.000129102	&	-0.000266985	&	-0.0003556	&	-0.000412883	&	-0.000448295	&	-0.000466766	&	-0.000471208	&	-0.000463757	&	-0.000446293	&	-0.000420624	\\
$r_{4}$	&		&		&	0.000797577	&	0.000610746	&	0.000382212	&	0.000192663	&	4.2481E-05	&	-7.95925E-05	&	-0.000181557	&	-0.000267697	&	-0.000340064	&	-0.000399663	&	-0.000447156	&	-0.00048322	\\
$r_{5}$	&		&		&		&	0.000918313	&	0.000708711	&	0.00051282	&	0.00034731	&	0.000205575	&	8.05956E-05	&	-3.17128E-05	&	-0.000133124	&	-0.000224124	&	-0.000304669	&	-0.000374652	\\
$r_{6}$	&		&		&		&		&	0.000909257	&	0.00072536	&	0.000565257	&	0.000425387	&	0.000299295	&	0.000182805	&	7.40483E-05	&	-2.73646E-05	&	-0.000121096	&	-0.000206596	\\
$r_{7}$	&		&		&		&		&		&	0.00086386	&	0.000717598	&	0.000589033	&	0.000472024	&	0.000362172	&	0.000257329	&	0.000156962	&	6.14409E-05	&	-2.84932E-05	\\
$r_{8}$	&		&		&		&		&		&		&	0.000822634	&	0.000708931	&	0.000605259	&	0.000506954	&	0.000411523	&	0.000318167	&	0.00022714	&	0.000139212	\\
$r_{9}$	&		&		&		&		&		&		&		&	0.000795754	&	0.000706693	&	0.000621761	&	0.000538112	&	0.000454636	&	0.000371389	&	0.000289071	\\
$r_{10}$	&		&		&		&		&		&		&		&		&	0.000783126	&	0.000711804	&	0.000640622	&	0.000568156	&	0.000494227	&	0.000419407	\\
$r_{11}$	&		&		&		&		&		&		&		&		&		&	0.000781798	&	0.000722877	&	0.000661583	&	0.000597495	&	0.000531026	\\
$r_{12}$	&		&		&		&		&		&		&		&		&		&		&	0.000788378	&	0.000737895	&	0.000683587	&	0.0006257	\\
$r_{13}$	&		&		&		&		&		&		&		&		&		&		&		&	0.000799822	&	0.000754911	&	0.000705469	\\
$r_{14}$	&		&		&		&		&		&		&		&		&		&		&		&		&	0.000813676	&	0.00077233	\\
$r_{15}$	&		&		&		&		&		&		&		&		&		&		&		&		&		&	0.000828107	\\
 \bottomrule
\end{tabular}
%\small Note: Robust standard errors in parentheses. Intercept included but not reported. \begin{tablenotes}\item[*] significant at 5\% level \item[**] significant at 10\% level\end{tablenotes}
\end{threeparttable}}%
\end{table*}
\end{center}

\begin{center}
\begin{table*}[!htbp]
\resizebox{\textwidth}{!}{ %
\begin{threeparttable}[!htbp]
 \caption{\label{tab:Chol_r_regu}n=2-15时Cholesky分解法正则化后计算结果的残差向量$r=b-Hy$}
 \begin{tabular}{ccccc ccccc ccccc}
 \toprule

 &n=2&n=3 &n=4 &n=5 &n=6 &n=7 &n=8 &n=9 &n=10 &n=11 &n=12 &n=13 &n=14 &n=15 \\ \hline
$r_{1}$	&	-0.000193354	&	-0.000152626	&	-3.64793E-05	&	7.70014E-05	&	0.00014631	&	0.000182718	&	0.000199694	&	0.000204546	&	0.000200949	&	0.000190971	&	0.000176033	&	0.000157302	&	0.000135818	&	0.000112523	\\
$r_{2}$	&	0.000587568	&	0.00013927	&	-0.000371582	&	-0.00068654	&	-0.000787298	&	-0.000779593	&	-0.00072272	&	-0.000640672	&	-0.000543539	&	-0.000436759	&	-0.000324321	&	-0.000209697	&	-9.59872E-05	&	1.41608E-05	\\
$r_{3}$	&		&	0.000545975	&	0.000362803	&	8.39121E-05	&	-0.000129102	&	-0.000266985	&	-0.0003556	&	-0.000412883	&	-0.000448295	&	-0.000466766	&	-0.000471208	&	-0.000463757	&	-0.000446293	&	-0.000420624	\\
$r_{4}$	&		&		&	0.000797577	&	0.000610746	&	0.000382212	&	0.000192663	&	4.2481E-05	&	-7.95925E-05	&	-0.000181557	&	-0.000267697	&	-0.000340064	&	-0.000399663	&	-0.000447156	&	-0.00048322	\\
$r_{5}$	&		&		&		&	0.000918313	&	0.000708711	&	0.00051282	&	0.00034731	&	0.000205575	&	8.05956E-05	&	-3.17128E-05	&	-0.000133124	&	-0.000224124	&	-0.000304669	&	-0.000374652	\\
$r_{6}$	&		&		&		&		&	0.000909257	&	0.00072536	&	0.000565257	&	0.000425387	&	0.000299295	&	0.000182805	&	7.40483E-05	&	-2.73646E-05	&	-0.000121096	&	-0.000206596	\\
$r_{7}$	&		&		&		&		&		&	0.00086386	&	0.000717598	&	0.000589033	&	0.000472024	&	0.000362172	&	0.000257329	&	0.000156962	&	6.14409E-05	&	-2.84932E-05	\\
$r_{8}$	&		&		&		&		&		&		&	0.000822634	&	0.000708931	&	0.000605259	&	0.000506954	&	0.000411523	&	0.000318167	&	0.00022714	&	0.000139212	\\
$r_{9}$	&		&		&		&		&		&		&		&	0.000795754	&	0.000706693	&	0.000621761	&	0.000538112	&	0.000454636	&	0.000371389	&	0.000289071	\\
$r_{10}$	&		&		&		&		&		&		&		&		&	0.000783126	&	0.000711804	&	0.000640622	&	0.000568156	&	0.000494227	&	0.000419407	\\
$r_{11}$	&		&		&		&		&		&		&		&		&		&	0.000781798	&	0.000722877	&	0.000661583	&	0.000597495	&	0.000531026	\\
$r_{12}$	&		&		&		&		&		&		&		&		&		&		&	0.000788378	&	0.000737895	&	0.000683587	&	0.0006257	\\
$r_{13}$	&		&		&		&		&		&		&		&		&		&		&		&	0.000799822	&	0.000754911	&	0.000705469	\\
$r_{14}$	&		&		&		&		&		&		&		&		&		&		&		&		&	0.000813676	&	0.00077233	\\
$r_{15}$	&		&		&		&		&		&		&		&		&		&		&		&		&		&	0.000828107	\\

 \bottomrule
\end{tabular}
%\small Note: Robust standard errors in parentheses. Intercept included but not reported. \begin{tablenotes}\item[*] significant at 5\% level \item[**] significant at 10\% level\end{tablenotes}
\end{threeparttable}}%
\end{table*}
\end{center}

%%%%%%%%%%%%%%%%%%%%%%%% Delta x%%%%%%%%%%%%%%%%%%%%%%5
\begin{center}
\begin{table*}[!htbp]
\resizebox{\textwidth}{!}{ %
\begin{threeparttable}[!htbp]
 \caption{\label{tab:Gauss_dx_regu}n=2-15时Gauss消去法正则化后计算结果的误差$\Delta x=y-x$}
 \begin{tabular}{ccccc ccccc ccccc}
 \toprule

 &n=2&n=3 &n=4 &n=5 &n=6 &n=7 &n=8 &n=9 &n=10 &n=11 &n=12 &n=13 &n=14 &n=15 \\ \hline
$\Delta x_{1}$	&	0.004298821	&	-0.009991912	&	-0.019417952	&	-0.019488126	&	-0.015355294	&	-0.010422874	&	-0.005647367	&	-0.001217068	&	0.002834614	&	0.006488677	&	0.009722412	&	0.012515934	&	0.014859627	&	0.016757701	\\
$\Delta x_{2}$	&	-0.008210934	&	0.066041208	&	0.08115782	&	0.058334086	&	0.029012539	&	0.003561422	&	-0.017241849	&	-0.034167283	&	-0.047847811	&	-0.058672041	&	-0.06689584	&	-0.072732553	&	-0.076399644	&	-0.07813359	\\
$\Delta x_{3}$	&		&	-0.068628197	&	0.004347947	&	0.03793107	&	0.047284045	&	0.04635532	&	0.041139992	&	0.033958205	&	0.025835772	&	0.01733812	&	0.008843097	&	0.00062883	&	-0.00709854	&	-0.014194716	\\
$\Delta x_{4}$	&		&		&	-0.090295179	&	-0.02033764	&	0.018198443	&	0.038706139	&	0.049461902	&	0.054445166	&	0.055638232	&	0.054159257	&	0.050738743	&	0.045916888	&	0.040124269	&	0.033713715	\\
$\Delta x_{5}$	&		&		&		&	-0.086575995	&	-0.029497785	&	0.006022157	&	0.029001798	&	0.04416527	&	0.053958481	&	0.059765514	&	0.062474449	&	0.062731848	&	0.061055444	&	0.057881672	\\
$\Delta x_{6}$	&		&		&		&		&	-0.082252121	&	-0.036714928	&	-0.004960087	&	0.018003781	&	0.034815755	&	0.04696769	&	0.055400269	&	0.06078449	&	0.063652814	&	0.06445756	\\
$\Delta x_{7}$	&		&		&		&		&		&	-0.082278974	&	-0.044375911	&	-0.015622288	&	0.006670544	&	0.024018468	&	0.037361094	&	0.047353433	&	0.054507111	&	0.059256218	\\
$\Delta x_{8}$	&		&		&		&		&		&		&	-0.085155772	&	-0.052176377	&	-0.025691405	&	-0.004204446	&	0.013196704	&	0.027132485	&	0.038077667	&	0.046432213	\\
$\Delta x_{9}$	&		&		&		&		&		&		&		&	-0.089229706	&	-0.059580679	&	-0.03484679	&	-0.014155129	&	0.003072943	&	0.017269504	&	0.028794211	\\
$\Delta x_{10}$	&		&		&		&		&		&		&		&		&	-0.093488959	&	-0.066235936	&	-0.042908224	&	-0.022970389	&	-0.006033655	&	0.008220257	\\
$\Delta x_{11}$	&		&		&		&		&		&		&		&		&		&	-0.097394675	&	-0.071974731	&	-0.049827927	&	-0.030606819	&	-0.014032963	\\
$\Delta x_{12}$	&		&		&		&		&		&		&		&		&		&		&	-0.100695522	&	-0.076762076	&	-0.055651422	&	-0.03712218	\\
$\Delta x_{13}$	&		&		&		&		&		&		&		&		&		&		&		&	-0.103310933	&	-0.0806477	&	-0.060480945	\\
$\Delta x_{14}$	&		&		&		&		&		&		&		&		&		&		&		&		&	-0.105260158	&	-0.08372986	\\
$\Delta x_{15}$	&		&		&		&		&		&		&		&		&		&		&		&		&		&	-0.106617096	\\
\bottomrule
\end{tabular}
%\small Note: Robust standard errors in parentheses. Intercept included but not reported. \begin{tablenotes}\item[*] significant at 5\% level \item[**] significant at 10\% level\end{tablenotes}
\end{threeparttable}}%
\end{table*}
\end{center}

\begin{center}
\begin{table*}[t]
\resizebox{\textwidth}{!}{ %
\begin{threeparttable}[t]
 \caption{\label{tab:Chol_dx_regu}n=2-15时Cholesky分解法正则化后计算结果的误差向量$\Delta x=y-x$}
 \begin{tabular}{ccccc ccccc ccccc}
 \toprule

 &n=2&n=3 &n=4 &n=5 &n=6 &n=7 &n=8 &n=9 &n=10 &n=11 &n=12 &n=13 &n=14 &n=15 \\ \hline
$\Delta x_{1}$	&	0.004298821	&	-0.009991912	&	-0.019417952	&	-0.019488126	&	-0.015355294	&	-0.010422874	&	-0.005647367	&	-0.001217068	&	0.002834614	&	0.006488677	&	0.009722412	&	0.012515934	&	0.014859627	&	0.016757701	\\
$\Delta x_{2}$	&	-0.008210934	&	0.066041208	&	0.08115782	&	0.058334086	&	0.029012539	&	0.003561422	&	-0.017241849	&	-0.034167283	&	-0.047847811	&	-0.058672041	&	-0.06689584	&	-0.072732553	&	-0.076399644	&	-0.07813359	\\
$\Delta x_{3}$	&		&	-0.068628197	&	0.004347947	&	0.03793107	&	0.047284045	&	0.04635532	&	0.041139992	&	0.033958205	&	0.025835772	&	0.01733812	&	0.008843097	&	0.00062883	&	-0.00709854	&	-0.014194716	\\
$\Delta x_{4}$	&		&		&	-0.090295179	&	-0.02033764	&	0.018198443	&	0.038706139	&	0.049461902	&	0.054445166	&	0.055638232	&	0.054159257	&	0.050738743	&	0.045916888	&	0.040124269	&	0.033713715	\\
$\Delta x_{5}$	&		&		&		&	-0.086575995	&	-0.029497785	&	0.006022157	&	0.029001798	&	0.04416527	&	0.053958481	&	0.059765514	&	0.062474449	&	0.062731848	&	0.061055444	&	0.057881672	\\
$\Delta x_{6}$	&		&		&		&		&	-0.082252121	&	-0.036714928	&	-0.004960087	&	0.018003781	&	0.034815755	&	0.04696769	&	0.055400269	&	0.06078449	&	0.063652814	&	0.06445756	\\
$\Delta x_{7}$	&		&		&		&		&		&	-0.082278974	&	-0.044375911	&	-0.015622288	&	0.006670544	&	0.024018468	&	0.037361094	&	0.047353433	&	0.054507111	&	0.059256218	\\
$\Delta x_{8}$	&		&		&		&		&		&		&	-0.085155772	&	-0.052176377	&	-0.025691405	&	-0.004204446	&	0.013196704	&	0.027132485	&	0.038077667	&	0.046432213	\\
$\Delta x_{9}$	&		&		&		&		&		&		&		&	-0.089229706	&	-0.059580679	&	-0.03484679	&	-0.014155129	&	0.003072943	&	0.017269504	&	0.028794211	\\
$\Delta x_{10}$	&		&		&		&		&		&		&		&		&	-0.093488959	&	-0.066235936	&	-0.042908224	&	-0.022970389	&	-0.006033655	&	0.008220257	\\
$\Delta x_{11}$	&		&		&		&		&		&		&		&		&		&	-0.097394675	&	-0.071974731	&	-0.049827927	&	-0.030606819	&	-0.014032963	\\
$\Delta x_{12}$	&		&		&		&		&		&		&		&		&		&		&	-0.100695522	&	-0.076762076	&	-0.055651422	&	-0.03712218	\\
$\Delta x_{13}$	&		&		&		&		&		&		&		&		&		&		&		&	-0.103310933	&	-0.0806477	&	-0.060480945	\\
$\Delta x_{14}$	&		&		&		&		&		&		&		&		&		&		&		&		&	-0.105260158	&	-0.08372986	\\
$\Delta x_{15}$	&		&		&		&		&		&		&		&		&		&		&		&		&		&	-0.106617096	\\
 \bottomrule
\end{tabular}
%\small Note: Robust standard errors in parentheses. Intercept included but not reported. \begin{tablenotes}\item[*] significant at 5\% level \item[**] significant at 10\% level\end{tablenotes}
\end{threeparttable}}%
\end{table*}
\end{center}

我们进一步分析, 如Table \ref{tab:Gauss_r_regu}, Table \ref{tab:Chol_r_regu},Table \ref{tab:Gauss_dx_regu}和Table \ref{tab:Chol_dx_regu} 所示, 采用正则化方法之后, 虽然残差向量相比于从前变大了一些, 但是误差向量变得很小了, 因此可以说采用正则化方法之后, 结果更加准确和稳定.

\section{实验小结}
~~~~通过这个实验, 我了解到了含有病态矩阵的线性代数方程组数值求解方法的不稳定性, 此时, 我们可以采用某些正则化方法来使得这种线性代数方程组的求解变得稳定.
\section{Matlab源程序}
\begin{enumerate}[(1)]
\item \textbf{lu\_pivot\_in\_col.m} \\
\begin{lstlisting}
function [LU, p] = lu_pivot_in_col(A)

[m, n] = size(A);
% Check if A a square matrix.
if m ~= n
    disp('A must be a square matrix, please check again.')
    return
end

% LU composition and permutation
p = 1:n;
LU = A;
for j = 1:n
    [Max, index] = max(abs(LU(j:n, j)));
    if Max == 0
        disp('The matrix is singular.')
        return
    end
    index = index + j - 1;
    if index ~= j
        tmp = p(j);
        p(j) = p(index);
        p(index) = tmp;

        % Exchange the two rows of the matrix
        tmp = LU(j,:);
        LU(j, :) = LU(index, :);
        LU(index,:) = tmp;
    end
    if j == n
        break;
    end

    LU(j+1:n, j) = LU(j+1:n, j) / LU(j, j);
    for k = j+1:n
        LU(k, j+1:n) = LU(k, j+1:n) - LU(k, j) * LU(j,j+1:n);
    end
    %LU(j+1:n, j+1:n) = LU(j+1:n, j+1:n) - LU(j+1:n, j) * LU(j,j+1:n);
end
\end{lstlisting}



\item \textbf{Gauss.m} \\
\begin{lstlisting}
function x = Gauss(A, b)

n = length(b);
%[L, U, P] = lu(A);
[LU, p] = lu_pivot_in_col(A);
b = b(p);
% Ly = P*b
y = zeros(n, 1);
y(1) = b(1);
for i = 2:n
    y(i) = b(i) - LU(i, 1:i-1) * y(1:i-1);
end
% Ux = y
x = zeros(n, 1);
x(n) = y(n) / LU(n, n);
for i = n-1:-1:1
    x(i) = (y(i) - LU(i, i+1:n)* x(i+1:n)) / LU(i, i);
end
end
\end{lstlisting}



\item \textbf{cholesky\_llt.m} \\
\begin{lstlisting}
function L = cholesky_llt(A)

n = length(A);
L = zeros(n);
L(1,1) = A(1,1)^0.5;
L(2:n, 1) = A(1,2:n)/L(1,1);
for j = 2:n-1
    L(j,j) = (A(j,j) - sum(L(j,1:j-1).^2))^0.5;
    for i = j+1:n
        L(i,j) = (A(i,j)- L(i, 1:j-1)*L(j, 1:j-1)') / L(j,j);
    end
end
L(n,n) = (A(n,n) - sum(L(n,1:n-1).^2))^0.5;
\end{lstlisting}




\item \textbf{Cholesky.m} \\
\begin{lstlisting}
function x = Cholesky(A, b)
n = length(b);
L = cholesky_llt(A);
%L = chol(A,'lower');
y = zeros(n, 1);
x = zeros(n, 1);
% solve Ly = b
y(1) = b(1) / L(1,1);
for i = 2:n
    y(i) = (b(i) - L(i, 1:i-1) * y(1:i-1)) / L(i, i);
end
% solve Ux = y
U = L';
x(n) = y(n) / U(n, n);
for i = n-1:-1:1
    x(i) = (y(i) - U(i, i+1:n)* x(i+1:n)) / U(i, i);
end
\end{lstlisting}



\item \textbf{Gauss\_Cholesky\_hw\_1.m} \\
\begin{lstlisting}
clear;
clc;
% Excels possible range cells
range = {'A','B','C','D','E','F','G','H','I', 'J','K','L','M','N','O','P',...
          'Q','R','S','T','U','V','W','X','Y','Z','AA','AB','AC','AD'};
file_gauss = 'Results_Gauss.xlsx';
file_chol = 'Results_Chol.xlsx';
file_r_gauss = 'Results_r_Gauss.xlsx';
file_r_chol = 'Results_r_Chol.xlsx';
file_dx_gauss = 'Results_delta_x_Gauss.xlsx';
file_dx_chol ='Results_delta_x_Chol.xlsx';
for n = 2:15
    H = hilb(n);
    x = ones(n, 1);
    b = H * x;

    % solve H*y_gauss = b
    %y_gauss = H \ b;
    y_gauss = Gauss(H, b);
    % solve H*y_Chol = b
    y_chol = Cholesky(H, b);
    % 残差向量
    r_gauss = b - H*y_gauss;
    r_chol = b - H*y_chol;
    % 误差向量
    delta_x_gauss = y_gauss - x;
    delta_x_chol = y_chol - x;
    % Write results to Excels
    xlswrite(file_gauss,y_gauss, [range{n-1},'2:', range{n-1}, num2str(n+1)])
    xlswrite(file_gauss,{['n=' num2str(n)]}, [range{n-1} '1:' range{n-1} '1']);
    xlswrite(file_chol,y_chol, [range{n-1},'2:', range{n-1}, num2str(n+1)])
    xlswrite(file_chol,{['n=' num2str(n)]}, [range{n-1} '1:' range{n-1} '1']);

    xlswrite(file_r_gauss, r_gauss, [range{n-1},'2:', range{n-1}, num2str(n+1)])
    xlswrite(file_r_gauss,{['n=' num2str(n)]}, [range{n-1} '1:' range{n-1} '1']);
    xlswrite(file_r_chol, r_chol, [range{n-1},'2:', range{n-1}, num2str(n+1)])
    xlswrite(file_r_chol,{['n=' num2str(n)]}, [range{n-1} '1:' range{n-1} '1']);

    xlswrite(file_dx_gauss, delta_x_gauss, [range{n-1},'2:', range{n-1}, num2str(n+1)])
    xlswrite(file_dx_gauss,{['n=' num2str(n)]}, [range{n-1} '1:' range{n-1} '1']);
    xlswrite(file_dx_chol, delta_x_chol, [range{n-1},'2:', range{n-1}, num2str(n+1)])
    xlswrite(file_dx_chol,{['n=' num2str(n)]}, [range{n-1} '1:' range{n-1} '1']);
    % 计算条件数cond(H)2
    condH(n-1) = cond(H, 2);
end

file_gauss_regu = 'Results_Gauss_regu.xlsx';
file_chol_regu = 'Results_Chol_regu.xlsx';
file_r_gauss_regu = 'Results_r_Gauss_regu.xlsx';
file_r_chol_regu = 'Results_r_Chol_regu.xlsx';
file_dx_gauss_regu = 'Results_delta_x_Gauss_regu.xlsx';
file_dx_chol_regu ='Results_delta_x_Chol_regu.xlsx';


% Tikhonov Regularization
alpha = 1e-4;
for n = 2:15
    H = hilb(n);
    x = ones(n, 1);
    b = H'* H * x;
    A = H' * H + alpha * diag(ones(n, 1));
    % Solve A*y_gauss = b
    y_gauss = Gauss(A, b);
    % Solve A*y_chol = b
    y_chol = Cholesky(A, b);
    r_gauss = H*x - H*y_gauss;
    r_chol = H*x - H*y_chol;
    delta_x_gauss = y_gauss - x;
    delta_x_chol = y_chol - x;

    % Write results to Excels
    xlswrite(file_gauss_regu, y_gauss, [range{n-1},'2:', range{n-1}, num2str(n+1)])
    xlswrite(file_gauss_regu,{['n=' num2str(n)]}, [range{n-1} '1:' range{n-1} '1']);
    xlswrite(file_chol_regu, y_chol, [range{n-1},'2:', range{n-1}, num2str(n+1)])
    xlswrite(file_gauss_regu,{['n=' num2str(n)]}, [range{n-1} '1:' range{n-1} '1']);

    xlswrite(file_r_gauss_regu, r_gauss, [range{n-1},'2:', range{n-1}, num2str(n+1)])
    xlswrite(file_r_gauss_regu,{['n=' num2str(n)]}, [range{n-1} '1:' range{n-1} '1']);
    xlswrite(file_r_chol_regu, r_chol, [range{n-1},'2:', range{n-1}, num2str(n+1)])
    xlswrite(file_r_chol_regu,{['n=' num2str(n)]}, [range{n-1} '1:' range{n-1} '1']);

    xlswrite(file_dx_gauss_regu, delta_x_gauss, [range{n-1},'2:', range{n-1}, num2str(n+1)])
    xlswrite(file_dx_gauss_regu,{['n=' num2str(n)]}, [range{n-1} '1:' range{n-1} '1']);
    xlswrite(file_dx_chol_regu, delta_x_chol, [range{n-1},'2:', range{n-1}, num2str(n+1)])
    xlswrite(file_dx_chol_regu,{['n=' num2str(n)]}, [range{n-1} '1:' range{n-1} '1']);

end
\end{lstlisting}
\end{enumerate}


%\newpage
%%%%%%%%%%%%%%%%%%%%Figure%%%%%%%%%%%%%%%%%%%%%%%%%%%%%%%%%%%%%%%%%%%%%%%%%%%%%%%%%%%
%%%%%%%%%%%%%%%%%%%%%%%%%%%%%%%%%%%%%%%%%%%%%%% Gauss results %%%%%%%%%%%%%%%%%%%%%%%%%%%%%%%%%%%%%%%
%\textbf{附录:}\par
%这里是本实验报告用到的所有的表格.


\end{CJK*}
\end{document}
