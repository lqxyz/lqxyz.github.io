\documentclass[a4paper]{article}
\usepackage{geometry}
\geometry{left=2.5cm,right=2.5cm,top=2.5cm,bottom=2.5cm}
\usepackage{listings}
\lstset{language=Matlab}%代码语言使用的是matlab
\lstset{breaklines}%自动将长的代码行换行排版
\lstset{extendedchars=false}%解决代码跨页时,章节标题,页眉等汉字不显示的问题
%\documentclass{article}
\usepackage{CJK}
\usepackage{amsmath}
\usepackage{amsfonts,amssymb}
\usepackage{graphicx}
\usepackage{enumerate}
\usepackage{booktabs}
\usepackage{threeparttable}
\usepackage{rotating}

\begin{document}
\begin{CJK*}{GBK}{song}
\title{\bf{线性方程组迭代解法部分上机实习报告}}
\author{\bf{姓名:刘群\ 学号:2014211591 \ 院系: 地学中心}\\ \bf{Email}:\ liu-q14@mails.tsinghua.edu.cn \\ (本文档由\LaTeX{}编写)}
\maketitle

\section{问题的描述}
\ \ \ \ 设$H_n=[h_{ij}] \in \mathbb{R}^{n \times n}$是Hilbert矩阵,即$h_{ij}=\frac{1}{i+j-1}\cdot$,取$x=\left(
\begin{array}{ccc}
1\\
\rotatebox{90}{$\cdots$} \\
1
\end{array}
\right) \in \mathbb{R}^n$, 并令$b_n=H_nx$, 用SOR迭代方法和共轭梯度法求解$H_nx=b_n$, 并与前面的直接方法做比较.


\section{SOR方法介绍}
\subsection{原理介绍}
\ \ \ \ 迭代法是一种求解求解线性代数方程组常用的方法, 它是从某些初始向量出发, 然后用设计好的步骤逐次进行迭代求解, 直至逼近真解.具体来说, 我们要求解线性代数方程组$Ax=b$, 我们可以将矩阵$A$分解成
\begin{equation}
A=M-N
\end{equation}
其中$M$是非奇异的,则原方程可以转化为
\begin{equation}
x=M^{-1}Nx+M^{-1}b=Bx+f
\end{equation}
其中
\begin{equation}
B=M^{-1}N=I-M^{-1}A \\
f=M^{-1}b
\end{equation}
B称为迭代矩阵. 我们可以通过选择不同的$M$和$N$来构造不同的迭代方法.\par
SOR方法是一种求解线性代数方程组的迭代方法,现简单介绍如下:\par
我们可以把$A$分解为
\begin{equation}
A=D-L-U
\end{equation}

其中,$D=diag(a_{11},a_{22},\cdots,a_{nn})$, $-L$和$-U$为$A$的严格下三角部分和严格上三角部分(不包括对角线),即
$$-L= \left(                 %左括号
            \begin{array}{cccc}   %该矩阵一共4列,每一列都居中放置
            0      &   &    &  \\  %第一行元素
            a_{21} & 0 &    &  \\  %第二行元素
            \rotatebox{-90}{$\cdots$}&\rotatebox{-45}{$\cdots$}&\rotatebox{-45}{$\cdots$}& \\
            a_{n1} & \cdots & a_{n, n-1} & 0 \\
            \end{array}
            \right)                 %右括号
$$

$$-U= \left(                 %左括号
            \begin{array}{cccc}   %该矩阵一共4列,每一列都居中放置
            0      & a_{12}  & \cdots  & a_{1n}  \\  %第一行元素
                   & \rotatebox{-45}{$\cdots$} & \rotatebox{-45}{$\cdots$} & \rotatebox{-90}{$\cdots$} \\  %第二行元素
                   &         & 0 & a_{n-1,n}\\
                   & & & 0 \\
            \end{array}
            \right)                 %右括号
$$
对于$A=D-L-U$ 和任意实数$\omega$来说, 我们有
\begin{equation}
\omega A=(D-\omega L)-((1-\omega)D+\omega U)
\end{equation}
如果取
\begin{equation}
M=\frac{1}{\omega}(D-\omega L),\\ N=\frac{1}{\omega}[(1-\omega)D+\omega U]
\end{equation}
就得到了超松弛迭代法, 也称为SOR迭代法.
\begin{equation}\label{sor_matrix}
(D-\omega L)x^{(k+1)}=[(1-\omega)D+\omega U]x^{(k)}+\omega b
\end{equation}
也可以写成
\begin{equation}
x^{(k+1)}=(D-\omega L)^{-1}[(1-\omega)D+\omega U]x^{(k)}+\omega(D-\omega L)^{-1} b
\end{equation}
即$$B=(D-\omega L)^{-1}[(1-\omega)D+\omega U]$$
$$f=\omega(D-\omega L)^{-1} b$$
按照公式(\ref{sor_matrix}), 我们可以写出分量形式为
\begin{equation}
x_i^{(k+1)}=(1-\omega)x_i^{(k)}+\frac{\omega}{a_{ii}}\left  (b_i-\displaystyle{\sum_{j=1}^{i-1}}a_{ij}x_j^{(k+1)}-\displaystyle{\sum_{j=i+1}^{n}}a_{ij}x_j^{(k)}\right),i=1,2,\cdots,n
\end{equation}
其中$\omega$ 叫做松弛因子.

\subsection{方案设计}
在这里我主要编写了函数sor.m来对线性方程组用SOR方法进行求解, 该函数有三个参数, 分别是$A,b$和$\omega$, 其中$A$和$b$分别是方程组$Ax=b$的系数矩阵$A$和常数向量$b$, $\omega$是松弛因子. 该函数的返回值是线性方程组的解$x$. 在求解过程中, 当前后两次$x$的结果小于$10^{-8}$时, 我们就停止迭代过程, 直接输出$x$. 在函数solve\_hilbert\_equ.m中, 我们在$n=2-15$的循环中对sor.m进行了调用, 并将每次的结果写在了Excel表格中. 在函数solve\_hilbert\_equ.m中, 我设定了松弛因子$\omega=1.05$. 同时, 我们还分析了结果的误差和残差向量的大小.

\subsection{实验结果与分析}
Table \ref{tab:SOR_x} 是采用SOR方法时计算出的结果, 从中可以看出, 即使是当n很大时, 计算出的结果仍然比较准确, 这一点我们也可以通过Table \ref{tab:SOR_delta_x} 看出. Table \ref{tab:SOR_delta_x} 表示的是我们计算出的$x$与理论真解$x$值之间的差别, 可以看出即使当n很大时, 误差的结果仍然很小. 此时, 通过 Table \ref{tab:SOR_r} 我们可以看出, 残差也非常小, 达到了$10^{-9}, 10^{-10}$左右的数量级, 从一个侧面反映出了结果的准确性. 从Table \ref{tab:SOR_norm} 中可以看出, 误差向量的二范数在$n=15$时达到了0.011820538, 稍后我们将看到, 这个误差向量的范数比采用直接法进行求解要小得多.
%---------------------表格 SOR方法 x -------------------------%
\begin{center}
\begin{table*}[!htbp]
\resizebox{\textwidth}{!}{ %
\begin{threeparttable}[!htpb]
 \caption{\label{tab:SOR_x}n=2-15时SOR方法计算所得的结果}
 \begin{tabular}{ccccc ccccc ccccc}
 \toprule
	&n=2	&	n=3	&	n=4	&	n=5	&	n=6	&	n=7	&	n=8	&	n=9	&	n=10	&	n=11	&	n=12	&	n=13	&	n=14	&	n=15	\\ \hline
$x_{1}$	&1.000000015	&	0.99999991	&	1.000000371	&	0.999998168	&	1.000010152	&	1.000004249	&	1.000000245	&	0.999994427	&	0.999990389	&	0.999993995	&	0.999996798	&	0.99999955	&	1.000002637	&	1.000006301	\\
$x_{2}$	&0.999999978	&	1.000000459	&	0.999996027	&	1.000033561	&	0.999716548	&	0.999905992	&	1.000036868	&	1.000219712	&	1.000276664	&	1.000148606	&	1.000051015	&	0.999957358	&	0.999855409	&	0.999738266	\\
$x_{3}$	&	&	0.999999578	&	1.000009301	&	0.999857195	&	1.001888919	&	1.00042827	&	0.999438089	&	0.998162804	&	0.99819952	&	0.999240514	&	1.000005974	&	1.000711692	&	1.001444919	&	1.002247953	\\
$x_{4}$	&	&		&	0.999994062	&	1.000213616	&	0.995140899	&	0.999654016	&	1.002360583	&	1.005263689	&	1.003829156	&	1.000823955	&	0.998772502	&	0.997028077	&	0.99536663	&	0.993697275	\\
$x_{5}$	&	&		&		&	0.999896245	&	1.005318055	&	0.998822217	&	0.996567037	&	0.995518287	&	0.998701526	&	1.001343632	&	1.002735659	&	1.003585371	&	1.004102615	&	1.004368357	\\
$x_{6}$	&	&		&		&		&	0.997918589	&	1.002236699	&	1.000270388	&	0.997435534	&	0.9970438	&	0.998513673	&	0.999879959	&	1.00121289	&	1.002532365	&	1.003828668	\\
$x_{7}$	&	&		&		&		&		&	0.998944875	&	1.003064003	&	1.00273845	&	0.999585205	&	0.998162775	&	0.997825882	&	0.998032997	&	0.998611761	&	0.999469359	\\
$x_{8}$	&	&		&		&		&		&		&	0.998258342	&	1.003713782	&	1.002498776	&	1.00010113	&	0.998526843	&	0.997490398	&	0.996890078	&	0.996672519	\\
$x_{9}$	&	&		&		&		&		&		&		&	0.996944651	&	1.002399345	&	1.002007286	&	1.000660962	&	0.999199063	&	0.997852997	&	0.996726944	\\
$x_{10}$&		&		&		&		&		&		&		&		&	0.997468601	&	1.001724576	&	1.00220416	&	1.001448066	&	1.000191091	&	0.998767723	\\
$x_{11}$&		&		&		&		&		&		&		&		&		&	0.997934196	&	1.00155475	&	1.002573108	&	1.002301974	&	1.00133333	\\
$x_{12}$&		&		&		&		&		&		&		&		&		&		&	0.997779854	&	1.001412978	&	1.002905659	&	1.003101475	\\
$x_{13}$&		&		&		&		&		&		&		&		&		&		&		&	0.997341735	&	1.00118708	&	1.003127874	\\
$x_{14}$&		&		&		&		&		&		&		&		&		&		&		&		&	0.996745871	&	1.000856185	\\
$x_{15}$&		&		&		&		&		&		&		&		&		&		&		&		&		&	0.996045381	\\
\bottomrule
\end{tabular}
%\small Note: Robust standard errors in parentheses. Intercept included but not reported. \begin{tablenotes}\item[*] significant at 5\% level \item[**] significant at 10\% level\end{tablenotes}
\end{threeparttable}}%
\end{table*}
\end{center}

%---------------- SOR Delta X ----------------------%
\begin{center}
\begin{table*}[!htbp]
\resizebox{\textwidth}{!}{ %
\begin{threeparttable}[!htpb]
 \caption{\label{tab:SOR_delta_x}n=2-15时SOR方法计算所得的x的误差$\Delta x$}
 \begin{tabular}{ccccc ccccc ccccc}
 \toprule
	&	n=2	&	n=3	&	n=4	&	n=5	&	n=6	&	n=7	&	n=8	&	n=9	&	n=10	&	n=11	&	n=12	&	n=13	&	n=14	&	n=15	\\ \hline
$\Delta x_{1}$	&	1.48392E-08	&	-9.046E-08	&	3.70947E-07	&	-1.83188E-06	&	1.01525E-05	&	4.24925E-06	&	2.44577E-07	&	-5.57275E-06	&	-9.61072E-06	&	-6.00452E-06	&	-3.20166E-06	&	-4.49838E-07	&	2.6366E-06	&	6.30111E-06	\\
$\Delta x_{2}$	&	-2.18608E-08	&	4.58795E-07	&	-3.97264E-06	&	3.35615E-05	&	-0.000283452	&	-9.40081E-05	&	3.68676E-05	&	0.000219712	&	0.000276664	&	0.000148606	&	5.10152E-05	&	-4.26418E-05	&	-0.000144591	&	-0.000261734	\\
$\Delta x_{3}$	&		&	-4.22292E-07	&	9.30101E-06	&	-0.000142805	&	0.001888919	&	0.00042827	&	-0.000561911	&	-0.001837196	&	-0.00180048	&	-0.000759486	&	5.97418E-06	&	0.000711692	&	0.001444919	&	0.002247953	\\
$\Delta x_{4}$	&		&		&	-5.93833E-06	&	0.000213616	&	-0.004859101	&	-0.000345984	&	0.002360583	&	0.005263689	&	0.003829156	&	0.000823955	&	-0.001227498	&	-0.002971923	&	-0.00463337	&	-0.006302725	\\
$\Delta x_{5}$	&		&		&		&	-0.000103755	&	0.005318055	&	-0.001177783	&	-0.003432963	&	-0.004481713	&	-0.001298474	&	0.001343632	&	0.002735659	&	0.003585371	&	0.004102615	&	0.004368357	\\
$\Delta x_{6}$	&		&		&		&		&	-0.002081411	&	0.002236699	&	0.000270388	&	-0.002564466	&	-0.0029562	&	-0.001486327	&	-0.000120041	&	0.00121289	&	0.002532365	&	0.003828668	\\
$\Delta x_{7}$	&		&		&		&		&		&	-0.001055125	&	0.003064003	&	0.00273845	&	-0.000414795	&	-0.001837225	&	-0.002174118	&	-0.001967003	&	-0.001388239	&	-0.000530641	\\
$\Delta x_{8}$	&		&		&		&		&		&		&	-0.001741658	&	0.003713782	&	0.002498776	&	0.00010113	&	-0.001473157	&	-0.002509602	&	-0.003109922	&	-0.003327481	\\
$\Delta x_{9}$	&		&		&		&		&		&		&		&	-0.003055349	&	0.002399345	&	0.002007286	&	0.000660962	&	-0.000800937	&	-0.002147003	&	-0.003273056	\\
$\Delta x_{10}$	&		&		&		&		&		&		&		&		&	-0.002531399	&	0.001724576	&	0.00220416	&	0.001448066	&	0.000191091	&	-0.001232277	\\
$\Delta x_{11}$	&		&		&		&		&		&		&		&		&		&	-0.002065804	&	0.00155475	&	0.002573108	&	0.002301974	&	0.00133333	\\
$\Delta x_{12}$	&		&		&		&		&		&		&		&		&		&		&	-0.002220146	&	0.001412978	&	0.002905659	&	0.003101475	\\
$\Delta x_{13}$	&		&		&		&		&		&		&		&		&		&		&		&	-0.002658265	&	0.00118708	&	0.003127874	\\
$\Delta x_{14}$	&		&		&		&		&		&		&		&		&		&		&		&		&	-0.003254129	&	0.000856185	\\
$\Delta x_{15}$	&		&		&		&		&		&		&		&		&		&		&		&		&		&	-0.003954619	\\
\bottomrule
\end{tabular}
%\small Note: Robust standard errors in parentheses. Intercept included but not reported. \begin{tablenotes}\item[*] significant at 5\% level \item[**] significant at 10\% level\end{tablenotes}
\end{threeparttable}}%
\end{table*}
\end{center}
%--------------------------- SOR 残差向量 r ---------------------------------%
\begin{center}
\begin{table*}[!htbp]
\resizebox{\textwidth}{!}{ %
\begin{threeparttable}[!htpb]
 \caption{\label{tab:SOR_r}n=2-15时SOR方法计算所得的x的残差$r=b-Ax$}
 \begin{tabular}{ccccc ccccc ccccc}
 \toprule
	&	n=2	&	n=3	&	n=4	&	n=5	&	n=6	&	n=7	&	n=8	&	n=9	&	n=10	&	n=11	&	n=12	&	n=13	&	n=14	&	n=15	\\ \hline
$r_{1}$	&	-3.9088E-09	&	1.82634E-09	&	-3.79215E-10	&	8.16662E-11	&	-1.81815E-11	&	-3.59917E-11	&	-6.22569E-11	&	-1.21566E-10	&	2.01088E-11	&	2.66671E-11	&	3.50369E-11	&	4.68301E-11	&	6.53038E-11	&	9.63802E-11	\\
$r_{2}$	&	-1.32666E-10	&	-2.12878E-09	&	1.15464E-09	&	-4.5585E-10	&	1.5966E-10	&	2.74189E-10	&	4.18583E-10	&	7.16711E-10	&	-1.87681E-10	&	-2.30796E-10	&	-2.82693E-10	&	-3.52599E-10	&	-4.5817E-10	&	-6.29833E-10	\\
$r_{3}$	&		&	-8.7105E-11	&	-9.68329E-10	&	9.09094E-10	&	-5.49664E-10	&	-7.7201E-10	&	-9.78097E-10	&	-1.33255E-09	&	6.61095E-10	&	7.36618E-10	&	8.20393E-10	&	9.26417E-10	&	1.07853E-09	&	1.31427E-09	\\
$r_{4}$	&		&		&	-4.34088E-11	&	-5.39092E-10	&	7.52969E-10	&	6.69282E-10	&	4.80739E-10	&	7.30567E-11	&	-7.71252E-10	&	-7.02705E-10	&	-6.26435E-10	&	-5.29674E-10	&	-3.9057E-10	&	-1.74539E-10	\\
$r_{5}$	&		&		&		&	-2.56981E-11	&	-3.38475E-10	&	1.98145E-10	&	5.32321E-10	&	9.29068E-10	&	-1.03178E-10	&	-2.65752E-10	&	-4.06147E-10	&	-5.52298E-10	&	-7.29506E-10	&	-9.61497E-10	\\
$r_{6}$	&		&		&		&		&	-1.69431E-11	&	-3.38612E-10	&	-5.19926E-11	&	4.9876E-10	&	3.74031E-10	&	2.61896E-10	&	1.31003E-10	&	-2.99232E-11	&	-2.51174E-10	&	-5.78382E-10	\\
$r_{7}$	&		&		&		&		&		&	-1.5972E-11	&	-3.59798E-10	&	-2.62173E-10	&	2.71524E-10	&	3.42925E-10	&	3.53897E-10	&	3.25833E-10	&	2.52054E-10	&	1.01861E-10	\\
$r_{8}$	&		&		&		&		&		&		&	-1.66241E-11	&	-5.54515E-10	&	-4.89065E-11	&	1.29985E-10	&	2.61146E-10	&	3.66031E-10	&	4.58946E-10	&	5.3672E-10	\\
$r_{9}$	&		&		&		&		&		&		&		&	-2.54676E-11	&	-2.13524E-10	&	-1.11098E-10	&	3.64302E-11	&	1.95591E-10	&	3.79505E-10	&	6.05669E-10	\\
$r_{10}$	&		&		&		&		&		&		&		&		&	-9.97724E-12	&	-1.8919E-10	&	-1.47675E-10	&	-3.01683E-11	&	1.46562E-10	&	4.04078E-10	\\
$r_{11}$	&		&		&		&		&		&		&		&		&		&	-8.74523E-12	&	-1.80203E-10	&	-1.85297E-10	&	-9.7775E-11	&	8.90015E-11	\\
$r_{12}$	&		&		&		&		&		&		&		&		&		&		&	-8.2786E-12	&	-1.90781E-10	&	-2.44759E-10	&	-1.95709E-10	\\
$r_{13}$	&		&		&		&		&		&		&		&		&		&		&		&	-8.73546E-12	&	-2.26581E-10	&	-3.44971E-10	\\
$r_{14}$	&		&		&		&		&		&		&		&		&		&		&		&		&	-1.03545E-11	&	-2.9443E-10	\\
$r_{15}$	&		&		&		&		&		&		&		&		&		&		&		&		&		&	-1.34334E-11	\\
\bottomrule
\end{tabular}
%\small Note: Robust standard errors in parentheses. Intercept included but not reported. \begin{tablenotes}\item[*] significant at 5\% level \item[**] significant at 10\% level\end{tablenotes}
\end{threeparttable}}%
\end{table*}
\end{center}

\begin{center}
\begin{table*}[!htbp]
\resizebox{\textwidth}{!}{ %
\begin{threeparttable}[!htpb]
 \caption{\label{tab:SOR_norm}n=2-15时SOR方法误差向量的范数}
 \begin{tabular}{ccccc ccccc ccccc}
 \toprule
	&n=2	&	n=3	&	n=4	&	n=5	&	n=6	&	n=7	&	n=8	&	n=9	&	n=10	&	n=11	&	n=12	&	n=13	&	n=14	&	n=15	\\ \hline
$\|\Delta x\|_2$&2.64216E-08&	6.30085E-07&	1.17342E-05&	0.000279142&	0.007737784&	0.002795579&	0.005492658&	0.009403121&	0.006854686&	0.004466417&	0.005342856&	0.007228831&	0.009423723&	0.011820538 \\
\bottomrule
\end{tabular}
%\small Note: Robust standard errors in parentheses. Intercept included but not reported. \begin{tablenotes}\item[*] significant at 5\% level \item[**] significant at 10\% level\end{tablenotes}
\end{threeparttable}}%
\end{table*}
\end{center}



\section{共轭梯度法(CG)}
\subsection{CG法基本原理}
我们要求线性方程组$Ax=b$的解, 等价于求函数
\begin{equation}
\varphi(x)=\frac{1}{2}(Ax,x)-(b,x)
\end{equation}
的最小值.\par
共轭梯度法(Conjugate Gradient)是一种也是采用一维极小搜索的概念,但是不再沿有正交性的$r^{(0)}$, $r^{(1)},\cdots$方向进行搜索, 而是要找另一组方向$p^{(0)},p^{(1)},\cdots$,这些方向向量是A-共轭的向量组(或者说A-正交向量组), 也就是说它们满足
\begin{equation}
(Ap^{(i)}, p^{(j)})=0, i\neq j
\end{equation}
我们从一个初始点$x^{(0)}$出发, 选取搜索方向$p^{(0)}$和合适的步长$\alpha_0$就可以得到$^{(1)}$.也就是说,对于$x^{(k+1)}$满足
\begin{equation}
x^{(k+1)}=x^{(k)}+\alpha_kp^{(k)}
\end{equation}
我们希望$\alpha$和${p^{(k)}}$向量组的选取, 不仅希望使
\begin{equation}
\varphi(x^{(k+1)})=\mathop{\text{min}}_{\alpha} \varphi(x^{(k)}+\alpha_kp^{(k)})
\end{equation}
而且还希望${p^{(k)}}$的选取使得
\begin{equation}
\varphi(x^{(k+1)})=\mathop{\text{min}}_{x\in\text{span}\{p^{(0)},p^{(1)},\cdots,p^{(k)}\}} \varphi(x)
\end{equation}
通过一系列运算, 我们可以得到
\begin{equation}
\alpha_k= \frac{(r^{(k)},p^{(k)})}{(Ap^{(k)},p^{(k)})}
\end{equation}
对于$p^{(k)}$的选取, 我们要让它与$p^{(0)},p^{(1)},\cdots,p^{(k-1)}$都共轭, 这种向量的取法不是唯一的, 在这里, 我们取得是$r^{(k)}$与$p^{(k-1)}$的线性组合, 即
\begin{equation}
p^{(k)}=r^{(k)}+\beta_{k-1} p^{(k-1)}
\end{equation}
通过令$(Ap^{(k)},p^{(k)})=0$, 我们可以得出
$$\beta_{k-1}= -\frac{(r^{(k)},Ap^{(k-1)})}{(Ap^{(k-1)},p^{(k-1)})}$$
从而$\beta_k$满足
\begin{equation}
\beta_{k}= -\frac{(r^{(k+1)},Ap^{(k)})}{(Ap^{(k)},p^{(k)})}
\end{equation}
上面的$p^{(k)}$的选取方法可以证明${p^{(k)}}$构成了一个A-共轭向量组.\par
综上所述, CG算法的步骤如下:
\begin{enumerate}[(1)]
\item 任取$x^{(0)}\in \mathbb{R}^n$;
\item $r^{(0)}=b-Ax^{(0)}, p^{(0)}=r^{(0)};$
\item 对于$k=0,1,\cdots,$
$$\alpha_k= \frac{(r^{(k)},p^{(k)})}{(Ap^{(k)},p^{(k)})}$$
$$x^{(k+1)}=x^{(k)}+\alpha_kp^{(k)}$$
$$r^{(k+1)}=r^{(k)}-\alpha_kAp^{(k)}$$
$$\beta_{k}= -\frac{(r^{(k+1)},Ap^{(k)})}{(Ap^{(k)},p^{(k)})}$$
$$p^{(k+1)}=r^{(k+1)}+\beta_{k} p^{(k)}$$
\end{enumerate}
对于上述(3), 我们可以进行适当的简化, 变成
$$\alpha_k= \frac{(r^{(k)},r^{(k)})}{(Ap^{(k)},p^{(k)})}$$
$$x^{(k+1)}=x^{(k)}+\alpha_kp^{(k)}$$
$$r^{(k+1)}=r^{(k)}-\alpha_kAp^{(k)}$$
$$\beta_{k}= -\frac{(r^{(k+1)},r^{(k+1)})}{(r^{(k)},r^{(k)})}$$
$$p^{(k+1)}=r^{(k+1)}+\beta_{k} p^{(k)}$$

\subsection{实验设计}
我们编写了函数CG.m采用上述CG算法对线性方程组进行求解, 其中该函数有两个参数, 分别是A和b, 即线性方程组$Ax=b$的系数矩阵和常数向量, 返回值是线性方程组的解. 在求解过程中, 当前后两次$x$的结果小于$10^{-8}$时, 我们就停止迭代过程, 直接输出$x$. 在函数solve\_hilbert\_equ.m中, 我们在$n=2-15$的循环中对CG.m进行了调用, 得到了方程组的解, 并将每次的结果写在了Excel表格中. 同时, 我们还分析了结果的误差和残差向量的大小.

\subsection{实验结果及分析}
Table \ref{tab:CG_x} 是采用CG方法时计算出的结果, 从中可以看出, 即使是当n很大时, 计算出的结果仍然比较准确, 这一点我们也可以通过Table \ref{tab:CG_delta_x} 看出. Table \ref{tab:CG_delta_x} 表示的是我们计算出的$x$与理论真解$x$值之间的误差, 可以看出即使当n很大时, 误差的结果仍然很小. 此时, 通过 Table \ref{tab:CG_r} 我们可以看出, 残差也非常小, 达到了$10^{-9}, 10^{-10}$左右的数量级, 从另一个侧面反映出了结果的准确性. 从Table \ref{tab:CG_norm} 中可以看出, 误差向量的二范数在$n=15$时达到了0.004318824
.

%---------------------表格 SOR方法 x -------------------------%
\begin{center}
\begin{table*}[!htbp]
\resizebox{\textwidth}{!}{ %
\begin{threeparttable}[!htpb]
 \caption{\label{tab:CG_x}n=2-15时CG方法计算所得的结果}
 \begin{tabular}{ccccc ccccc ccccc}
 \toprule
 	&	n=2	&	n=3	&	n=4	&	n=5	&	n=6	&	n=7	&	n=8	&	n=9	&	n=10	&	n=11	&	n=12	&	n=13	&	n=14	&	n=15	\\ \hline
$x_{1}$	&	1	&	1	&	0.999999995	&	1	&	1.000001161	&	1.000004967	&	1.000012767	&	1.000025608	&	0.999998591	&	0.999997335	&	0.9999955	&	0.999992987	&	0.999989716	&	0.999985621	\\
$x_{2}$	&	1	&	1	&	0.999999997	&	1	&	0.999966893	&	0.999876321	&	0.999715578	&	0.999480264	&	1.000042231	&	1.000074161	&	1.000117519	&	1.000173019	&	1.000241048	&	1.00032174	\\
$x_{3}$	&		&	1	&	0.999999998	&	1	&	1.000223778	&	1.000690539	&	1.001360589	&	1.002183407	&	0.999719051	&	0.999549659	&	0.999341633	&	0.999098319	&	0.99882348	&	0.998520949	\\
$x_{4}$	&		&		&	0.999999998	&	1	&	0.999418469	&	0.998758943	&	0.998230288	&	0.997908104	&	1.000599223	&	1.000820994	&	1.001037884	&	1.001239441	&	1.001418425	&	1.001570103	\\
$x_{5}$	&		&		&		&	1	&	1.000641352	&	1.000229423	&	0.999354479	&	0.998333188	&	0.999825585	&	0.999974797	&	1.00019086	&	1.000458276	&	1.000762499	&	1.001090819	\\
$x_{6}$	&		&		&		&		&	0.999747475	&	1.001211514	&	1.001369856	&	1.000788987	&	0.999494975	&	0.999348737	&	0.999285468	&	0.999306502	&	0.999405871	&	0.999574327	\\
$x_{7}$	&		&		&		&		&		&	0.99922407	&	1.001461138	&	1.00229747	&	0.999890578	&	0.999564828	&	0.999253107	&	0.998992146	&	0.998799709	&	0.998682263	\\
$x_{8}$	&		&		&		&		&		&		&	0.998483224	&	1.001371353	&	1.000432121	&	1.000215896	&	0.999870712	&	0.999479015	&	0.99909189	&	0.998740105	\\
$x_{9}$	&		&		&		&		&		&		&		&	0.997585006	&	1.000466468	&	1.000686786	&	1.00058524	&	1.000292935	&	0.999898231	&	0.999460503	\\
$x_{10}$	&		&		&		&		&		&		&		&		&	0.999529785	&	1.000480007	&	1.00088924	&	1.000935185	&	1.00074327	&	1.000402385	\\
$x_{11}$	&		&		&		&		&		&		&		&		&		&	0.999283956	&	1.000428184	&	1.001024116	&	1.001231463	&	1.001167247	\\
$x_{12}$	&		&		&		&		&		&		&		&		&		&		&	0.998999504	&	1.000315992	&	1.001088282	&	1.001458829	\\
$x_{13}$	&		&		&		&		&		&		&		&		&		&		&		&	0.998683522	&	1.000150588	&	1.001084962	\\
$x_{14}$	&		&		&		&		&		&		&		&		&		&		&		&		&	0.998342246	&	0.999939561	\\
$x_{15}$	&		&		&		&		&		&		&		&		&		&		&		&		&		&	0.997980981	\\
\bottomrule
\end{tabular}
%\small Note: Robust standard errors in parentheses. Intercept included but not reported. \begin{tablenotes}\item[*] significant at 5\% level \item[**] significant at 10\% level\end{tablenotes}
\end{threeparttable}}%
\end{table*}
\end{center}

%-------------------------- CG Delta x ----------------------------%
\begin{center}
\begin{table*}[!htbp]
\resizebox{\textwidth}{!}{ %
\begin{threeparttable}[!htpb]
 \caption{\label{tab:CG_delta_x}n=2-15时CG方法计算所得的x的误差$\Delta x$}
 \begin{tabular}{ccccc ccccc ccccc}
 \toprule
	&	n=2	&	n=3	&	n=4	&	n=5	&	n=6	&	n=7	&	n=8	&	n=9	&	n=10	&	n=11	&	n=12	&	n=13	&	n=14	&	n=15	\\ \hline
$\Delta x_{1}$	&	2.88658E-15	&	1.11022E-14	&	-5.476E-09	&	-1.91243E-11	&	1.16085E-06	&	4.96688E-06	&	1.27673E-05	&	2.56079E-05	&	-1.40923E-06	&	-2.6655E-06	&	-4.50039E-06	&	-7.01277E-06	&	-1.02842E-05	&	-1.43792E-05	\\
$\Delta x_{2}$	&	1.55431E-15	&	9.99201E-15	&	-3.13044E-09	&	6.69838E-11	&	-3.31069E-05	&	-0.000123679	&	-0.000284422	&	-0.000519736	&	4.22309E-05	&	7.41606E-05	&	0.000117519	&	0.000173019	&	0.000241048	&	0.00032174	\\
$\Delta x_{3}$	&		&	-6.88338E-15	&	-2.19943E-09	&	-1.2107E-12	&	0.000223778	&	0.000690539	&	0.001360589	&	0.002183407	&	-0.000280949	&	-0.000450341	&	-0.000658367	&	-0.000901681	&	-0.00117652	&	-0.001479051	\\
$\Delta x_{4}$	&		&		&	-1.75614E-09	&	-1.01916E-11	&	-0.000581531	&	-0.001241057	&	-0.001769712	&	-0.002091896	&	0.000599223	&	0.000820994	&	0.001037884	&	0.001239441	&	0.001418425	&	0.001570103	\\
$\Delta x_{5}$	&		&		&		&	-5.81637E-11	&	0.000641352	&	0.000229423	&	-0.000645521	&	-0.001666812	&	-0.000174415	&	-2.52035E-05	&	0.00019086	&	0.000458276	&	0.000762499	&	0.001090819	\\
$\Delta x_{6}$	&		&		&		&		&	-0.000252525	&	0.001211514	&	0.001369856	&	0.000788987	&	-0.000505025	&	-0.000651263	&	-0.000714532	&	-0.000693498	&	-0.000594129	&	-0.000425673	\\
$\Delta x_{7}$	&		&		&		&		&		&	-0.00077593	&	0.001461138	&	0.00229747	&	-0.000109422	&	-0.000435172	&	-0.000746893	&	-0.001007854	&	-0.001200291	&	-0.001317737	\\
$\Delta x_{8}$	&		&		&		&		&		&		&	-0.001516776	&	0.001371353	&	0.000432121	&	0.000215896	&	-0.000129288	&	-0.000520985	&	-0.00090811	&	-0.001259895	\\
$\Delta x_{9}$	&		&		&		&		&		&		&		&	-0.002414994	&	0.000466468	&	0.000686786	&	0.00058524	&	0.000292935	&	-0.000101769	&	-0.000539497	\\
$\Delta x_{10}$	&		&		&		&		&		&		&		&		&	-0.000470215	&	0.000480007	&	0.00088924	&	0.000935185	&	0.00074327	&	0.000402385	\\
$\Delta x_{11}$	&		&		&		&		&		&		&		&		&		&	-0.000716044	&	0.000428184	&	0.001024116	&	0.001231463	&	0.001167247	\\
$\Delta x_{12}$	&		&		&		&		&		&		&		&		&		&		&	-0.001000496	&	0.000315992	&	0.001088282	&	0.001458829	\\
$\Delta x_{13}$	&		&		&		&		&		&		&		&		&		&		&		&	-0.001316478	&	0.000150588	&	0.001084962	\\
$\Delta x_{14}$	&		&		&		&		&		&		&		&		&		&		&		&		&	-0.001657754	&	-6.0439E-05	\\
$\Delta x_{15}$	&		&		&		&		&		&		&		&		&		&		&		&		&		&	-0.002019019	\\
\bottomrule
\end{tabular}
%\small Note: Robust standard errors in parentheses. Intercept included but not reported. \begin{tablenotes}\item[*] significant at 5\% level \item[**] significant at 10\% level\end{tablenotes}
\end{threeparttable}}%
\end{table*}
\end{center}
%--------------------- CG r----------------------------%
\begin{center}
\begin{table*}[!htbp]
\resizebox{\textwidth}{!}{ %
\begin{threeparttable}[!htpb]
 \caption{\label{tab:CG_r}n=2-15时CG方法计算所得的x的残差$r=b-Ax$}
 \begin{tabular}{ccccc ccccc ccccc}
 \toprule
	&	n=2	&	n=3	&	n=4	&	n=5	&	n=6	&	n=7	&	n=8	&	n=9	&	n=10	&	n=11	&	n=12	&	n=13	&	n=14	&	n=15	\\ \hline
$r _{1}$	&	-3.55271E-15	&	-1.37668E-14	&	8.21339E-09	&	2.16271E-13	&	-6.15827E-11	&	4.11347E-10	&	-5.04263E-11	&	-1.37405E-10	&	-1.24802E-10	&	4.14779E-13	&	-2.38716E-11	&	-2.24802E-11	&	4.94675E-11	&	6.18807E-11	\\
$r _{2}$	&	-1.9984E-15	&	-7.10543E-15	&	4.68256E-09	&	-7.30971E-13	&	2.46692E-11	&	-4.67535E-11	&	3.7727E-10	&	1.5005E-09	&	2.4154E-10	&	-1.80789E-11	&	-5.66658E-12	&	-7.25051E-11	&	-3.0887E-10	&	-4.55455E-10	\\
$r _{3}$	&		&	-4.88498E-15	&	3.34052E-09	&	-1.21458E-13	&	1.23026E-11	&	-5.58534E-10	&	-1.85518E-09	&	-6.23657E-09	&	1.92464E-10	&	1.15119E-10	&	3.25267E-10	&	6.79465E-10	&	1.22188E-09	&	2.28631E-09	\\
$r _{4}$	&		&		&	2.61253E-09	&	3.12417E-13	&	1.02497E-10	&	3.98693E-10	&	2.55719E-09	&	6.67677E-09	&	-2.1174E-11	&	-2.18952E-10	&	-4.73774E-10	&	-9.3297E-10	&	-1.68944E-09	&	-2.68549E-09	\\
$r _{5}$	&		&		&		&	5.70544E-13	&	-3.05532E-11	&	-4.07377E-10	&	5.6006E-10	&	3.88814E-09	&	2.02176E-11	&	3.32578E-11	&	-1.76505E-11	&	-1.93477E-10	&	-6.03065E-10	&	-1.32712E-09	\\
$r _{6}$	&		&		&		&		&	6.44179E-11	&	-7.64671E-10	&	-1.83326E-09	&	-2.80397E-09	&	1.26565E-13	&	1.61806E-10	&	3.23486E-10	&	5.48714E-10	&	7.96409E-10	&	9.72499E-10	\\
$r _{7}$	&		&		&		&		&		&	1.40887E-10	&	-1.55253E-09	&	-5.63901E-09	&	-8.98244E-11	&	8.03863E-11	&	2.56775E-10	&	6.07121E-10	&	1.17122E-09	&	1.89065E-09	\\
$r _{8}$	&		&		&		&		&		&		&	1.79958E-09	&	-2.58523E-09	&	-1.70271E-10	&	-6.83963E-11	&	-2.09688E-11	&	2.02043E-10	&	6.97863E-10	&	1.44407E-09	\\
$r _{9}$	&		&		&		&		&		&		&		&	5.7437E-09	&	-1.80055E-10	&	-1.48772E-10	&	-2.69393E-10	&	-2.90851E-10	&	-9.19697E-11	&	3.20513E-10	\\
$r _{10}$	&		&		&		&		&		&		&		&		&	-9.24143E-11	&	-8.4861E-11	&	-3.31745E-10	&	-5.92696E-10	&	-7.53008E-10	&	-8.28254E-10	\\
$r _{11}$	&		&		&		&		&		&		&		&		&		&	1.48029E-10	&	-1.37795E-10	&	-5.54208E-10	&	-1.01568E-09	&	-1.56779E-09	\\
$r _{12}$	&		&		&		&		&		&		&		&		&		&		&	3.22969E-10	&	-1.25129E-10	&	-7.59572E-10	&	-1.6712E-09	\\
$r _{13}$	&		&		&		&		&		&		&		&		&		&		&		&	6.83114E-10	&	3.52884E-11	&	-1.0617E-09	\\
$r _{14}$	&		&		&		&		&		&		&		&		&		&		&		&		&	1.33019E-09	&	2.4331E-10	\\
$r _{15}$	&		&		&		&		&		&		&		&		&		&		&		&		&		&	2.17424E-09	\\
\bottomrule
\end{tabular}
%\small Note: Robust standard errors in parentheses. Intercept included but not reported. \begin{tablenotes}\item[*] significant at 5\% level \item[**] significant at 10\% level\end{tablenotes}
\end{threeparttable}}%
\end{table*}
\end{center}


\begin{center}
\begin{table*}[!htbp]
\resizebox{\textwidth}{!}{ %
\begin{threeparttable}[!htpb]
 \caption{\label{tab:CG_norm}n=2-15时CG方法误差向量的范数}
 \begin{tabular}{ccccc ccccc ccccc}
 \toprule
	&n=2	&	n=3	&	n=4	&	n=5	&	n=6	&	n=7	&	n=8	&	n=9	&	n=10	&	n=11	&	n=12	&	n=13	&	n=14	&	n=15	\\ \hline
$\|\Delta x\|_2$& 3.27845E-15&	1.64463E-14&	6.90708E-09&	9.13286E-11&	0.00092976&	0.002038347&	0.003434076&	0.005079961&	0.001167327&	0.001660695&	0.002227918	&0.002863483&	0.003562104&	0.004318824\\
\bottomrule
\end{tabular}
%\small Note: Robust standard errors in parentheses. Intercept included but not reported. \begin{tablenotes}\item[*] significant at 5\% level \item[**] significant at 10\% level\end{tablenotes}
\end{threeparttable}}%
\end{table*}
\end{center}



%-------Gauss-------------%
%%%%%%%%%%%%%%%%%%%%%%%% Delta x%%%%%%%%%%%%%%%%%%%%%%5
\begin{center}
\begin{table*}[!htbp]
\resizebox{\textwidth}{!}{ %
\begin{threeparttable}[!htpb]
 \caption{\label{tab:Gauss_dx_regu}n=2-15时Gauss消去法正则化后计算结果的误差$\Delta x$}
 \begin{tabular}{ccccc ccccc ccccc}
 \toprule
 &n=2&n=3 &n=4 &n=5 &n=6 &n=7 &n=8 &n=9 &n=10 &n=11 &n=12 &n=13 &n=14 &n=15 \\ \hline
$\Delta x_{1}$	&	0.004298821	&	-0.009991912	&	-0.019417952	&	-0.019488126	&	-0.015355294	&	-0.010422874	&	-0.005647367	&	-0.001217068	&	0.002834614	&	0.006488677	&	0.009722412	&	0.012515934	&	0.014859627	&	0.016757701	\\
$\Delta x_{2}$	&	-0.008210934	&	0.066041208	&	0.08115782	&	0.058334086	&	0.029012539	&	0.003561422	&	-0.017241849	&	-0.034167283	&	-0.047847811	&	-0.058672041	&	-0.06689584	&	-0.072732553	&	-0.076399644	&	-0.07813359	\\
$\Delta x_{3}$	&		&	-0.068628197	&	0.004347947	&	0.03793107	&	0.047284045	&	0.04635532	&	0.041139992	&	0.033958205	&	0.025835772	&	0.01733812	&	0.008843097	&	0.00062883	&	-0.00709854	&	-0.014194716	\\
$\Delta x_{4}$	&		&		&	-0.090295179	&	-0.02033764	&	0.018198443	&	0.038706139	&	0.049461902	&	0.054445166	&	0.055638232	&	0.054159257	&	0.050738743	&	0.045916888	&	0.040124269	&	0.033713715	\\
$\Delta x_{5}$	&		&		&		&	-0.086575995	&	-0.029497785	&	0.006022157	&	0.029001798	&	0.04416527	&	0.053958481	&	0.059765514	&	0.062474449	&	0.062731848	&	0.061055444	&	0.057881672	\\
$\Delta x_{6}$	&		&		&		&		&	-0.082252121	&	-0.036714928	&	-0.004960087	&	0.018003781	&	0.034815755	&	0.04696769	&	0.055400269	&	0.06078449	&	0.063652814	&	0.06445756	\\
$\Delta x_{7}$	&		&		&		&		&		&	-0.082278974	&	-0.044375911	&	-0.015622288	&	0.006670544	&	0.024018468	&	0.037361094	&	0.047353433	&	0.054507111	&	0.059256218	\\
$\Delta x_{8}$	&		&		&		&		&		&		&	-0.085155772	&	-0.052176377	&	-0.025691405	&	-0.004204446	&	0.013196704	&	0.027132485	&	0.038077667	&	0.046432213	\\
$\Delta x_{9}$	&		&		&		&		&		&		&		&	-0.089229706	&	-0.059580679	&	-0.03484679	&	-0.014155129	&	0.003072943	&	0.017269504	&	0.028794211	\\
$\Delta x_{10}$	&		&		&		&		&		&		&		&		&	-0.093488959	&	-0.066235936	&	-0.042908224	&	-0.022970389	&	-0.006033655	&	0.008220257	\\
$\Delta x_{11}$	&		&		&		&		&		&		&		&		&		&	-0.097394675	&	-0.071974731	&	-0.049827927	&	-0.030606819	&	-0.014032963	\\
$\Delta x_{12}$	&		&		&		&		&		&		&		&		&		&		&	-0.100695522	&	-0.076762076	&	-0.055651422	&	-0.03712218	\\
$\Delta x_{13}$	&		&		&		&		&		&		&		&		&		&		&		&	-0.103310933	&	-0.0806477	&	-0.060480945	\\
$\Delta x_{14}$	&		&		&		&		&		&		&		&		&		&		&		&		&	-0.105260158	&	-0.08372986	\\
$\Delta x_{15}$	&		&		&		&		&		&		&		&		&		&		&		&		&		&	-0.106617096	\\
\bottomrule
\end{tabular}
%\small Note: Robust standard errors in parentheses. Intercept included but not reported. \begin{tablenotes}\item[*] significant at 5\% level \item[**] significant at 10\% level\end{tablenotes}
\end{threeparttable}}%
\end{table*}
\end{center}



\section{与直接法结果的比较}
这里我仅仅列上采用正则化之后的直接法(Gauss消去法和Cholesky分解法)的误差结果, 作为与本实验结果的比较, 如Table \ref{tab:Gauss_dx_regu} 和 Table \ref{tab:Chol_dx_regu} 所示.


通过Table \ref{tab:Gauss_dx_regu}, Table \ref{tab:Chol_dx_regu}与 Table \ref{tab:SOR_delta_x} 和 Table \ref{tab:CG_delta_x} 对比可以看出, 不论是采用SOR方法还是CG方法, 采用迭代法之后的效果比直接法的结果更加准确. 我们再来看一下误差向量的情况. 如 Table \ref{tab:Gauss_regu_norm} 和 Table \ref{tab:Chol_dx_regu}所示, 是采用正则化后的Gauss消去法和Cholesky分解法时所得的误差向量的二范数(\textbf{注意:它们实际上不相等,相差$10^{-8}$左右, 但这里取的位数比较少, 因此看起来相等, 实际上不相等, 特此说明.}) 可以看出, 采用正则化后的Gauss消去法和Cholesky分解法时, 当$n=15$时, 误差的向量范数都在0.0324左右. 从 Table \ref{tab:SOR_norm} 和 Table \ref{tab:CG_norm}中可以看出, SOR方法和CG方法的误差向量的二范数在$n=15$分别为0.011820538和0.004318824, 明显小于采用正则化后直接法的结果. 因此可以说采用迭代法的效果较好.
%-------------Cholesky------------------------%
\begin{center}
\begin{table*}[!htbp]
\resizebox{\textwidth}{!}{ %
\begin{threeparttable}[!htpb]
 \caption{\label{tab:Chol_dx_regu}n=2-15时Cholesky分解法正则化后计算结果的误差向量$\Delta x$}
 \begin{tabular}{ccccc ccccc ccccc}
 \toprule

 &n=2&n=3 &n=4 &n=5 &n=6 &n=7 &n=8 &n=9 &n=10 &n=11 &n=12 &n=13 &n=14 &n=15 \\ \hline
$\Delta x_{1}$	&	0.004298821	&	-0.009991912	&	-0.019417952	&	-0.019488126	&	-0.015355294	&	-0.010422874	&	-0.005647367	&	-0.001217068	&	0.002834614	&	0.006488677	&	0.009722412	&	0.012515934	&	0.014859627	&	0.016757701	\\
$\Delta x_{2}$	&	-0.008210934	&	0.066041208	&	0.08115782	&	0.058334086	&	0.029012539	&	0.003561422	&	-0.017241849	&	-0.034167283	&	-0.047847811	&	-0.058672041	&	-0.06689584	&	-0.072732553	&	-0.076399644	&	-0.07813359	\\
$\Delta x_{3}$	&		&	-0.068628197	&	0.004347947	&	0.03793107	&	0.047284045	&	0.04635532	&	0.041139992	&	0.033958205	&	0.025835772	&	0.01733812	&	0.008843097	&	0.00062883	&	-0.00709854	&	-0.014194716	\\
$\Delta x_{4}$	&		&		&	-0.090295179	&	-0.02033764	&	0.018198443	&	0.038706139	&	0.049461902	&	0.054445166	&	0.055638232	&	0.054159257	&	0.050738743	&	0.045916888	&	0.040124269	&	0.033713715	\\
$\Delta x_{5}$	&		&		&		&	-0.086575995	&	-0.029497785	&	0.006022157	&	0.029001798	&	0.04416527	&	0.053958481	&	0.059765514	&	0.062474449	&	0.062731848	&	0.061055444	&	0.057881672	\\
$\Delta x_{6}$	&		&		&		&		&	-0.082252121	&	-0.036714928	&	-0.004960087	&	0.018003781	&	0.034815755	&	0.04696769	&	0.055400269	&	0.06078449	&	0.063652814	&	0.06445756	\\
$\Delta x_{7}$	&		&		&		&		&		&	-0.082278974	&	-0.044375911	&	-0.015622288	&	0.006670544	&	0.024018468	&	0.037361094	&	0.047353433	&	0.054507111	&	0.059256218	\\
$\Delta x_{8}$	&		&		&		&		&		&		&	-0.085155772	&	-0.052176377	&	-0.025691405	&	-0.004204446	&	0.013196704	&	0.027132485	&	0.038077667	&	0.046432213	\\
$\Delta x_{9}$	&		&		&		&		&		&		&		&	-0.089229706	&	-0.059580679	&	-0.03484679	&	-0.014155129	&	0.003072943	&	0.017269504	&	0.028794211	\\
$\Delta x_{10}$	&		&		&		&		&		&		&		&		&	-0.093488959	&	-0.066235936	&	-0.042908224	&	-0.022970389	&	-0.006033655	&	0.008220257	\\
$\Delta x_{11}$	&		&		&		&		&		&		&		&		&		&	-0.097394675	&	-0.071974731	&	-0.049827927	&	-0.030606819	&	-0.014032963	\\
$\Delta x_{12}$	&		&		&		&		&		&		&		&		&		&		&	-0.100695522	&	-0.076762076	&	-0.055651422	&	-0.03712218	\\
$\Delta x_{13}$	&		&		&		&		&		&		&		&		&		&		&		&	-0.103310933	&	-0.0806477	&	-0.060480945	\\
$\Delta x_{14}$	&		&		&		&		&		&		&		&		&		&		&		&		&	-0.105260158	&	-0.08372986	\\
$\Delta x_{15}$	&		&		&		&		&		&		&		&		&		&		&		&		&		&	-0.106617096	\\
 \bottomrule
\end{tabular}
%\small Note: Robust standard errors in parentheses. Intercept included but not reported. \begin{tablenotes}\item[*] significant at 5\% level \item[**] significant at 10\% level\end{tablenotes}
\end{threeparttable}}%
\end{table*}
\end{center}


\begin{center}
\begin{table*}[!htbp]
\resizebox{\textwidth}{!}{ %
\begin{threeparttable}[!htpb]
 \caption{\label{tab:Gauss_regu_norm}n=2-15时Gauss方法正则化后计算结果误差向量的范数}
 \begin{tabular}{ccccc ccccc ccccc}
 \toprule
	&n=2	&	n=3	&	n=4	&	n=5	&	n=6	&	n=7	&	n=8	&	n=9	&	n=10	&	n=11	&	n=12	&	n=13	&	n=14	&	n=15	\\ \hline
$\|\Delta x\|_2$& 9.4824E-06&0.001401083&	0.020726106&	0.022569853	&0.016451714&	0.017159102	& 0.021882421	&0.02717609	& 0.031531454	&0.034181662	&0.035008877	&0.034494486	&0.033405417&	0.032438342\\
\bottomrule
\end{tabular}
%\small Note: Robust standard errors in parentheses. Intercept included but not reported. \begin{tablenotes}\item[*] significant at 5\% level \item[**] significant at 10\% level\end{tablenotes}
\end{threeparttable}}%
\end{table*}
\end{center}




\begin{center}
\begin{table*}[!htbp]
\resizebox{\textwidth}{!}{ %
\begin{threeparttable}[!htpb]
 \caption{\label{tab:Chol_regu_norm}n=2-15时Cholesky法正则化后计算结果误差向量的范数}
 \begin{tabular}{ccccc ccccc ccccc}
 \toprule
	&n=2	&	n=3	&	n=4	&	n=5	&	n=6	&	n=7	&	n=8	&	n=9	&	n=10	&	n=11	&	n=12	&	n=13	&	n=14	&	n=15	\\ \hline
$\|\Delta x\|_2$& 9.4824E-06&	0.001401083&	0.020726106&	0.022569854&	0.016451715&	0.017159102&	0.02188242&	0.027176094	&0.03153145&	0.034181662&	0.035008877	&0.034494485&	0.033405416	&0.032438342\\
\bottomrule
\end{tabular}
%\small Note: Robust standard errors in parentheses. Intercept included but not reported. \begin{tablenotes}\item[*] significant at 5\% level \item[**] significant at 10\% level\end{tablenotes}
\end{threeparttable}}%
\end{table*}
\end{center}
\section{实验小结}
同过这次实验可以看出, 采用迭代法对线性代数方程组求解(特别是大型稀疏矩阵)时效果比直接法要好, 精度更高. 并且, 通过SOR方法与CG方法的对比也可以看出, 采用CG算法的精度要比SOR方法高.



\section{Matlab源程序}
\begin{enumerate}[(1)]
\item \textbf{sor.m} \\
\begin{lstlisting}
function x=sor(A, b, w)
% x=sort(A, b, w)
% 超松弛迭代法
% x: return value, the solution to Ax=b
% w should between (0, 2)
% Author: LIU Qun
% Time: 2014-10-24

n=length(b);
x=zeros(n, 1);
err=1;
while(err>1e-8)
    x_init=x;
    for i=1:n
        x(i)=(1-w)*x_init(i)+w/A(i,i)*(b(i)-A(i,1:i-1)*x(1:i-1)-A(i,i+1:n)*x(i+1:n));
    end
    err=max(abs(x-x_init));
end
\end{lstlisting}



\item \textbf{CG.m} \\
\begin{lstlisting}
function x = CG(A, b)
% x = CG(A, b)
% Calculate Ax=b using Conjugate Gradient method
% Author: LIU Qun
% Email: liu-q14@mails.tsinghua.edu.cn
% Time: 2014-10-25

n=length(b);
x=zeros(n, 1);
rk=b-A*x;
p=rk;
while(max(abs(rk))>1e-8)
    alpha = (rk' * rk)/(p' * A*p);
    x = x + alpha * p;
    rk1 = rk - alpha*A*p;
    beta = (rk1'*rk1)/(rk'*rk);
    p = rk1 + beta * p;
    rk = rk1;
end
\end{lstlisting}

\item \textbf{solve\_hilbert\_equ.m}
\begin{lstlisting}
clear;
clc;

% 定义要写入位置
range = {'A','B','C','D','E','F','G','H','I', 'J','K','L','M','N','O'};
% 定义松弛因子w
w=1.05;

% 定义写入文件的名称
file_sor='sor_x.xlsx';
file_sor_err='sor_delta_x.xlsx';
file_sor_r='sor_r.xlsx';
file_sor_norm_dx='sor_norm_dx.xlsx';

file_CG='CG_x.xlsx';
file_CG_err='CG_delta_x.xlsx';
file_CG_r='CG_r.xlsx';
file_CG_norm_dx='CG_norm_dx.xlsx';
N=15;
norm_delta_x_sor = zeros(1,N-1);
norm_delta_x_CG = zeros(1,N-1);
for n=2:N
    A=hilb(n);
    x=ones(n,1);
    b=A*x;
    %%---------------------- SOR Method ----------------------%%
    y_sor=sor(A, b, w);
    delta_x_sor=y_sor-x;    % 解的误差
    r_sor=b-A*y_sor;        % 残差向量
    % 计算误差的向量范数(二范数)
    norm_delta_x_sor(n-1) = norm(delta_x_sor);
    % 写入x
    xlswrite(file_sor,y_sor,[range{n-1},'2:', range{n-1}, num2str(n+1)])
    xlswrite(file_sor,{['n=' num2str(n)]}, [range{n-1} '1:' range{n-1} '1']);
    % 写入x的误差
    xlswrite(file_sor_err,delta_x_sor,[range{n-1},'2:', range{n-1}, num2str(n+1)])
    xlswrite(file_sor_err,{['n=' num2str(n)]}, [range{n-1} '1:' range{n-1} '1']);
    % 写入残差
    xlswrite(file_sor_r,r_sor,[range{n-1},'2:', range{n-1}, num2str(n+1)])
    xlswrite(file_sor_r,{['n=' num2str(n)]}, [range{n-1} '1:' range{n-1} '1']);
    xlswrite(file_sor_norm_dx,{['n=' num2str(n)]},[range{n-1},'1:', range{n-1}, '1'])
    %%---------------------- CG Method ----------------------%%
    y_CG=CG(A, b);
    delta_x_CG=y_CG-x;    % 解的误差
    r_CG=b-A*y_CG;        % 残差向量
    % 计算误差的向量范数(二范数)
    norm_delta_x_CG(n-1) = norm(delta_x_CG);
    % 写入x
    xlswrite(file_CG,y_CG,[range{n-1},'2:', range{n-1}, num2str(n+1)])
    xlswrite(file_CG,{['n=' num2str(n)]}, [range{n-1} '1:' range{n-1} '1']);
    % 写入x的误差
    xlswrite(file_CG_err,delta_x_CG,[range{n-1},'2:', range{n-1}, num2str(n+1)])
    xlswrite(file_CG_err,{['n=' num2str(n)]}, [range{n-1} '1:' range{n-1} '1']);
    % 写入残差
    xlswrite(file_CG_r,r_CG,[range{n-1},'2:', range{n-1}, num2str(n+1)])
    xlswrite(file_CG_r,{['n=' num2str(n)]}, [range{n-1} '1:' range{n-1} '1']);
    xlswrite(file_CG_norm_dx,{['n=' num2str(n)]},[range{n-1},'1:', range{n-1}, '1'])
end
xlswrite(file_sor_norm_dx,norm_delta_x_sor,[range{1},'2:', range{N-1}, '2'])
xlswrite(file_CG_norm_dx,norm_delta_x_CG,[range{1},'2:', range{N-1}, '2'])
\end{lstlisting}
\end{enumerate}


%\newpage
%%%%%%%%%%%%%%%%%%%%Figure%%%%%%%%%%%%%%%%%%%%%%%%%%%%%%%%%%%%%%%%%%%%%%%%%%%%%%%%%%%
%%%%%%%%%%%%%%%%%%%%%%%%%%%%%%%%%%%%%%%%%%%%%%% Gauss results %%%%%%%%%%%%%%%%%%%%%%%%%%%%%%%%%%%%%%%
%\textbf{附录:}\par
%这里是本实验报告用到的所有的表格.


\end{CJK*}
\end{document}
