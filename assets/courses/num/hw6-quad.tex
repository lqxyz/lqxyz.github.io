\documentclass[a4paper]{article}
\usepackage{caption}
\usepackage{geometry}
%\geometry{left=2.5cm,right=2.5cm,top=2.5cm,bottom=2.5cm}
\usepackage{setspace}   %行间距的宏包
\usepackage{listings}
\lstset{language=Matlab}%代码语言使用的是matlab
\lstset{breaklines}%自动将长的代码行换行排版
\lstset{extendedchars=false}%解决代码跨页时,章节标题,页眉等汉字不显示的问题
\usepackage{CJK}
\usepackage{amsmath}
\usepackage{amsfonts,amssymb}
\usepackage{mathrsfs}% 手写字体
\usepackage{graphicx}
\usepackage{enumerate}
\usepackage{booktabs}
\usepackage{threeparttable}
\usepackage{rotating}
\usepackage{indentfirst}
\setlength\parindent{2em}
\begin{document}
\begin{spacing}{1.5}
\begin{CJK*}{GBK}{song}
\title{\bf{数值积分与数值微分上机实习报告}}
\author{\bf{姓名:刘群\ 学号:2014211591 \ 院系: 地学中心}\\ \bf{Email}:\ liu-q14@mails.tsinghua.edu.cn \\ (本文档由\LaTeX{}编写)}
\date{}
\maketitle
\section{问题的描述}
试用不同数值积分方法计算 $I(f) =\int_1^3 f(x)dx$ 的近似值, 其中$f(x) = \frac{1}{x^2}\text{sin}\frac{2\pi}{x}$.
注:$I(f) =-0.238732414637843\cdots$.\\
1、把[1,3]分成4个子区间,用五点Gauss-Legendre 求积公式的复合求积公式计算。\\
2、用 Romberg 求积算法计算积分,取$\varepsilon= 10^{-7}$, 并与第一种办法比较。
\section{方法描述}
在数值计算时我经常要计算某些积分的值, 一般来说, 对于比较简单的函数的积分来说, 我们可以通过 Newton-Leibniz 公式得到原函数, 然后再进行计算. 但是现实中, 有些情况下原函数特别复杂, 有些时候根本没有可以解析的原函数. 另外, 很多时候我们并不需要知道原函数的表达式, 只是希望得到一个积分的结果. 在这种情况下, 我们就需要用到数值积分的方法来计算积分的值. 常见的数值积分方法有 Newton-Cotes公式、复化求积公式、Gauss求积公式、Romberg求积等.

\subsection{Gauss-Legendre公式}
Gauss求积公式与 Newton-Cotes 公式的区别就是, 求积公式中的节点$x_0, x_1, \cdots, x_n$ 选取的是[a,b]上以$\rho$为权的$n+1$次正交多项式的零点. 对于本题中, 由于权函数为1, 因此可以选择Legendre多项式的根(Legendre多项式是[-1,1]上以$\rho(x)\equiv 1$的正交多项式). 这样, 我们可以得到Gauss-Legendre求积的表达式如下:
\begin{equation}\label{gl}
  \int_{-1}^1 f(x)dx \approx \sum_{k=0}^nA_kf(x_k)
\end{equation}
其中, Gauss点$x_0,x_1,\cdots, x_n$为$n+1$次 Legendre正交多项式$P_{n+1}$的零点. 公式(\ref{gl})中的系数$A_k$的表达式如下:
\begin{equation}\label{Ak}
  A_k =\frac{2}{(1-x_k^2)[P'_{n+1}(x_k)]^2}, k=0, 1, \cdots, n.
\end{equation}
或者
\begin{equation}\label{Ak2}
  A_k =\frac{2}{n+1}\frac{1}{P_{n}(x_k)P'_{n+1}(x_k)}, k=0, 1, \cdots, n.
\end{equation}
其中(\ref{Ak})式和(\ref{Ak2})式是等价的, 可以通过 Legendre 多项式的性质得到.\par

Gauss-Legendre求积公式中的节点$x_k$和系数$A_k$可以实现计算得到, 如下表所示.

\begin{table}[!htpb]
\begin{center}
\begin{spacing}{1.1}
\caption{Gauss-Legendre求积的节点和系数}\label{node_coef}
\begin{tabular}{ccc}

\toprule
n& $x_k$& $A_k$ \\
\midrule
1 &	0	              & 2 \\ \hline
2 & ±0.5773502692	  & 1 \\  \hline
3 & ±0.7745966692     & 0.5555555556 \\
  &  0                & 0.8888888889 \\  \hline
4 & ±0.8611363116     & 0.3478548451 \\
  & ±0.3399810436     & 0.6521451549 \\  \hline

5 &±0.9061798459      & 0.2369268851 \\
  &±0.5384693101      & 0.4786286705 \\
  &0                  & 0.5688888889 \\ \hline
6 &±0.9324695142      & 0.1713244924 \\
  &±0.6612093865      & 0.3607615730 \\
  &±0.2386191861	  & 0.4679139346 \\
\bottomrule
\end{tabular}
\end{spacing}
\end{center}
\end{table}
在一般的区间[a,b]上进行求积, 如果采用Gauss-Legendre求积公式, 可以按照下式进行变量替换,
\begin{equation}\label{vartrans}
  x=\frac{1}{2}(a+b)+\frac{1}{2}(b-a)t
\end{equation}
从而可以使区间[a,b]变换到[-1,1], 并且有
\begin{equation}\label{gl_trans}
  \int_{a}^b f(x)dx = \frac{a+b}{2}\int_{-1}^1 f\left[\frac{1}{2}(a+b)+\frac{1}{2}(b-a)t\right]dt
\end{equation}

对于本题中, 如果我们仅仅是采用Gauss-Legendre求积公式的话, 可以仅仅做变换
\begin{equation}\label{vartrans_t}
  x=2+t
\end{equation}
得到
\begin{equation}\label{gl_trans}
  \int_{1}^3 f(x)dx =2\int_{-1}^1 f(2+t)dt =2\int_{-1}^1 \frac{1}{t^2}\text{sin}\frac{2\pi}{t}dt
\end{equation}
但是, 这里我们要求采用复合五点Gauss-Legendre求积公式, 需要将区间[1,3]分为4个子区间, 因此, 我们下面就介绍一下如何构造复合五点Gauss-Legendre求积公式.

\subsection{复合Gauss-Legendre求积公式}
我们可以看到, 如果一味的提高求积公式的阶数, 有时候并不能获得很高的精度, 但是我们可以采用复合求积可以提高求积分的精度. 方法是将整个积分区间分成若干个子区间(通常是等分), 再在每个子区间上采用低阶的求积公式, 这样使整个区间的积分获得较高的精度.\par
本题中, 我们要求解决的是五点Gauss-Legendre求积的复合求积公式, 并要求将区间[1,3]分为4 个子区间, 因此有
\begin{equation}\label{gl_spec}
  \int_{1}^3 f(x)dx = \int_{1}^{\frac{3}{2}} f(x)dx + \int_{\frac{3}{2}}^{2}f(x)dx + \int_{2}^{\frac{5}{2}}f(x)dx +\int_{\frac{5}{2}}^{3}f(x)dx
\end{equation}
其中, $f(x) = \frac{1}{x^2}\text{sin}\frac{2\pi}{x}$.\par
接下来, 在每个小区间上,我们分别按照公式(\ref{vartrans})进行变量变换, 从而可以得到

\begin{table}[!htpb]
\begin{center}
\caption{变量转换的结果}
\begin{spacing}{1.5}
\begin{tabular}{ccc}
\toprule
$x\in[a, b]$ & $t\in[-1,1]$ & $x \rightarrow t$  \\
\midrule
$x\in[1, \frac{3}{2}]$ &  $t\in[-1, 1]$   & $x=\frac{1}{4}t + \frac{5}{4}$  \\
$x\in[\frac{3}{2}, 2]$ &  $t\in[-1, 1]$   & $x=\frac{1}{4}t + \frac{7}{4}$  \\
$x\in[2, \frac{5}{2}]$ &  $t\in[-1, 1]$   & $x=\frac{1}{4}t + \frac{9}{4}$   \\
$x\in[\frac{5}{2}, 3]$ &  $t\in[-1, 1]$   & $x=\frac{1}{4}t + \frac{11}{4}$  \\
\bottomrule
\end{tabular}
\end{spacing}
\end{center}
\end{table}
从而有方程变为
\begin{equation}\label{gl_spec}
\begin{split}
& \int_{1}^3 f(x)dx = \frac{1}{4}\int_{-1}^1 f\left(\frac{t}{4} + \frac{5}{4}\right)dt + \frac{1}{4}\int_{-1}^1 f\left(\frac{t}{4} +
 \frac{7}{4}\right)dt \\
 & \quad \quad  \quad \quad \quad + \frac{1}{4}\int_{-1}^1 f\left(\frac{t}{4} + \frac{9}{4}\right)dt + \frac{1}{4}\int_{-1}^1 f\left(\frac{t}{4} +\frac{11}{4}\right)dt\\ %=I_1+I_2+I_3+I_4  \\
\end{split}
\end{equation}
其中, $f(x)=\frac{1}{x^2}\text{sin}\frac{2\pi}{x}$.
根据Table \ref{node_coef}中的数据, 对于式(\ref{gl_spec})中右边的每一项,有%$I_i,i=1,2,3,4$(由于其前面的系数相同), 都
\begin{equation}
\begin{split}
& \frac{1}{4}\int_{-1}^1 f\left(\frac{t}{4} + \frac{5}{4}\right)dt = \frac{1}{4}\left[0.2369268851 \times \frac{16}{(5+0.9061798459)^2}\text{sin}\frac{8\pi}{5+0.9061798459}\right.\\
& \quad \quad  \quad \quad \quad  \quad \quad \quad  \quad+ 0.2369268851 \times \frac{16}{(5-0.9061798459 )^2}\text{sin}\frac{8\pi}{5-0.9061798459} \\
& \quad \quad  \quad \quad \quad  \quad  \quad \quad  \quad+0.4786286705 \times \frac{16}{(5+0.5384693101)^2}\text{sin}\frac{8\pi}{5+0.5384693101} \\
& \quad \quad  \quad \quad \quad  \quad \quad \quad  \quad+ 0.4786286705\times  \frac{16}{(5-0.5384693101)^2}\text{sin}\frac{8\pi}{5-0.5384693101} \\
&\quad \quad  \quad \quad \quad  \quad\quad \quad  \quad \left.0.5688888889\times \frac{16}{(5+0)^2}\text{sin}\frac{8\pi}{5+0}\right] \\
\end{split}
\end{equation}

\begin{equation}
\begin{split}
& \frac{1}{4}\int_{-1}^1 f\left(\frac{t}{4} + \frac{7}{4}\right)dt = \frac{1}{4}\left[0.2369268851 \times \frac{16}{(7+0.9061798459)^2}\text{sin}\frac{8\pi}{7+0.9061798459}\right.\\
& \quad \quad  \quad \quad \quad  \quad \quad \quad  \quad+ 0.2369268851 \times \frac{16}{(7-0.9061798459 )^2}\text{sin}\frac{8\pi}{7-0.9061798459} \\
& \quad \quad  \quad \quad \quad  \quad  \quad \quad  \quad+0.4786286705 \times \frac{16}{(7+0.5384693101)^2}\text{sin}\frac{8\pi}{7+0.5384693101} \\
& \quad \quad  \quad \quad \quad  \quad \quad \quad  \quad+ 0.4786286705\times  \frac{16}{(7-0.5384693101)^2}\text{sin}\frac{8\pi}{7-0.5384693101} \\
&\quad \quad  \quad \quad \quad  \quad\quad \quad  \quad \left.0.5688888889\times \frac{16}{(7+0)^2}\text{sin}\frac{8\pi}{7+0}\right] \\
\end{split}
\end{equation}

\begin{equation}
\begin{split}
& \frac{1}{4}\int_{-1}^1 f\left(\frac{t}{4} + \frac{9}{4}\right)dt = \frac{1}{4}\left[0.2369268851 \times \frac{16}{(9+0.9061798459)^2}\text{sin}\frac{8\pi}{9+0.9061798459}\right.\\
& \quad \quad  \quad \quad \quad  \quad \quad \quad  \quad+ 0.2369268851 \times \frac{16}{(9-0.9061798459 )^2}\text{sin}\frac{8\pi}{9-0.9061798459} \\
& \quad \quad  \quad \quad \quad  \quad  \quad \quad  \quad+0.4786286705 \times \frac{16}{(9+0.5384693101)^2}\text{sin}\frac{8\pi}{9+0.5384693101} \\
& \quad \quad  \quad \quad \quad  \quad \quad \quad  \quad+ 0.4786286705\times  \frac{16}{(9-0.5384693101)^2}\text{sin}\frac{8\pi}{9-0.5384693101} \\
&\quad \quad  \quad \quad \quad  \quad\quad \quad  \quad \left.0.5688888889\times \frac{16}{(9+0)^2}\text{sin}\frac{8\pi}{9+0}\right] \\
\end{split}
\end{equation}

\begin{equation}
\begin{split}
& \frac{1}{4}\int_{-1}^1 f\left(\frac{t}{4} + \frac{11}{4}\right)dt = \frac{1}{4}\left[0.2369268851 \times \frac{16}{(11+0.9061798459)^2}\text{sin}\frac{8\pi}{11+0.9061798459}\right.\\
& \quad \quad  \quad \quad \quad  \quad \quad \quad  \quad+ 0.2369268851 \times \frac{16}{(11-0.9061798459 )^2}\text{sin}\frac{8\pi}{11-0.9061798459} \\
& \quad \quad  \quad \quad \quad  \quad  \quad \quad  \quad+0.4786286705 \times \frac{16}{(11+0.5384693101)^2}\text{sin}\frac{8\pi}{11+0.5384693101} \\
& \quad \quad  \quad \quad \quad  \quad \quad \quad  \quad+ 0.4786286705\times  \frac{16}{(11-0.5384693101)^2}\text{sin}\frac{8\pi}{11-0.5384693101} \\
&\quad \quad  \quad \quad \quad  \quad\quad \quad  \quad \left.0.5688888889\times \frac{16}{(11+0)^2}\text{sin}\frac{8\pi}{11+0}\right] \\
\end{split}
\end{equation}

这样就可以得到用计算得到了用五点Gauss-Legendre积分公式的复合公式的计算公式.

\subsection{Romberg积分}
Romberg 求积方法是一种数值积分的加速算法, 它可以看成是Richardson 外推方法的一种特例, 下面将Richardson外推方法介绍如下:
假定$F(h)$, 当$h\rightarrow 0$时有$F(h)\rightarrow F^*(F^*$与$h$无关), 并有
\begin{equation}\label{Richardson1}
  F^*-F(h)=\sum^{\infty}_{k=1} \alpha_kh^{p_k}, 0<p_1<p_2<\cdots
\end{equation}
其中$p_k,\alpha_k$为与$h$无关的常数, $\alpha_k\neq 0, h \geqslant 1$,则由

\begin{equation}\label{Richardson_ite}
 \begin{split}
 & F_1(h)=F(h) \\
 & F_{m+1}(h) = \frac{F_m(qh)-q^{p_m}F_m(h)}{1-q^{p_m}}, m=1, 2,\cdots
\end{split}
\end{equation}
确定的序列$\{F_m(h)\}$有

\begin{equation}\label{Richardsonerr}
  F^*-F(h)=\sum^{\infty}_{k=1} \alpha_{m+k}^{(m+1)}h^{p_{m+k}},
\end{equation}
式中$\alpha_{m+k}^{(m+1)}$为与$h$无关的非零常数, $0<q<1$. \\

特别地, 如果我们在Richardson外推算法中取$q=\frac{1}{2}, p_k=2k$, 则可以得到 Romberg 求积方法, 即
\begin{equation}\label{Romberg1}
 \begin{split}
 & (T_1f)(h)=\frac{h}{2}\sum_{m=1}^n[f(x_{m-1})+f(x_m)], \\
 & (T_{j+1}f)(h) = \frac{4^j(T_jf)\left(\frac{h}{2}\right) -(T_jf)(h)  }{4^j-1}, j=1, 2,\cdots
\end{split}
\end{equation}

在Romberg积分中, 我们可以引入记号
$$T_j^kf=(T_jf)\left(\frac{h}{2^k}\right),j=1,2,\cdots$$
此称为Romberg数列, 在未外推时(即$T_1^k$), k表示在复合梯形公式中区间[a, b]分成$2^k$等份, 此时有(\ref{Romberg1})式可以表示为
\begin{equation}\label{Romberg2}
T_{j+1}^kf = \frac{4^jT_j^{k+1}f-T_j^kf}{4^j-1}
\end{equation}
从而, Romberg求积方法的计算过程如下:
\begin{enumerate}[(1)]
\item 求梯形面积$T_1^0f=\frac{h}{2}[f(a)+f(b)], h=b-a.$
\item 将区间[a,b]分半, 求出两个小梯形面积之和, 记为$T_1^1f$, 应用公式(\ref{Romberg2})有
$$T_{2}^0f = \frac{4T_1^1f-T_1^0f}{4-1}\quad\text{Simpson求积公式}$$
置$l=1$, 转入(4).
\item 对区间[a,b]作$2^l$等分,其符合梯形求积值记为$T_1^lf$, 构造数列
$$T_{j+1}^{k-1}f = \frac{4^jT_j^{k}f-T_j^{k-1}f}{4^j-1}$$
由此可得$T_{l+1}^0f.$
\item 若$|T_l^0f-T^0_{l+1}f|<varepsilon$(预先给定的误差控制), 则停止计算, 否则转(3).
\end{enumerate}
当误差小于$\varepsilon$时, 我们就取$T^0_{l+1}$的值作为定积分的值.在本题中, 我们取$\varepsilon= 10^{-7}$, 然后计算出最后的结果.

\section{方案设计}
我们首先编写了一个主函数quadrature.m, 在里面定义了区间[a,b], 函数表达式等信息, 然后分别调用相应的函数进行积分求解. 其中, 调用CompoundGLfive.m 函数 计算五点Gauss-Legendre积分公式的复合公式的计算结果, 调用Romberg.m来计算Romberg积分的值. 最后计算了二者结果各自的绝对误差和相对误差.\par
针对第一个问题, 即五点Gauss-Legendre积分公式的复合公式, 我们编写了一个函数来计算. (文件名为 CompoundGLfive.m ) 该函数主要就是计算五点Gauss-Legendre 积分复合公式在区间[a, b] 上的值, 其中它有四个参数, 分别是a(区间端点最小值)、b(区间端点最大值)、n(复合求积中区间分的段数)、f(要求积的函数的表达式, 以字符串的形式). 这样我们就可以返回求积计算的结果了. 在程序中, 我们主要用符号函数表示了要求积分的函数, 并且通过定义t, 完成了变量替换, 构成了新的积分的函数, 然后代入五点Gauss-Legendre求积公式的节点值和系数值, 这样就求出了一个小区间上的积分的值. 最后在外层循环中, 将所有的区间段上的积分加和, 就得到了最终的计算结果. 调用格式为 CompoundGLfive(1, 3, 4, '1/x\^{}2*sin(2*pi/x)').\par

针对Romberg积分, 我们首先编写了函数CompoundTrapezoid.m来计算复合梯形公式的积分, 其中该函数有4个参数, 分别是f,a,b,n. 其中f是要求积分的函数(以字符串的形式表示),a、b分别为区间[a,b]的端点, n 为复合求积中区间[a,b]要分的份数. 在函数中, 同样, 我们采用了Matlab内置的函数将字符串表示的函数转化为句柄函数.该函数的返回值即为复合积分的结果.\\
接下来, 我们按照前面介绍的Romberg求积算法的流程编写了函数Romberg.m来计算Romberg积分, 其中, 在该函数中, 我们调用了CompoundTrapezoid.m函数来计算中间用到的复合梯形求积的值. Romberg.m函数共有4个参数, 分别为a, b, epsilon, f. 其中, a和b为区间[a,b]的端点值, epsilon为控制误差, f 为要积分的函数, 以字符串的形式表达. 在函数中, 我们通过一个while循环对算法的整个流程进行控制, 当误差小于epsilon时程序停止. 同时, 需要特别指出的是, 在程序中, 我特别引入了一个矩阵T, 将Romberg积分中的每一个中间结果都记录了下来.

\section{计算结果及其分析}
计算的结果如下(真值为$I(f) = -0.238732414637843\cdots$):
\begin{enumerate}[(1)]
\item 当我们采用五点Gauss-Legendre公式的复合公式进行计算, 并将区间[1, 3]分为四个小段后, 计算的结果为$I(f)\approx -0.238732340666179$, 并且可以得出该方法的绝对误差为$7.3971663683281\times 10^{-8}$, 相对误差为$3.09851780268281\times 10^{-7}$.
\item 当我们采用Romberg积分时, 我们得到的积分结果为$I(f)\approx -0.238732414621623 $, 并且每一步计算的结果如Table \ref{romberg_res} 所示, 并且可以得出该方法的绝对误差为$1.62196922559588\times 10^{-11}$, 相对误差为$6.79408880464097\times 10^{-11}$.
\end{enumerate}
\begin{table}[!htpb]
\begin{tiny}
\begin{center}
\begin{spacing}{1}
\caption{Romberg积分的结果}\label{romberg_res}
\begin{tabular}{ccccccccc}
\toprule
$k$ & $T_1^k$ & $T_2^k$  & $T_3^k$  & $T_4^k$  & $T_5^k$  & $T_6^k$  & $T_7^k$  & $T_8^k$  \\
\midrule
0	&	0.096225045	&		&		&		&		&		&		&		\\
1	&	0.048112522	&	0.032075015	&		&		&		&		&		&		\\
2	&	-0.121371008	&	-0.177865519	&	-0.191861554	&		&		&		&		&		\\
3	&	-0.206400287	&	-0.23474338	&	-0.238535237	&	-0.239276089	&		&		&		&		\\
4	&	-0.230583448	&	-0.238644501	&	-0.238904576	&	-0.238910438	&	-0.238909005	&		&		&		\\
5	&	-0.23669501	&	-0.238732198	&	-0.238738044	&	-0.238735401	&	-0.238734714	&	-0.238734544	&		&		\\
6	&	-0.238223125	&	-0.238732497	&	-0.238732516	&	-0.238732429	&	-0.238732417	&	-0.238732415	&	-0.238732414	&		\\
7	&	-0.238605097	&	-0.238732421	&	-0.238732416	&	-0.238732415	&	-0.238732415	&	-0.238732415	&	-0.238732415	&	-0.238732415	\\


\bottomrule
\end{tabular}
\end{spacing}
\end{center}
\end{tiny}
\end{table}
两种方法所得的数值结果对比如Table \ref{twomethods} 所示. 从中可以看出, 采用Romberg积分计算的误差较小.
\begin{table}[!htpb]
\begin{tiny}
\begin{center}
\begin{spacing}{1}
\caption{两种求积方法结果对比}\label{twomethods}
\begin{tabular}{ccccc}
\toprule
求积方法& 真值 & 计算所得结果  & 绝对误差  & 相对误差   \\
\midrule
五点Gauss-Legendre复合求积 & $-0.238732414637843\cdots$ & -0.238732340666179 & $7.3971663683281\times 10^{-8}$ & $3.09851780268281\times 10^{-7}$ \\
Romberg积分 & $-0.238732414637843\cdots$&  -0.238732414621623 & $1.62196922559588\times 10^{-11}$ & $6.79408880464097\times 10^{-11}$\\
\bottomrule
\end{tabular}
\end{spacing}
\end{center}
\end{tiny}
\end{table}



\section{结论}
无论是五点Gauss-Legendre复合积分公式, 还是Romberg积分, 二者都能在很高的精度下求得函数的积分, 但是从计算的结果我们也可以看出, 采用Romberg积分计算的相对误差更小, 更接近真值, 且收敛速度很快. 因此, 在实际中采用 Romberg 积分的效果较好.

\section{附录: Matlab程序}
\begin{scriptsize}
\begin{enumerate}[(1)]
\item \textbf{quadrature.m} \\
\begin{lstlisting}
% 数值积分和数值微分上机实验
% Qun Liu 2014-12-24
clear;
clc;
% 定义要计算的函数f
f='1/x^2*sin(2*pi/x)';
% 定义区间[a,b]
a = 1;
b = 3;

% 求五点Gauss-Legendre复合求积,分成4个子区间
n = 4;
fGL = CompoundGLfive(a, b, n, f);
% Romberg积分
% 定义误差
epsilon = 1e-7;
[fR, T] = Romberg(a, b, epsilon, f);
% 定义真值y
y=-0.238732414637843;
% 计算绝对误差和相对误差
fGL_abs = abs(fGL-y);
fGL_rel = fGL_abs /abs(y);
fR_abs = abs(fR-y);
fR_rel = fR_abs /abs(y);
disp(['五点Gauss-Legendre复合求积(分成' num2str(n) '段)的值为:' num2str(fGL,15) ', 绝对误差为' num2str(fGL_abs,15) ', 相对误差为' num2str(fGL_rel,15)])
disp(['Romberg积分的值为:' num2str(fR,15) ', 绝对误差为' num2str(fR_abs,15) ', 相对误差为' num2str(fR_rel,15)])
\end{lstlisting}


\item \textbf{CompoundGLfive.m} \\
调用格式为: sum=CompoundGLfive(1, 3, 4, '1/x\^{}2*sin(2*pi/x)');
\begin{lstlisting}
function sum = CompoundGLfive(a, b, n, f)
% Gauss-Legendre Quadrature
% 采用五点Gauss-Legendre求积公式
% 其中, a, b为区间[a, b]的端点值, n为复合求积中要分的段数, f为要求积分的函数
% Qun Liu 2014-12-24
h = (b-a)/n;
aa = a : h : b-h;
bb = a+h : h : b;
% 积分的累积结果
sum = 0;
syms t
for i = 1 : n
    x2t = 0.5*( aa(i)+bb(i) ) + 0.5*( bb(i)-aa(i) )*t;
    f = subs(f, 'x', x2t);
    f = eval(['@(t)',vectorize(f)]);
    sum = sum + (bb(i)-aa(i))/2*0.5*(bb(i)-aa(i)) * ( f(0.9061798459)*0.2369268851...
        + f(-0.9061798459)*0.2369268851 + f(0.5384693101)*0.4786286705...
        + f(-0.5384693101)*0.4786286705+ f(0)*0.5688888889 );
end
\end{lstlisting}


\item \textbf{CompoundTrapezoid.m} \\
\begin{lstlisting}
function r=CompoundTrapezoid(f,a,b,n)
% Computing Compound Trapezoid Integration);
% f is a string containing x, denote a function
% Example: 'x/(4+x^2)'
% a,b are the start and end point of the interval
% n is the number of small segments of interval
% Qun Liu 2014-12-20

f=eval(['@(x)',vectorize(f)]);
h = (b-a)/n;
r = h/2*(f(a)+2*sum(f(a+h:h:b-h))+f(b));
\end{lstlisting}

\item \textbf{Romberg.m} \\
\begin{lstlisting}
function [f, T] = Romberg(a, b, epsilon, f)
% 计算函数f的Romberg积分
% a, b 为区间端点值,epsilon为控制误差, f为关于x的函数, 字符串形式
% f 为计算的积分的结果, T 为计算过程中所有的中间结果
% Qun Liu 2014-12-24

L=1; % 区间[a,b]要分割的份数
T(1,1)=CompoundTrapezoid(f,a,b,L);
T(2,1)=CompoundTrapezoid(f,a,b,L+1);
T(1,2) = (4*T(2,1) - T(1,1)) / (4-1);
while(abs(T(1,L)-T(1,L+1)) > epsilon)
    L = L+1;
    T(L+1,1) = CompoundTrapezoid(f,a,b,2^L);
    for j=1:L
        k= L-j+1;
        T(k, j+1) =  (4^j*T(k+1, j) - T(k, j)) / (4^j-1);
    end
end
f = T(1, L+1);
for j=2:L+1
    T(j:L+1, j)=T(1:L+2-j, j);
    T(1:j-1,j)=0;
end
\end{lstlisting}


\end{enumerate}
\end{scriptsize}
\end{CJK*}
\end{spacing}
\end{document}
